% Initial and revised submissions should be 12 point; this will be removed in the final version.
\documentclass[12pt]{TD-CJS}

% Initial and revised submissions should also be double spaced.  This command will be removed in the final version.
\renewcommand{\baselinestretch}{2}

\usepackage{latexsym}
\usepackage{amsmath}
\usepackage{amsfonts}
\usepackage{amssymb}
\usepackage{psfrag}
\usepackage{graphicx}

\begin{document}
\firstpage{1}
\lastpage{25}
\jvol{xx}
\issue{yy}
\jyear{20??}
\jid{CJS}
\aid{???}
% The running head contains the author names
\rhauthor{BlindedA and BlindedB}
\copyrightline{Statistical Society of Canada}
\Frenchcopyrightline{Soci\'et\'e statistique du Canada}
% History: received and accepted dates
\received{\rec{9}{July}{2009}}
\accepted{\acc{8}{July}{2010}}

% User-defined commands go here
\renewcommand{\eqref}[1]{(\ref{#1})}
\newcommand{\mb}[1]{\mathbf{#1}}
\newcommand{\mbb}[1]{\mathbb{#1}}
\newcommand{\mt}[1]{\mathrm{#1}}
\newcommand{\rv}{random variable}

% Title, authors, affiliations
\title[]{Title on one or two lines without capitals, except after a colon}%\query{Q1}
\author{BlindedA\authorref{1}\thanksref{*}}
\author{BlindedB\authorref{2}}
\affiliation[1]{Author affiliations will go here in the accepted manuscript, 
but do NOT include them in your initial submission because it must be anonymous.}
\affiliation[2]{Second Affiliation}

% Abstract, keywords, and classification codes
\startabstract{%
\keywords{
\KWDtitle{Key words and phrases}
Hidden Markov Models\sep biologging\sep multi-state model\sep missing data\sep pairwise
likelihood\sep repeated measurements.
% MSC 2010 subject classification codes
\KWDtitle{MSC 2010}Primary 62M05 \sep secondary 62P12}
\begin{abstract}
\abstractsection{}
\abstractsection{Abstract}
Insert your abstract here; it should
typically be up to ten lines long. Avoid symbols as much as possible. Formulas
are strongly discouraged, and citations should be avoided. 
The title and the abstract should be concise and descriptive;
list the key words in alphabetical order.  The MSC 2010 subject classification codes can be found here:
http://www.ams.org/mathscinet/msc/pdfs/classifications2010.pdf.
\abscopyright
\fabstractsection{}
\fabstractsection{R\'{e}sum\'{e}}
Ins\'{e}rer votre r\'{e}sum\'{e}
ici. We will supply a French abstract for those authors who
can't prepare it themselves.\Frenchabscopyright
\end{abstract}}
\makechaptertitle

% Email address for corresponding author
\correspondingauthor[*]{\\\email{Insert your email address here only after your paper has been accepted}}

\section{INTRODUCTION}
Your text starts here. For English spelling,
we follow the style of the Canadian Oxford Dictionary (Barber, 2004).  
If the dictionary lists more than one acceptable spelling, choose the main entry.

Manuscripts may be submitted in either English or French.  We do accept submissions in Microsoft Word.  We do not provide a Word template, and we ask you to follow the general instructions in this document.

\subsection{Subsections start this way}
Refer to papers by authors (date) throughout. For example,
you might quote Author1 \& Author2 (1986a,~b) or Author (1987, 1992) or 
Author1, Author2, \& Author3 (2011). If there
are four authors or more, refer to (for example) Author1 et al.\ (1990).

When references are in parentheses, separate them by a semicolon and do not
put brackets around the year (e.g., Author1 \& Author2, 1994; Author3 \&
Author4, 1999).

\section{MATHEMATICAL TYPESETTING}
Symbols should not be used at the start of a sentence, and footnotes are not allowed. Try to avoid double subscripts, and never use triple subscripts. Unless central or essential to the flow of the discussion, mathematical arguments should be deferred to the Appendix.
Note that equations should be numbered consecutively, i.e., (1), (2), etc. Number
\textit{only} those equations that are referred to in the text.  Punctuation should be given after equations. Equations should be cited as, for example, Equation (6) or Equations (6)--(8). Within parentheses use an abbreviation: (Eq. 7).

Likewise, number consecutively your definitions, lemmas, propositions,
theorems, corollaries and the like. For example:

\begin{theorem}{Theorem 1.}{}%
Here is the statement of our theorem.
\end{theorem}
\begin{proof}{Proof}{}%
We prove our theorem using Equation (\ref{Einstn}) below:
\begin{equation}
e = mc^2.
\label{Einstn}
\end{equation}
\end{proof}

\section{FIGURES AND TABLES}
When you submit the final version of your manuscript in
\LaTeX \, form, please include postscript files (.ps or .eps) for the figures,
labelling them fig1.ps, fig2.ps, etc. When referring to a figure, spell out the word (e.g., Figure 1) whether or not it is in parentheses.

% Instructions for a figure
%\begin{figure}
%\centerline{\epsfbox{gr1.eps}}
%\caption{This is a full width figure}
%\end{figure}

See Table 1 for an example table layout; note that we do not use
vertical lines between columns.  The data in this table are from Genest (1999).

\begin{table}
% Next comes the table caption
\tbl{Top 10 countries for gross national publication (\textsc{gnp}) of
research in statistics. The ranks are based on variable \textsc{pag}$^{\star}%
$.}{%
\begin{tabular*}{30pc}{@{\hskip5pt}@{\extracolsep{\fill}}r@{}c@{}r@{}r@{}r@{}r@{}r@{}r@{}@{\hskip5pt}}
\toprule
Rank & Country & \textsc{pag}$^{\star}$ & \textsc{pag} & \textsc{art}$^{\star
}$ & \textsc{art} & $\frac{\mbox{{\sc pag}}} {\mbox{{\sc art}}}$ &
$\frac{\mbox{{\sc aut}}}{\mbox{{\sc art}}}$\\
\colrule
1 & \textsc{usa} & 109338 & 60369 & 7240 & 4061 & 14.9 & 1.83\\
2 & United Kingdom\tref{a} & 12597 & 7504 & 884 & 538 & 14.1 & 1.81\\
3 & Canada & 12407 & 6837 & 909 & 516 & 13.6 & 1.89\\
4 & Australia & 7872 & 4261 & 578 & 323 & 13.5 & 1.95\\
5 & Germany & 6782 & 4500 & 456 & 306 & 14.9 & 1.63\\
6 & France & 3647 & 1843 & 261 & 129 & 14.5 & 2.18\\
7 & Japan & 2865 & 1880 & 241 & 163 & 11.6 & 1.60\\
8 & Netherlands & 2864 & 1702 & 191 & 116 & 15.1 & 1.80\\
9 & India & 2559 & 1395 & 275 & 151 & 9.5 & 1.91\\
10 & Israel & 2097 & 1160 & 148 & 83 & 14.5 & 1.99\\
\botrule
\end{tabular*}}
\begin{tabnote}
\tblfno{a}This is a footnote to the table.
\end{tabnote}
\end{table}

\begin{ack}{ACKNOWLEDGEMENTS}
Place all acknowledgements here.  In your initial and revised submission, ensure that any acknowledgements are anonymous; include the full acknowledgements only after your paper has been accepted. Granting agencies should not be abbreviated, and do not include grant numbers.  We are grateful for your assistance with our publication process.\\
\\
A note on references: they should be listed alphabetically. Initials should be separated by blank characters and
journal names should be {\it in italics}. Don't use boldface for volume
number and separate pages by two dashes.
Abbreviations should not be used for authors' or journal's names, nor for the titles of the articles.
\end{ack}

\begin{thebibliography}{}
% For English spelling, we follow the style of this dictionary
\bibitem[]{bib1}
Barber, K. (Ed.) (2004). {\it The Oxford Canadian Dictionary}, 2nd ed., 
Oxford University Press, Toronto.

% Single-author journal article
\bibitem[]{bib2}
Cleveland, W. (1979). Robust locally-weighted regression and smoothing scatterplots. 
{\it Journal of the American Statistical Association}, 74, 829-836.

% Genest provided the data for our sample table
\bibitem[]{bib3}
Genest, C. (1999). Probability and statistics: A tale of two worlds? 
{\it The Canadian Journal of Statistics}, 27(2), 421--444.

% Multiple-author journal article.  All the author names are separated by commas, and 
% note the uppercase letter after the colon in the title.
\bibitem[]{bib4}
Johnes, J., Taylor, J., \& Francis, B. (1993). The research
performance of \textsc{uk} universities: A statistical analysis of the results
of the 1989 research selectivity exercise. {\it Journal of the Royal
Statistical Society Series A}, 156, 271--286.

% Book in French.  Be careful with accents.
\bibitem[]{bib5}
Kidron, M. \& Segal, R. (1992). {\it Atlas du Nouvel \'Etat
du Monde}. \'Editions Autrement, Paris.

% Book, two authors, second edition
\bibitem[]{bib6}
Little, R.~J.~A. \& Rubin, D.~B. (2002). {\it Statistical Analysis with Missing Data}, 2nd ed., 
John Wiley \& Sons, New York.

% Unpublished paper with website link
\bibitem[]{bib7}
Molina, I. \& Rao, J.~N.~K. (2009). Small area estimation of poverty indicators. Working Paper 09-15, Statistics and Econometric Series 05, Universidad Carlos III de Madrid, \\http://hdl.handle.net/10016/5644.

% Edited book
\bibitem[]{bib8}
Roberts, G., Ren, Q., \& Rao, J.~N.~K. (2009). 
Using marginal mean models for data from longitudinal surveys
with a complex design: Some advances in methods. In Lynn, R. (Ed.) {\it Methodology of Longitudinal Surveys}, 
John Wiley \& Sons, New York, 351-366.
\end{thebibliography}

\begin{appendix}
There should be just one appendix, for proofs and longer
mathematical arguments. These proofs are in the following form:

\begin{proof}{Proof of Theorem 1}{}%
We now prove the two parts of Theorem 1.
\end{proof}
\end{appendix}

\CJShistory
\end{document}
