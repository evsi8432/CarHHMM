% !TeX root = ../main.tex

\section{Capturing Non-Markovian Behavior via the Fourier Transform}

Although the CarHMM and continuous-time model can effectively capture autocorrelation within a process, both still assume Markovian dynamics, i.e. that the observation $Y_t$ depends only the value of the behaviorial state $X_t$ and the previous observation $Y_{t-1}$. However, there are many animal movement process which violate the markov propertey. In particular, on a fine scale, swimming behavior of marine mammals can exhibit periodic behavior as the animal repeatedly flukes to propel itself forward. Work has been done in the past to model non-markovian dynamics in the \textit{behavioral} process \cite{Langrock:2012}, but addressing non-markovian dynamics within observations $Y*$ is still a relatively unstudied area. With improvements in tagging technology allowing for data collection at very high frequencies, noisy and non-markovian fine scale behavior appears to be on the rise.

To address this issue, we recommend borrowing techniques from the signal processing literature to compress the data and summarize its essential elements. In particular, we suggest a method which can capture periodic behavior by decompsing a signal into its fourier components.

Suppose a one-dimensional fine-scale process of length $T^*$, $y^*$ exhibits significant periodic and structured behavior that is clearly non-markovian. It is possible to transform this fine-scale process $y^*$ to a shorter, but higher dimensional process $\hat y^*$ by selecting a window of size $w$, dividing $y^*$ into $\lfloor T^* / w \rfloor$ seperate intervals of length $w$ (truncating $y^*$ appropriately), and taking the discrete fourier transform (DFT) of each interval. The resulting time series $\hat y^*$ will be a complex-valued, $w$-dimensional time series of length $\lfloor T^* / w \rfloor$.  

Picking the window length $w$ should be done with care. $w$ should be long enough to capture the periodic behavior of the underlying process (at least twice as long as the length of a period), but short enough so that the resolution of the process remains high and behavioral changes can be captured via the HMM.

A visualization of transforming a one-dimensional sequence $y^*$ to a $w$-dimensional complex sequence $\hat y^*$ can be seen in figure (\ref{fig:fourier_example}).