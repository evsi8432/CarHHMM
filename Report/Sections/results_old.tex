% !TeX root = ../main.tex

%\section{Killer Whale Case Study}

%In order to test the advantages of the CarHHM-DFT over the other models on real-world data, we used the four models described in the simulation study to analyze dive data from a Northern Resident Killer Whale (NRKW) off the coast of British Columbia, Canada. 

%\section{Results}

We fit all four models from the simulation study to the data shown in Figure \ref{fig:data} to test their performance in a real-world setting. We first report the results from the full CarHHMM-DFT (defined in Section \ref{sec:data}) in detail and then asses the quality of the fit. We also compare the results of the CarHHMM-DFT with those of the other candidate models.

The emission distribution parameter estimates suggest that the killer whale has at least two distinct dive behaviours (Table \ref{table:emis_dists_CarHHMM-DFT}, Figure \ref{fig:coarse_emis}). 
Dive type 1 corresponds to shorter, shallower dives which may serve a variety of purposes including resting, travelling, and searching for prey.
On the other hand, dive type 2 is longer and deeper, and these dives have been associated with behaviours such as foraging and socialization \citep{Tennessen:2019b}. No dive in this data set has a maximum depth deeper than $\approx$ 30 meters, and \citet{Wright:2017} note that most prey captures occur at depths deeper than 100m. It is therefore unlikely that any of the dives used in this study are successful foraging dives.

The three subdive behaviours have distinct average ``wiggliness" values, as the means of $\Ztwo$ for each state are separated by an order of magnitude (Table \ref{table:emis_dists_CarHHMM-DFT}, Figure \ref{fig:fine_emis}). 
Subdive state 1 has the smallest mean of $\Ztwo$ and the smallest variance corresponding to $\Zone$. It also has the highest auto-correlation in $\Zone$. This implies less overall activity and more consistent acceleration compared to the other subdive states. 
Subdive state 2 has a mean ``wiggliness" ($\Ztwo$) that is one order of magnitude higher than subdive state 1 and its acceleration has about twice the variance compared to subdive state 1. The auto-correlation of acceleration is also slightly lower in than subdive state 1. We therefore hypothesize that subdive type 2 corresponds to fluking (active swimming), as strong sinusoidal behaviour in acceleration is characteristic of this behaviour in marine mammals \citep{Simon:2012}.
Finally, the mean of $\Ztwo$ and variance of $\Zone$ in subdive state 3 is much higher than the other two states. The auto-correlation of $\Zone$ is also much lower, implying more variation in acceleration between segments. This corresponds to vigorous swimming activity, especially as the killer whale begins or ends a dive (see Figure \ref{fig:labeled_dives}). 

The estimated probability transition matrices and associated stationary distributions are
%
$$\hat \Gamma = \begin{pmatrix} 
0.788 & 0.212 \\
0.809 & 0.191
\end{pmatrix} \text{ and }$$
$$\hat \delta = \begin{pmatrix} 0.792 & 0.208 \end{pmatrix}$$
%
for the transitions between dives. Within dives, we have
$$\hat \Gamma^{*(1)} = \begin{pmatrix} 
0.679 & 0.321 & 0.000 \\
0.038 & 0.904 & 0.058 \\
0.000 & 0.232 & 0.768
\end{pmatrix}, \qquad 
\hat \Gamma^{*(2)} = \begin{pmatrix} 
0.859 & 0.141 & 0.000 \\
0.114 & 0.841 & 0.045 \\
0.000 & 0.216 & 0.784
\end{pmatrix},$$
$$\hat \delta^{*(1)} = \begin{pmatrix} 0.087 & 0.731 & 0.182 \end{pmatrix}, \enspace \text{and} \enspace \hat \delta^{*(2)} = \begin{pmatrix} 0.401 & 0.496 & 0.103 \end{pmatrix}$$
%
for dive types 1 and 2.
The estimated coarse-scale stationary distribution, $\hat{\delta}$, indicates that most dives are short, type 1 dives. The estimated coarse-scale transition probability matrix $\hat \Gamma$ further indicates that the whale usually performs multiple short type 1 dives before doing only one (or a few) long type 2 dive. This finding is consistent with those of \citet{Tennessen:2019b} and \citet{Williams:2009}, both of whom describe common bouts of short resting dives before a killer whale performs a longer, more energy-intensive deep dive. The fine-scale stationary distributions $\hat{\delta}^{*(i^*)}$ indicate that this killer whale is more likely to be in the less active subdive state 1 when performing long deep dives than when performing short shallow dives. Using less active swimming behaviour is consistent with the need for marine mammals to conserve energy when diving at depth and holding breath for long periods of time \citep{Williams:1999,Hastie:2006}. Figure \ref{fig:labeled_dives} shows the decoded dive behaviour of six selected dives. The supplementary material also shows the estimated probability of each dive and subdive state given the data (Section S-2.2).

\subsection{Model Validation}
\label{subsec:model_validation}

Two visual tools are used to evaluate this model: pseudo-residuals and empirical histograms. A pseudo-residual is the marginal cumulative distribution function of an observation conditioned on all other observations \citep{Zucchini:2016}. To easily visualize outliers, pseudo-residuals are often passed through the quantile function of the standard Normal distribution. Mathematically, the pseudo-residual of an observation $y_t$ of a traditional HMM is equal to $\Phi^{-1} \left(Pr(Y_t < y_t|\{Y_1,\ldots,Y_T\}/\{Y_t\}) \right)$, where $\Phi$ is the cumulative distribution function of a standard Normal distribution. If the model is correct, then all pseudo-residuals are independent and follow a standard Normal distribution. Histograms of the pseudoresiduals mostly support that this model is well-specified. One exception is $\Ztwo$, whose pseudo-residuals are noticeably right-skewed (Figure \ref{fig:pseudoresids}). This implies that the true distribution of $\Ztwo$ may follow a heavier-tailed distribution than the gamma distribution used in the case study. 

We also plot histograms of $Y$ and $Z^*$ corresponding to each dive type or subdive state in Figure \ref{fig:empirical_dist}. When constructing the histogram corresponding to each dive type or subdive state, observations are weighted by the decoded probability that they correspond to that hidden state. This allows each hidden state to be evaluated separately. These histograms are plotted over their corresponding emission distribution learned by the CarHHMM-DFT (see Figure \ref{fig:empirical_dist}). Our results again mostly support a well-specified model with the exception of $\Ztwo$, which is again right-skewed. Another exception is $\Zone$, which has heavy tails in subdive state 3, indicating the existence of rare events corresponding to exceptionally violent thrashing of the killer whale. These outliers are potential subjects for future study and may indicate biologically relevant phenomena such as prey capture \citep{Tennessen:2019a}. See Section S-2.? for histograms for every observation and hidden state. 

\subsection{Comparison with candidate models}

The CarHMM-DFT produces very similar parameter estimates and state estimates compared to those of the CarHHMM-DFT on the fine scale. However, its lack of a hierarchical structure means that it fails to differentiate between short and long dives. This model therefore does not give any information regarding the dive-level Markov chain or the relationship between the dive and subdive levels. For example, the CarHMM-DFT does not indicate that the whale is much more likely to be in subdive state 1 when engaged in longer dives compared to shorter one.

The HHMM-DFT decodes dive types and subdive states similarly to the CarHHMM-DFT, but it is less likely to categorize the behaviour at the beginning and end of dives as subdive type 3. In addition, the HHMM-DFT produces estimates of $\sigma_A^{*(1)}$, $\sigma_A^{*(2)}$, and $\sigma_A^{*(3)}$ (the standard deviation of acceleration) which are approximately 50-100 \% larger than those of the CarHHMM-DFT across all axes of acceleration. The estimated uncertainties of the estimates of $\mu_A^{*(1)}$, $\mu_A^{*(2)}$, and $\mu_A^{*(3)}$ are also less than half of those for the CarHHMM-DFT across all axes of acceleration. Further, the empirical auto-correlation of $\Zone$ within each decoded subdive state was at least 0.5 for all acceleration axes and subdive states, and the pseudoresiduals of the HHMM-DFT contain several outliers. These findings suggest that the HHMM-DFT is a significantly worse fit to this data than the full CarHHMM-DFT.

Finally, the CarHHMM does not use the ``wiggliness" of the acceleration data as an observation, so it regularly fails to pick up obvious behavioural changes corresponding to the periodicity shown in Figure \ref{fig:labeled_dives}. This fact alone essentially disqualifies the CarHHMM as a viable model for this data. The pseudoresiduals corresponding to the CarHHMM also appear to be less heavy-tailed than a normal distribution. See section S-2 for a more complete set of results for each of the candidate models.