% !TeX root = ../main.tex

%\section{Discussion}

Current Functional Data Analysis literature addresses dependence between curves either with multilevel models \citep{Chen:2012,Di:2009}, which lack a time component, or with functional time series, which overlook the possibility that curves have several distinct ``types" \citep{Kokoszka:2018}. Our work addresses these issues and introduces a flexible framework to model functional time-series data using HMMs.
We suggest handling temporal dependence between curves by using either an HMM or a CarHMM to model the curve sequence. We then suggest viewing each individual curve as an HMM emission whose distribution is described by a fine-scale model. Here we use a CarHMM as the fine-scale model, but there are a wide range of possible fine-scale models, including a Poisson process or continuous time approach similar to that of \citet{Michelot:2019}. We also incorporate a moving-window transformation at the fine-scale model to capture intricate dependence structures on short time scales. Together, the coarse- and fine-scale models make up a hierarchical structure which can account for simultaneous processes taking place at different time scales. Provided the construction is not overly complex, a hierarchical model created using our method can be both flexible and easy to fit using maximum likelihood estimation.

We demonstrate the usefulness of this framework using a biomechanical/ecological example, where we use HMMs to classify the coarse- and fine-scale diving behaviour of a northern resident killer whale in Queen Charlotte Sound off the coast of British Columbia, Canada. Our analysis gives a deeper understanding of a killer whale's tri-axial movement and thus its behaviour and energy expenditure \citep{Gleiss:2011,Qasem:2012}, both of which are important for understanding the foraging ecology and nutritional status of northern resident killer whales \citep{Noren:2011}. Our model is applicable to many diving animals such as sharks \citep{Adam:2019}, seals \citep{Dot:2016}, and porpoises \citep{Barajas:2017}. In addition, since complicated state-switching processes with temporal dependence are common in settings ranging from speech recognition \citep{Juang:1991} and neuroscience \citep{Langrock:2013} to oceanography \citep{Bulla:2012} and ecology \citep{Adam:2019}, we believe that researchers can adapt our methodology for the analysis of a wide range of time series data in a variety of fields.

\iffalse

Current FDA literature addresses dependence between curves either with multilevel models with random effects and no time component at all \citep{Chen:2012,Di:2009}, or with functional time series which model the curves as an evolving process \citep{Kokoszka:2018} which overlook the possibility that curves have several distinct ``types". However, many curve sequences exhibit both temporal dependence and a discrete number of types. One obvious example is the case study of killer whale behaviour examined here, where one dive can serve different purposes such as travelling, socializing, resting, or foraging \citep{Tennessen:2019a}. Another example is accelerometer data in school children, where bouts of activity can be associated with a behaviour such as running or walking. \citep{Morris:2007}. In this work, we deal with both temporal dependence between curves and distinct curve ``types" by using either an HMM or CarHMM. To help select the appropriate model between these two, lag plots can be used to test if there is significant auto-correlation between curves.

When used traditionally, each of these models require that practitioners take summary statistics of curves as emissions which may lack important information about the structure of the curve. However, HMM emissions can also be viewed as the entire curve, where the emission distribution of that curve is described by yet another fine-scale model. There are a wide range of possible fine-scale models, including, but not limited to, another HMM, a CarHMM, a Poisson process, or a spline-based approach similar to that of \citet{Langrock:2018}. The parameters of this fine-scale model then depend upon the curve type as determined by the coarse-scale model. Together, these coarse- and fine-scale models make up a hierarchical model which can account for simultaneous processes taking place at different time-scales.

One interesting class of possible fine-scale models is the set of continuous time methods such as the state-switching Ornstein-Uhlenbeck (OU) process \citep{Michelot:2019} and the continuous-time HMM (CTHMM) \citep{Liu:2015}, which are often based on stochastic differential equations. Unfortunately, most continuous-time models require MCMC algorithms to perform inference and can be slow to fit. However, we show the appendix that the state-switching OU process of \citet{Michelot:2019} is equivalent to a CarHMM under certain conditions. This equivalence implies that a CarHMM is interpretable as a continuous-time model, but has the added advantage of relative computational efficiency.

Another fine-scale model that we investigate in detail here is yet another HMM. If the fine-scale model is an HMM, then the resulting overall model is just the HHMM as introduced by \citet{Barajas:2017}. \citet{Adam:2019} in particular use the HHMM to analyze 25 Hz accelerometer data from horn sharks off the coast of southern California. This model can be inappropriate at very fine scales, as \citet{Adam:2019} find that subsequent fine-scale observations are not independent when conditioned on the hidden states due to periodic behaviour in the acceleration data. This indicates the need to account for complicated dependence structures within fine-scale processes, especially those with high sampling frequencies. To deal with these problematic fine-scale phenomena, we explore a moving-window transformation over the fine-scale observations. Since our case study data clearly exhibits periodic behaviour, we specifically use a DFT-based transformation. The effectiveness of this approach is evidenced by both our case study, where the CarHHMM without a DFT transformation visually fails to recognize fluking behaviour in a killer whale, and our simulation study, where the average accuracy associated with the subdive state is by far the worst when omitting $\Ztwo_{t,t^*}$ (``wiggliness'') from the model. After transforming the fine-scale observations, either subject area experts or exploratory analysis may suggest a large degree of temporal correlation. In this case, we suggest using a CarHMM instead of a traditional HMM to account for the auto-correlation in the transformed fine-scale observations. Our simulation study shows how ignoring this auto-correlation can result in underestimating standard errors and overestimating the variance of the observations. 

When using HMMs to build complex hierarchical models, it is important to build a model that is complex enough to fit the data but not so complex that it over-fits the data or is computationally inefficient. Model complexity inevitably grows rapidly as more layers are added to the hierarchical structure. For example, by far the fastest model to train in the simulation study is the CarHMM-DFT, as it has no hierarchical component and is the simplest of the four candidate models. The CarHMM-DFT also has near-perfect accuracy when decoding the simulated subdive behavioural states. However, this model does not allow ecologists to study the joint relationship between the dive type and intra-dive behaviour. One way to temper model complexity is to reduce the number of parameters within a given model structure. In particular, we assume that the fine-scale state parameters are the same for all coarse-scale states for all models in both our case study and simulation study.

%To illustrate the potential applications of this method, we model the behaviour of a northern resident killer whale animal behaviour using data collected from time depth recorders and accelerometers. Understanding the diving behaviour and energetic requirements of NRKWs is essential to understand their foraging behaviour and survival \citep{Williams:2009,Noren:2011}. However, it is necessary to classify the underlying activity \citep{Dot:2016} or dive type \citep{Hastie:2006} of an animal in order to effectively understand and estimate its energy expenditure from biologging data. The CarHMMM-DFT model uses coarse-scale data to separate dives into different types while also describing the distribution of acceleration within each subdive state on the fine scale.  

Our work provides a flexible framework to model functional time-series data using HMMs. Provided the model is not overly complex, a hierarchical model constructed using the proposed building blocks can be both flexible and easy to fit using maximum likelihood methods. We demonstrate the usefulness of such a model using an ecological example, where we use a complex HMM to classify the coarse- and fine-scale diving behaviour of a killer whale. Our analysis gives a deeper understanding of the animal's tri-axial movement and thus its fine-scale energy expenditure \citep{Gleiss:2011,Qasem:2012}, which is essential when discerning the survival of both NRKWs and their prey \citep{Noren:2011}. Although our case study focuses on northern resident killer whales, this model is likely applicable to many diving animals such as sharks \citep{Adam:2019}, seals \citep{Dot:2016}, and porpoises \citep{Barajas:2017}. In addition, since complicated state-switching processes with temporal dependence are common in settings ranging from speech recognition \citep{Juang:1991} and neuroscience \citep{Langrock:2013} to oceanography \citep{Bulla:2012} and ecology \citep{Adam:2019}, we believe that researchers can adapt this methodology for the analysis of a wide range of time-series data in a variety of fields.

\fi