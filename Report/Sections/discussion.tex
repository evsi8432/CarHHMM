% !TeX root = ../main.tex

%\section{Discussion}

Current functional data analysis literature addresses dependence between curves either with multi-level models \citep{Di:2009,Chen:2012}, which lack a time component, or with functional time series, which overlook the possibility that curves have several distinct ``types" \citep{Kokoszka:2018}. Our work addresses these issues and introduces a flexible framework to model functional time series data using HMMs.
We suggest handling temporal dependence between curves by using either an HMM or a CarHMM to model the curve sequence. We then suggest viewing each individual curve as an HMM emission whose distribution is described by a fine-scale model. Here we use a CarHMM as the fine-scale model, but there are a wide range of possible fine-scale models, including a Poisson process or a continuous time approach similar to that of \citet{Michelot:2019}. We also incorporate a moving-window transformation at the fine scale to capture intricate dependence structures. Together, the coarse- and fine-scale models make up a hierarchical structure that can account for simultaneous processes taking place at different time scales. Provided that the construction is not overly complex, a hierarchical model created using our method can be both flexible and easy to fit using maximum likelihood estimation. Our method is not intended to minimize prediction error for functional data, but incorporating HMMs into autoregressive predictive models such as those described in \citet{Aue:2015} and \citet{Gao:2019} is a promising and natural direction for future study. 

We demonstrate the usefulness of this framework using a biomechanical/ecological example, where we use HMMs to classify the coarse- and fine-scale diving behaviour of a northern resident killer whale in Queen Charlotte Sound off the coast of British Columbia, Canada. Our analysis gives a deeper understanding of the killer whale's tri-axial movement and thus its behaviour and energy expenditure \citep{Gleiss:2011,Qasem:2012}, both of which are important for understanding the foraging ecology and nutritional status of northern resident killer whales \citep{Noren:2011}. Our model is also applicable to many diving animals such as sharks \citep{Adam:2019}, seals \citep{Dot:2016}, and porpoises \citep{Barajas:2017}. In addition, since complicated state-switching processes with temporal dependence are common in settings ranging from speech recognition \citep{Juang:1991} and neuroscience \citep{Langrock:2013} to oceanography \citep{Bulla:2012} and ecology \citep{Adam:2019}, we believe that researchers can adapt our methodology for the analysis of a wide range of time series data in a variety of fields.