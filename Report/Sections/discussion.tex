% !TeX root = ../main.tex

%\section{Discussion}

Current FDA literature addresses sequences of curves either as multilevel models with random effects and no temporal dependence \citep{chen:2012,di:2009} or as functional time series which model the curves as an evolving process and overlook the possibility of a discrete number of curve ``types" \citep{Kokoszka:2018}. However, many real-world curve sequences exhibit both temporal dependence and a discrete number of types. One obvious example is the case study examined here, where a sequence of marine mammal dives serve a number of discrete of purposes such as traveling, socializing, resting, or foraging \citep{Tennessen:2019a}. Another example is accelerometer data in school children, where bouts of activity can be associated with running, walking, etc. \citep{Morris:2007}.

To deal with both temporal dependence between curves and a discrete number of curve types, we suggest modelling the sequence of curves with either an HMM or CarHMM, depending upon the level auto-correlation between curves. Several useful model selection tools such as a lag plot can be used to test if there is significant auto-correlation within a sequence of curves. One may use summary statistics of the curve as the observations of the HMM, but it is also possible to use yet another within-curve parametric model as the HMM emissions. The parameters of this fine-scale model then in turn depend upon the curve type. Together, the two levels make up a hierarchical model which can account for processes taking place simultaneously and at different time-scales.

%Traditional HMMs can be used to model a state-switching process with conditionally independent observations and Markovian dynamics when conditioned of the hidden state. However, many real-world processes require more complex models. Here, we presented a flexible framework that can account for complicated dependence structures within time-series data. Our framework is a collection of HMM models which can be combined together to form increasingly complex hierarchical models to match the complexity of particular problems faced by researchers.

%The CarHMM generalizes the HMM by explicitly modeling auto-correlation in the emission distributions of the HMM \citep{Lawler:2019}. CarHMMs also maintain the structure needed to evaluate the likelihood using the forward algorithm. In our Normal model formulation, we have added only one additional parameter, $\phi^{(i)}$, per possible hidden state.

Several possible choices exist for the fine-scale model, including yet another HMM. \citet{Barajas:2017} introduce this particular situation as an HHMM, and \citet{Adam:2019} use it to analyze 25 Hz accelerometer data from horn sharks off the coast of southern California. However, traditional HMMs prove insufficient at very fine-scales, as \citet{Adam:2019} run into clear violations of conditional independence and observe apparent periodic behavior in the acceleration data. This indicates the need to account for complicated dependence structures within fine-scale processes, especially those with high sampling frequencies. 

%For simultaneous observed processes taking place at different time scales, the HHMM \citep{Barajas:2017,Adam:2019} utilizes hierarchical structures to jointly model both as HMMs. In particular, each hidden state of the coarse-scale HMM is assumed to emit both an observation $Y_t$ as well as another fine-scale HMM with hidden states $X^*_t$ and observations $Y^*_t$.

%For processes such as these with very high sampling frequencies and/or intricate fine-scale structure, 

%To deal with these issues, it is possible to generalize the HHMM such that the fine-scale model can be any model which admits an easy-to-calculate likelihood. For example, if the sampling rate of the fine-scale process is very high and clearly period, then the fine-scale model can be described by a simple probability distribution over the summary statistics of a moving window of observations. 

To deal with these problematic fine-scale processes, we suggest using a moving-window transformation over the fine-scale observations. For periodic behaviour in particular, we use a DFT-based transformation. The effectiveness of this approach is evidenced by both our simulation study, where the subdive decoding accuracy is nearly perfect when including $Z^{*(2)}$ in the model, and our case study, where the CarHHMM without DFT visually fails to recognize fluking behaviour in a killer whale. On top of the DFT transformation, we also suggest using a CarHMM instead of a traditional HMM when expert opinion and careful observation (possibly using a lag plot) calls for it. After putting all of these building blocks together, we arrive at our final model: the CarHMM-DFT. This model effectively decodes both dive types and subdive states of marine mammals while also taking into account the connection between these two levels. 

This solves an important problem, as the energetic requirements of killer whales together with their energetic intake from predation is directly related to their survival \citep{Williams:2009,Noren:2011}. Previous studies have looked into the energy consumption of killer whales, but many rely on invasive and expensive methodology \citep{Williams:2009,Noren:2011}. Accelerometers are an appealing alternative, but require decoding the behavioral state of the animal in order to effectively estimate energy expenditure \citep{Dot:2016}. The CarHHMM-DFT solves this problem by decoding both the subdive behavioral state and the dive type of a killer whale. It also describes the distribution of acceleration in the states. All of this information meaningfully effects the energy expenditure of killer whales and is therefore valuable information for conservation efforts \citep{Williams:2009,Noren:2011}.

While we combine these models together effectively for our case study, in general this practice should be done with care. There are drawbacks to building arbitrarily complex models with HMMs. For example, HMMs require observations to be taken at regular intervals. Continuous-time methods such as the state-switching OU process \citep{Michelot:2019} are based on stochastic differential equations and resolve this issue, but we do not describe them in detail here. This is because most continuous-time models require relatively slow MCMC algorithms to perform inference and are prohibitively difficult to fit. In addition, we show in Theorem 1 (see appendix) that the state-switching OU process of \citet{Michelot:2019} is equivalent to an CarHMM under certain conditions. This implies that a CarHMM is interpretable like a continuous-time model, but as easy to fit as a traditional HMM.

Another consideration when using HMMs to build complex models is the importance of effectively capturing the process being modelled while avoiding over-fitting and slow parameter estimation. One way to temper model complexity is to share fine-scale state parameters across the coarse-scale states (i.e. subdive state 1 is the same across all dive types). Even still, model complexity inevitably grows rapidly as hierarchical structures are stacked on top of each other. For example, by far the fastest model to train in the simulation study is the CarHMM-DFT, as it has no hierarchical component and is the simplest of the four candidate models. The CarHMM-DFT also has near-perfect accuracy when decoding the simulated subdive behavioral states. However, this model is not sufficient if ecologists wish to understand to joint relationship between dive type and intra-dive behaviour. 

%The simulation study also shows that the observed Fisher information serves as a suitable approximation for the standard errors of parameter estimates in most cases. One notable exception is the HHMM-DFT, which ignores auto-correlation and therefore underestimates the standard error of the parameters associated with acceleration. This can be dangerous in the field of ecology, as overconfident parameter estimates can result in overconfident conservation decisions. While the point estimates are generally good in this case study, this may not be the case in more general settings and for shorter time series in particular.

%We used the CarHHMM-DFT to model the behavior of a killer whale off the coast of British Columbia, Canada. The CarHHMM-DFT was able to simultaneously distinguish three distinct subdive behaviors and two dive types. The DFT component proved useful in determining the subdive behaviour of the whale, as the mean of the emission distribution of $Z^{*(2)}$ for each subdive state was separated by an order of magnitude. Finally, the estimated auto-correlation parameter for $\mathbf{Z}^{*(1)}$, $\phi^*$, was above 0.5 for every dimension and subdive type, providing evidence that the conditionally auto-regressive component of the CarHHMM-DFT resulted in a better fit to the data. The introduction of the parameter $\phi^*$ also allows $\mathbf{Z}^{*(1)}$ to be interpreted as a state-switching OU process (see appendix).

%Because traditional information criteria tend to overestimate the number of states in biological processes \citep{Pohle:2017}, the number of dive types and subdive behaviors was selected in an ad-hoc manner. There does appear to be some heterogeneity within dive types, and future work can be done to determine the optimal number of dive types and within-dive behaviors.

Despite these drawbacks, this work provides a flexible framework to model functional time-series data using HMM models. When model complexity is kept under control, the resulting hierarchical model is both highly flexible and easy to fit using maximum likelihood methods. We demonstrate the usefulness of such a structure using an ecological example to determine the energetic requirements of a killer whale. It is our hope that researchers adapt the methodology described here to solve problems and perform inference in other settings which involve temporally dependent state-switching processes.