% !TeX root = ../main.tex

\section{Introduction}

The field of animal movement is in the midst of a ``data renaissance" in which advancements in tagging technology have given rise to an explosion of data available for statistical modeling. In response, researchers have introduced a variety of new statistical techniques to infer animal behavior from movement data \cite{Hooten:2017}. 

These techniques usually fall into one of two different categories: continuous-time models and discrete-time models. Continuous-time models often model the dynamics of an animal as the solution to some stochastic differential equation and can incorporate observations taken at irregular time intervals. These methods are often computationally intractable and require approximate inference techniques such as Markov-chain Monte Carlo (MCMC) methods to perform inference. On the other hand, discrete-time techniques use summary statistics taken at discrete time steps to model the animal's behavior and require that observations be taken at regular time intervals. The primary advantage of discrete-time methods is that the associated likelihood is relatively easy to compute, facilitating fast maximum likelihood estimates for parameters. 

One of the most prevalent discrete-time techniques in the literature of late is the hidden Markov model (HMM), where observations depend upon the state of an associated unobserved behavioral process following Markovian dynamics \cite{Patterson:2017}. Importantly, under the traditional HMM model, subsequent observations are assumed to be independent from one another conditioned on the underlying behavioral process. However, this assumption is often violated in the real world processes. For example, the location of an animal at a given time is highly correlated with the location of that animal one second later. Several publications have dealt with this issue in the past, including the hidden movement Markov model (HMMM) by Whoriskey et al \cite{Whoriskey:2016}. 

\textit{The conditionally autoregressive hidden Markov model (CarHMM): Inferring Behavioral States from animal tracking data exhibiting conditional autocorrelation} by Ethan Lawler, Kim Whoriskey, William H. Aeberhard, Chris Fields, and Joanna Mills Flemming introduces a new framework to model autocorrelation in hidden Markov models. The paper also recommends several best practices for data exploration, model selection, and data preprocessing. This paper summarizes the findings of Lawler et al, elaborates on some of the findings and recommendations within, and draws a connection between the CarHMM and a continuous-time method presented by Michelot et al \cite{Michelot:2019}. The CarHMM is also adapted and applied to real-world dive data to analyze the intra-dive behavior of a killer whale off the coast of British Columbia, Canada.