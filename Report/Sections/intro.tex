Biologging technology now provides researchers with data collected almost continuously in time \citep{Hooten:2017}.
The collection and analysis of data from devices such as accelerometers has brought new insights to areas ranging from monitoring machine health \citep{Getman:2009} to understanding physical activity levels in children \citep{Morris:2007}. In particular, the study of animal movement has been transformed by tracking devices that collect biologging data \citep{Borger:2020}. Tags can record observations at rates of tens of hertz, resulting in time series containing millions of observations over the course of several hours. These data contain a wealth of knowledge about human and animal behaviour, but parsing this vast amount of information is a challenge for statisticians.

Biologging data are sometimes modelled as a set of curves that can be analyzed by methods of Functional Data Analysis (FDA, see, e.g., \citealt{Ramsay:2005}). In \cite{Morris:2007}, a child's activity level throughout the day is viewed as a curve where metabolic activity is a function of time. The dive profile of a marine mammal can also be viewed as a curve. \citet{Fu:2017} show that estimating the amplitude and phase variation of elephant seal (\textit{Mirounga leonina}) dive curves is useful to classify dive types, although they ignore sequential dependence between dives. Similarly, our case study views the dive profile of a killer whale (\textit{Orcinus orca}) as a sequence of curves where three orthogonal axes of acceleration are each functions of time.

FDA was originally developed for curves that can be considered independent replicates, but the structure of time-series data is often complicated by additional dependencies. In particular, biologging data often display significant temporal dependence both between curves (coarse-scale) and within curves (fine-scale) \citep{Barajas:2017,Adam:2019}, as is the case for our case study data. Namely, the dive profile (Figure \ref{fig:data}a) of a killer whale from the threatened Northern resident population off the coast of British Columbia, Canada, clearly shows temporal dependence \textit{between} dives, as dives of various time durations and depths appear to be clustered together in time. Previous studies have categorized these dives into different types where each dive type is associated with a different behaviour, including resting, travelling, and foraging \citep{Tennessen:2019b}.
Figure {\ref{fig:labeled_dives}} also shows clear temporal dependence \textit{within} dives, as the accelerometer data within each dive indicate bouts of short-term periodicity. The combination of apparent behavioural changes combined with periodic behaviour makes this accelerometer data difficult to model using simple stochastic processes. Another example of behaviourally-dependent periodicity is in the field of machine health, where \citet{Lucero:2019} attach accelerometers to railway axles to detect cracks. The raw acceleration data exhibit extreme periodicity (in the form of vibration) that depends upon the behaviour of the train as the load and speed of the train change.

FDA literature provides models that account for some between-curve dependence, but these methods are not adequate when this dependence is in time. For example, multilevel models incorporate dependence via random effects. In the aforementioned study of children's activity levels \citep{Morris:2007}, the authors use a multilevel approach since each child yields two daily activity curves; the model includes a random effect for ``child''. In a classic paper that introduced this multilevel structure,  \cite{Bromback:1998} used cubic splines in a mixed effects model to study menstrual cycles. More recent work on multilevel models includes that of \citet{Di:2009}, \citet{Chen:2012} and \citet{Crainiceanu:2009}. However, multilevel models are not appropriate in many biologging applications since they account for neither temporal dependence nor different curve types. 
In addition to multilevel models, FDA researchers have used functional time series to model dependence in sequences of curves. Functional time series extend the ideas of classic time series to model the evolution of one curve into the next \citep{Kokoszka:2018}. While functional time series methods can in theory be applied to biologging data such as ours, they do not account for sequences of curves that switch between a discrete number of types over time.
 
Traditional FDA techniques similarly fall short when modelling complicated within-curve acceleration data. In particular, within-curve structure is usually modelled by a generic smooth mean function and a covariance function. \citet{Yao:2005} use this framework to construct a method of sparse longitudinal data analysis using functional principal component analysis. Alternatively, within-curve structure is sometimes generated via random regression. For example, \citet{Rice:2001} use random regression to construct non-parametric mixed effects models for unequally sampled noisy curves. However, radically changing behaviour on short time-scales similar to our dive sequence is challenging to model with these classic FDA techniques.

In light of these challenges, we turn to vast field of methodology for the analysis of animal movement \citep{Hooten:2017}, where one of the most prevalent techniques in the literature of late is the hidden Markov model, or HMM \citep{Patterson:2017,McClintock:2020}. HMMs interpret animal movement data as arising from different types of unobserved behavioural states, allowing biologists to infer the underlying behaviour of the animal from movement observations. While ubiquitous in the ecology literature, HMMs have seen little use in non-parametric functional modelling, with a few notable exceptions. In particular, \citet{Langrock:2018} take a non-parametric approach to model the distribution of HMM observations in terms of B-splines. This is useful for both incorporating covariates and flexibly describing complex data, since these functions are non-parametric and take covariates as arguments. However, this approach does not account for certain types of time correlation within or between the time series themselves. We focus on the time series themselves as functions here and show how HHMs can provide a new tool to the FDA toolbox.

Under a traditional HMM, observations are independent given the underlying hidden state process. However, this assumption is often violated in real-world processes, especially when observations are taken at high frequencies. Several solutions have been proposed in the ecology literature, such as the hidden movement Markov model \citep{Whoriskey:2016} and the conditionally auto-regressive hidden Markov model, or CarHMM \citep{Lawler:2019}. The CarHMM in particular explicitly models auto-correlation into an HMM while adding minimal complexity to the base structure of an HMM. Animal movement modellers have also used hierarchical hidden Markov models (HHMMs) to model simultaneous behavioural processes that occur at different time scales \citep{Barajas:2017,Adam:2019}. We borrow from ideas from both the CarHMM and the HHMM here.

Although they are useful in many situations, traditional HMMs, CarHMMs, and HHMMs cannot easily capture complicated structures at short time scales. The HHMM of \citet{Adam:2019} fails to capture the fine-scale periodic swimming pattern of horn sharks (\textit{Heterodontus francisci}). \citet{Heerah:2017} have demonstrated the advantages of using Fourier analysis within an HMM to account for periodic behaviour in the context of describing daily behavioural cycles of marine mammals. In addition, Fourier analysis has previously been used on accelerometer data to explain animal behaviour \citep{Fehlmann:2017,Shorter:2017}. Thus, incorporating Fourier analysis within the structure of an HMM appears to be a promising approach to account for periodic structures in fine-scale accelerometer data.

In this paper, we combine several existing methods from the ecology literature in novel ways to account for complex temporally dependent functional data. The resulting suite of methods make up a ``tool box" that can be used to build arbitrarily complex hierarchical models and thus extend the methods available for the analysis of functional data. We first describe HMMs as well as to two variants (CarHMMs and HHMMs) and discuss how Fourier analysis can handle fine-scale dependence structures. We also show how these methods can be combined to analyze increasingly complex data. In section 3, we construct several candidate models and fit them to the data from Figure \ref{fig:data}. Section 4 details a simulation study based on these candidate models, and in section 5 we discuss our results.