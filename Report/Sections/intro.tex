% !TeX root = ../main.tex

Biologging technology now provides researchers with kinematic data collected almost continuously in time \citep{Hooten:2017}.
The collection and analysis of data from devices such as accelerometers have brought new insights to research tasks ranging from monitoring machine health \citep{Getman:2009} to understanding physical activity levels in children \citep{Morris:2007}. The study of animal movement in particular has been transformed by tracking devices that record kinematic information in a variety of environments \citep{Dot:2016b,Borger:2020}. Tags can record over 50 observations per second, resulting in time series containing millions of observations over the course of several hours. 
These data contain a wealth of information about human and animal behaviour, but modelling these large data sets poses a challenge for statisticians and biologists. One particular difficulty is that simultaneous coarse- and fine-scale processes are often reflected in high-frequency data, and each scale can exhibit a unique and complicated structure.

Biologging data are frequently modelled as a set of curves and analyzed using methods for functional data analysis, or FDA \nocite{Ramsay:2005}(e.g., Ramsay and Silverman, 2005). For example, \citet{Morris:2007} views a child's activity level as a set of daily curves where metabolic activity is a function of time. Similarly, \citet{Fu:2017} views the dive profile of a southern elephant seal (\textit{Mirounga leonina}) as a set of dive curves. Dive amplitude and phase variation are used for classification into ``dive types".

FDA was originally developed to process curves assumed to be independent replicates (i.e., there is no between-curve dependence). Within-curve, fine-scale structure is not usually incorporated in FDA models. However, sets of curves often exhibit complex sequential dependencies, both between and within curves. This is especially the case for biologging data \citep{Barajas:2017}.
On a coarse scale, the dive profiles of marine animals show a discrete number of distinct dive types, where the sequence of dive types exhibits state-switching behaviour \citep{Tennessen:2019b}. On a fine scale, these profiles can display bouts of short-term periodicity within each dive \citep{Adam:2019}. Fine-scale periodicity nested within a coarser state-switching process is also common in fields such as machine health \citep{Xin:2018,Lucero:2019} and speech recognition \citep{Juang:1991}. 

Some FDA models account for between-curve dependence that occurs when multiple curves arise from separate groups of individuals, but these models are inadequate when modelling certain kinds of time dependencies. For example, previous studies have used multi-level models with random effects to model variation between and within individuals in the daily activity levels of children \citep{Morris:2007} or in the menstrual cycles of adults \citep{Bromback:1998}. More recent work involving multi-level models includes that of \citet{Crainiceanu:2009}, \citet{Di:2009}, and \citet{Chen:2012}. 
However, the first two papers do not account for temporal between-curve dependence and the third article's model of temporal dependence assumes that curves evolve smoothly in time. This is not appropriate in many biologging applications, where behaviours often change suddenly between a discrete number of types.
In addition to multi-level models, FDA researchers have used functional time series to model dependence in a sequence of curves. Functional time series extends the ideas of classic time series to model the evolution of one curve into the next \citep{Kokoszka:2018}, but does not account for sequences of time series curves whose distributions are determined by a discrete number of well-defined hidden states.

Traditional FDA techniques similarly fall short when modelling complicated within-curve data. In particular, within-curve structure is usually modelled by a generic smooth mean function and a covariance function \citep{Yao:2005} or with random regression \citep{Rice:2001}. However, time series data exhibiting both sharp behavioural changes and periodic fine-scale structure are difficult to model with these classical FDA techniques.

Our goal is to identify and describe discrete behavioural states at multiple scales within functional data (e.g., coarse-scale curve types and fine-scale behavioural states) and to model the dependence structure between those states. To accomplish this, we turn to the field of animal movement modelling \citep{Hooten:2017}, where one of the most prevalent techniques of late is the hidden Markov model, or HMM \citep{Patterson:2017,McClintock:2020}. HMMs interpret animal movement data as arising from a Markov chain on a discrete number of behavioural states, allowing biologists to infer the underlying behaviour of an animal from sequential observations of its position. While ubiquitous in ecology literature, HMMs have seen little use in nonparametric functional modelling, with a few notable exceptions. In particular, \citet{Langrock:2018} takes a nonparametric approach to model the distributions of HMM observations with B-splines, while \citet{DeSouza:2014} and \citet{DeSouza:2017} use HMMs to model state-switching behaviour within functional data. However, none of these papers account for temporal correlation taking place on multiple scales. 

While useful, HMMs alone are also not sufficient to model high-frequency time series data for three primary reasons.
First, classical HMMs fail to model simultaneous behavioural processes that occur at different time scales (e.g., both between and within curves). To address this issue, statistical ecologists have employed hierarchical HMMs (HHMMs) \citep{Barajas:2017,Adam:2019}, which model both scales with conditionally dependent HMMs.
Second, HMMs assume that subsequent observations are independent given an underlying hidden state process, but this is often not the case when observations are taken at extremely high frequencies. Several solutions have been proposed in the ecology literature, including the hidden movement Markov model \citep{Whoriskey:2016} and the conditionally autoregressive HMM, or CarHMM \citep{Lawler:2019}. 
Third, traditional HMMs, CarHMMs, and HHMMs cannot easily capture complicated dependence structures on short time scales. For example, \citet{Adam:2019} fails to capture the fine-scale periodic swimming patterns of horn sharks (\textit{Heterodontus francisci}) using a traditional HHMM. \citet{Heerah:2017} successfully use Fourier analysis within an HMM to account for daily behavioural cycles in marine mammals. Fourier analysis has previously been used with accelerometer data to explain animal behaviour \citep{Fehlmann:2017,Shorter:2017}. Thus, incorporating Fourier analysis into the structure of an HMM appears to be a promising approach to account for fine-scale periodic structures in biologging data.

We combine these existing methods from statistical ecology literature in novel ways to classify and describe state-switching behaviours in functional data while accounting for complex temporal dependence. The resulting methods make up a tool box that can be used to build arbitrarily complex hierarchical models to explain multi-scale functional and time series data with intricate dependence structures.
We begin in Section 2 by describing HMMs as well as two variants, CarHMMs and HHMMs, and discuss how summary statistics over moving windows can handle fine-scale dependence structures. We also show how these methods can be combined to analyze increasingly complex data. In Section 3 we fit several candidate models to data from a killer whale (\textit{Orcinus orca}) from the threatened northern resident population off the coast of British Columbia, Canada. Section 4 details a simulation study based on these candidate models, and in Section 5, we discuss our results.