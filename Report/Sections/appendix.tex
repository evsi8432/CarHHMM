% !TeX root = ../main.tex

\addcontentsline{toc}{section}{Appendices}
\renewcommand*{\thesubsection}{\Alph{subsection}}

\citet{Michelot:2019} model the movement of an animal as the solution to the following stochastic differential equation, which they refer to as a state-switching Ornstein-Uhlenbeck process:
%
\begin{equation}
    \label{eqn:SDE}
    dY_t = \beta^{(X_t)}(\gamma^{(X_t)} - Y_t)dt + \omega^{(X_t)} dW_t,
\end{equation}
%
where $X_t$ is some stochastic process which defines the hidden behaviour of the animal at time $t$, $\beta^{(X_t)}$ relates to rate at which the process returns to its mean value, $\gamma^{(X_t)}$ is the long-term mean value of the process, $\omega^{(X_t)}$ is related to short-term variance, and $W$ is a Wiener process. Unlike HMMs, $t \in \mathbb{R}$ indexes a continuous time process and is therefore not necessarily an integer. If the behavioral state $X_t$ is known and does not change between observations $\Big($i.e $X_s = X_t$ for all $s \in [t,t+\Delta t)\Big)$, the solution to Equation (\ref{eqn:SDE}) is known to be the following \citep{Michelot:2019}:
%
\begin{align}
    Y_{t+\Delta t} | X_{t} \sim \mathcal{N}\left((1-e^{-\beta^{(X_t)}\Delta t})\gamma^{(X_t)} + e^{-\beta^{(X_t)}\Delta t} Y_t,\quad \frac{\omega^{(X_t)^2}}{2\beta^{(X_t)}} (1-e^{-2\beta^{(X_t)}\Delta t})\right)
    \label{eqn:OU_sol}
\end{align}
%
where $\Delta t$ is the time difference between any two observations $Y_t$ and $Y_{t+\Delta t}$. In fact, under certain conditions, the output of a state-switching OU process is identical to that of a CarHMM, as described by Theorem 1.

\begin{theorem}{Theorem 1.}{}%
If the following conditions are met:
\begin{enumerate}
    \item The hidden behavioural process $X$ from Equation (\ref{eqn:SDE}) follows a Markov chain with $N$ possible states and transitions occur at equi-spaced time stamps $\left(\Delta t, \ldots, (T-1)\Delta t\right)$,
    
    \item Observations of the SDE from Equation 2 are taken at times $\left(0, \Delta t, \ldots, (T-1)\Delta t\right)$,
\end{enumerate}
then the observations $Y$ are equivalent to the output of a conditionally auto-regressive hidden Markov model with normal emission distributions and parameters $\Theta = (\theta^{(1)}, \ldots, \theta^{(N)}); \enspace \theta^{(i)} = \{\mu^{(i)},\sigma^{(i)},\phi^{(i)}\}$, where:

\begin{align}
\mu^{(i)} = \gamma^{(i)}, \qquad \sigma^{(i)} = \sqrt{\frac{\omega^{(i)^2}}{2\beta^{(i)}} (1-e^{-2\beta^{(i)}\Delta t})}, \qquad \phi^{(i)} = e^{-\beta^{(i)}\Delta t} \label{eqn:CarHMM_to_OU}
\end{align}

\end{theorem}

\begin{proof}{Proof of Theorem 1}{}
Combining Equation (\ref{eqn:OU_sol}) with Equation (\ref{eqn:CarHMM_to_OU}) gives:
%
\begin{align*}
\left(Y_{s \Delta t} | X_{(s-1)\Delta t} = i \right) \sim \mathcal{N}\left((1-\phi^{(i)}) \mu^{(i)} + \phi^{(i)} Y_{(s-1) \Delta t}, \enspace \left(\sigma^{(i)}\right)^2 \right),\\
s = 1, \ldots, T-1
\end{align*}
%
If $X_t$ follows a Markov chain with transitions at the observation times, then the behavioural state $X_t$ does not change between observations and we can re-index $Y_{(s-1) \Delta t} = Y'_s$ and $X_{(s-2)\Delta t} = X'_s$ for $s = 2,\ldots T$, yielding the desired result:

\begin{align*}
\left(Y'_s| X'_s = i \right) \sim \mathcal{N}\left((1-\phi^{(i)}) \mu^{(i)} + \phi^{(i)} Y'_{s-1}, \enspace \left(\sigma^{(i)}\right)^2 \right)\\
s = 2, \ldots, T
\end{align*}
%
$X'$ follows a (possibly unobserved) Markov chain, and the distribution of $(Y'_s|X'_s,Y'_{s-1})$ is consistent with that of a CarHMM with Normal emission distributions.
\end{proof}

%%%%%%%%%%%%%%%%%%%%%%%%%%%%%%%%%%%%%%%%%%%%%%%%%%%%%%%%%%%%%%%%%%%%%%%%%%%%%%%%%%%%%%%%%%%%%%%%%%%%%%%%%%%%%%%%%%%%%%%%%%%%%%%%%%%%%%%%%%%%%%%%%%%%%%%%%%

\begin{proof}{Detailed description of simulated data}{}

Assume that $X$, $Y$, and $X^*$ are known and simulated using the procedure detailed in Section \ref{sec:sim_study}. Let $DFT\{Y^*_{t,t^*}\}(k)$ be denoted as $\hat{Y}^{*(k)}_{t,t^*}$ and let $DFT\{Y^*_{t,t^*}\}$ be denoted as $\hat{Y}^*_{t,t^*}$ for shorthand. 

Then, $\hat{Y}^*_{t}$ is simulated using the following procedure for every dive. First, simulate the average acceleration of the first two-second window of the dive:
$$
	(\hat{Y}^{*(0)}_{t,1}|X^*_{t,1} = i^*) \sim \mathcal{N} \left(0, \left(\sigma^{*(i^*)}\right)^2 \right)
$$
Remember that the average acceleration within a window is equal to the real part of the $0^{th}$ Fourier mode, so $\hat{Y}^{*(0)}_{t,1}$ is the average acceleration within the first window. Next, simulate the average acceleration for each subsequent two-second window:
\begin{equation}
	\Big(\hat{Y}^{*(0)}_{t,t^*}|X^*_{t,t^*} = i^*,\hat{Y}^{*(0)}_{t,t^*-100}\Big) \sim \mathcal{N} \left(\phi^{*(i^*)} \hat{Y}^{*(0)}_{t,t^*-100}, \left(\sigma^{*(i^*)}\right)^2 \right), \quad t^* = 101,201,\ldots
	\label{eqn:yhat_0}
\end{equation}
Note that $t^*$ has increments of 100 because $h=100$ when using two-second window sizes. Finally, simulate the other Fourier modes within each window:
$$
	\hat{Y}^{*(k)}_{t,t^*} = a_{t,t^*}^{(k)} i\sqrt{b^{(k)}_{t,t^*}}, \quad k = 1,\ldots,49;
$$
where $a$ and $b$ are random variables with the following distributions:
\begin{align*}
    a_{t,t^*}^{(k)} &\sim  \left\{\begin{array}{lr}
	-1 & \quad w.p. \enspace 1/2 \\
	1  & \quad w.p. \enspace 1/2
	\end{array}\right. \\
	%
	(b^{(k)}_{t,t^*}|X^*_{t,t^*}  = 1) &\sim {\rm{Gamma}}(5/k^2, 1) \\
	%
	(b^{(k)}_{t,t^*}|X^*_{t,t^*} = 2) &\sim \left\{\begin{array}{lr}
	{\rm{Gamma}}(5/k^2, 1), \quad & k \notin \{1,2\} \\
	{\rm{Gamma}}(250,1), & k = 1 \\
	{\rm{Gamma}}(50,1), & k = 2.
	\end{array}\right. 
\end{align*}
Note that $b_{t,t^*}$ decays like $1/k^2$ to ``smooth out" the raw acceleration. For Fourier modes 50 through 99, we use the following procedure to ensure that the raw acceleration is real-valued:
$$
    \hat{Y}^{*(50)}_{t,t^*} = 0; \qquad
	\hat{Y}^{*(k)}_{t,t^*} = -\hat{Y}^{*(100-k)}_{t,t^*} \enspace \text{for} \enspace k = 51,\ldots,99.
$$
%
$\Big(Y^*_{t,t^*},\ldots,Y^*_{t,t^*+99}\Big)$ is finally set using the inverse discrete Fourier transform of $\hat{Y}^*_{t,t^*}$:
%
$$Y^*_{t,t^*+n} = IDFT\left\{\hat{Y}^*_{t,t^*}\right\}(n); \quad n = 0,\ldots,99; \quad t^* = 1,101,\ldots$$
%
%$\hat{Y}^*_{t,t^*}$ is anti-symmetric about $\hat{Y}^{*(50)}_{t,t^*}$ so that its inverse Fourier transform is real-valued. $\hat{Y}^{*(k)}_{t,t^*}$ also decays like $1/k$ so that $Y^*_{t,t^*}$ remains continuous within a two-second window. $Y^*_{t,t^*}$ is not continuous \textit{between} windows, but the jump discontinuities are not very severe since $\hat{Y}^{*(0)}_{t,t^*}$ and $\hat{Y}^{*(0)}_{t,t^*+1}$ are highly correlated. See Figure \ref{fig:sim_data} for details. 

We now prove that this construction of $Y^*$ results in the correct distributions listed in Section \ref{sec:sim_study}. 

It is straightforward to calculate both $\Zone$ and $\Ztwo$ from $Y^*$. We pick $\tilde{\omega} = 10$ periods per window, or 5 Hz. To find the distribution of $\Zone_{t,t^*}$, use Equation (\ref{eqn:yhat_0}) together with the substitution $\Zone_{t,t^*} = \text{Re}\left(\hat{Y}^{(0)}_{t,t^*}\right) = \hat{Y}^{(0)}_{t,t^*}$:
%
$$\left(\Zone_{t,t^*} | X^*_{t,t^*} = i^*, \Zone_{t,t^*-100} \right) \sim \mathcal{N} \left(\phi^{*(i^*)} \Zone_{t,t^*-100}, \left(\sigma^{*(i^*)}\right)^2 \right).$$
%
This distributions is consistent with a CarHMM with Normal emission distributions and $\mu_A^{*(i^*)} = 0$. The other parameters $\sigma_A^{*(i^*)}$ and $\phi_A^{*(i^*)}$ are free, and we select $\sigma_A^{*(1)} = 0.05s$, $\phi_A^{*(1)} = 0.99$, $\sigma_A^{*(2)} = 0.1s$, and $\phi_A^{*(2)} = 0.95$.

The wiggliness $\Ztwo_{t,t^*}$ is the sum of gamma-distributed random variables with identical scale parameters, so the distribution of $\Ztwo_{t,t^*}$ is also gamma:
%
$$\Ztwo_{t,t^*} = \sum_{k=1}^{10} b^{(k)}_{t,t^*}$$
%
$$\left(\Ztwo_{t,t^*}|X^*_{t,t^*} = 1\right) \sim {\rm{Gamma}}\left(\alpha = \sum_{k=1}^{10} 5/k^2 = 10.10, \beta = 1.00 \right)$$
%
$$\left(\Ztwo_{t,t^*}|X^*_{t,t^*} = 2\right) \sim {\rm{Gamma}}\left(\alpha = 300 + \sum_{k=3}^{10} 5/k^2 = 305.94 , \beta = 1.00 \right).$$
%
It follows that $\mu_W^{*(1)} = 10.10$, $\sigma_W^{*(1)} = 3.18$, $\mu_W^{*(2)} = 305.94$, and $\sigma_W^{*(2)} = 17.49$.
\end{proof}

\begin{proof}{Likelihood of Simulation study model}{}

The overall likelihood of the CarHHMM-DFT model is as follows:
%
$$\calL_{\text{CarHHMM-DFT}}(y,z^*;\Theta,\Theta^*,\Gamma,\Gamma^*) = \delta P(y_1,z_1^*;\Theta,\Theta^*,\Gamma^*) \prod_{t=2}^T \Gamma P(y_t,z_t^*;\Theta,\Theta^*,\Gamma^*) \mathbf{1}_N.$$
%
In particular,
%
\begin{align*}
P(y_t,z_t^*;\Theta,\Theta^*,\Gamma^*)  = \text{diag}\Big[&f^{(1)}(y_t;\Theta^{(i)})\calL_{\text{CarHMM-DFT}}\left(z_t^*;\Theta^{*(1)},\Gamma^{*(1)}\right), \ldots , \\
&f^{(N)}(y_t;\Theta^{(i)})\calL_{\text{CarHMM-DFT}}\left(z_t^*;\Theta^{*(N)},\Gamma^{*(N)}\right) \Big],
\end{align*}
%
where $f^{(i)}(y_t;\Theta^{(i)})$ is the emission distribution of the dive duration $Y_t$ conditioned on the fact that $X_t = i$.

The likelihood $\calL_{\text{CarHMM-DFT}}$ corresponds to the fine-scale chain and has the following expression:
%
$$\calL_{\text{CarHMM-DFT}}\left(z_t^*;\Theta^{*(i)},\Gamma^{*(i)}\right) = \delta^{*(i)} \prod_{t=2}^T \Gamma^{*(i)} P(z^*_{t,t^*}|z^*_{t,t^*-100};\Theta^{*(i^*)}) \mathbf{1}_N,$$
%
where $P(z^*_{t,t^*}|z^*_{t,t^*-100};\Theta^{*(i^*)})$ is an $N^* \times N^*$ diagonal matrix with $(i^*,i^*)^{th}$ entry equal to $f^{*(\cdot,i^*)}(z^*_{t,t^*}|z^*_{t,t^*-100}; \theta^{*(\cdot,i^*)})$.
%
Note that $f^{*(\cdot,i^*)}(z^*_{t,t^*}|z^*_{t,t^*-100}; \theta^{*(\cdot,i^*)})$ is the emission distribution of $\Z_{t,t^*}$ conditioned on $X^*_{t,t^*} = i^*$ and $\Z_{t,t^*-100} = z^*_{t,t^*-100}$.

\end{proof}

%%%%%%%%%%%%%%%%%%%%%%%%%%%%%%%%%%%%%%%%%%%%%%%%%%

\iffalse


\subsection{Dimension Reduction on $\hat{Y_t^*}$}

First, we simply use down-sampling and only record every $w^{th}$ time step of $\hat{Y}_t^*$. This both reduces the space of $\hat{Y}_t^*$ to $\mathbb{C}^{\lfloor t^*_t / w \rfloor \times w}$ and ensures that none of the sliding windows overlap when taking the STFT. This is important because HMMs assume temporal independence between observations. Next, the dimension of $\hat{Y}_t^*$ can be cut in half by recognizing that the $Y_t^*$ is real-valued, and therefore $\hat{Y}_{t,k}^*$ is equal to the complex-congugate of $\hat{Y}_{t,w-k-1}^*$. Finally, by Parseval's Thereom we have that:

$$\sum_{n = 0}^{w-1} |Y^*_{t+n}|^2 = \sum_{n = 0}^{w-1} |\hat{Y}^*_{t,n}|^2$$

\subsection{Equivalency of CarHMM and one-dimensional state-switching Ornstein-Uhlenbeck process}

If it is the case that (1) the underlying behavioral state of the continuous-time model must follow a Markov chain rather than a Markov process, and (2) the emission distributions of the CarHMM are gaussian, then the CarHMM and the state-switching continuous model are equivalent. This allows the theoretically grounded continuous-time state-switching model to be used in the computational convieneint HMM (and therefore HHMM) framework. In addition, it gives new intpretation to the learned parameters of the CarHMM in the context of an Ornstein-Uhlenbeck process.

A one-dimensional state-switching Ornstein-Uhlenbeck process $y^*$ is the solution to the following stochastic differential equation:
%
$$dy^*_t = \beta_{x^*_t}(\gamma_{x^*_t} - y^*_t)dt + \omega_{x^*_t} dW_t$$
%
where $x^*_t$ is the fine-scale behavior of the animal at time $t$, $\beta_{x^*_t}$ relates to rate at which the process returns to its mean value, $\gamma_{x^*_t}$ is the long-term mean value of the process, $\omega_{x^*_t}$ is related to short-term variance, and $W$ is a Wiener process. As before, $x^*_t$ is described by an unobserved Markov process. The solution to this equation is known to be the following \cite{Michelot:2019}:
\begin{align*}
y^*_{t+\delta} \sim \mathcal{N}\left((1-e^{-\beta_{x^*_t}\delta})\gamma_{x^*_t} + e^{-\beta_{x^*_t}\delta} y^*_t,\quad \frac{\omega_{x^*_t}^2}{2\beta_{x^*_t}} (1-e^{-2\beta_{x^*_t}\delta})\right)
\end{align*}
Now, suppose that $\delta$ is constant for all observations, as is the case for hidden Markov models. In addition, introduce the following transformations:
\begin{align*}
\mu_{x^*_t} = \gamma_{x^*_t}, \qquad \phi_{x^*_t} = e^{-\beta_{x^*_t}\delta}, \qquad \sigma^2_{x^*_t} = \frac{\omega_{x^*_t}^2}{2\beta_{x^*_t}} (1-e^{-2\beta_{x^*_t}\delta})
\end{align*}
Then, we have the following:
\begin{align*}
y^*_{t+\delta} \sim \mathcal{N}\left((1-\phi_{x^*_t})\mu_{x^*_t} + \phi_{x^*_t} y^*_t,\quad \sigma_{x^*_t}^2 \right)
\end{align*}
%
If $\delta$ is fixed and $x^*_t$ is adjusted to follow a Markov chain rather than a Markov process, then this model is equivalent to the CarHMM with normal emission probabilities. Note that all of the parameter transformations above are one-to-one, so it is easy to go from the CarHMM to the continuous model and back again. This allows for the principled construction of the continuous-time model to be combined with the computational convenience of the CarHMM.

\fi
