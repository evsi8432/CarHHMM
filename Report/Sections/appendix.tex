% !TeX root = ../main.tex

\subsection{Detailed description of data simulation}

It is straightforward to simulate realizations of the coarse-scale HMM ($X$ and $Y$) given the parameters $\Gamma$ and $\theta$. It is also straightforward to simulate $\tilde X^*_t$ given the length of each fine-scale Markov chain ($\tilde T^*_t = \lfloor Y_t/2 \rfloor$), the dive types $X_t$, dive durations $Y_t$, and fine-scale probability transition matrices $\Gamma^{*(1)}$ and $\Gamma^{*(2)}$. 
We assume that $X^*_{t,t^*}$ is equal to $\tilde X^*_{t,\tilde t^*}$ within the window corresponding to time-stamp $t^*$ (i.e., $X^*_{t,t^*} = \tilde X_{t,\lceil t^*/100 \rceil}$). It is much more difficult to simulate the raw acceleration data $Y^*_t$, and we focus on this task for the remainder of this section. 

Let $DFT\{Y^*_{t,100(\tilde t^*-1) + 1},\ldots,Y^*_{t,100t^*}\}(k)$ be denoted as $\hat{Y}^{*(k)}_{t,\tilde t^*}$ and $DFT\{Y^*_{t,100(\tilde t^*-1) + 1},\ldots,Y^*_{t,100t^*}\}$ be denoted as $\hat{Y}^{*}_{t,\tilde t^*}$, where $\{Y^*_{t,100(\tilde t^*-1) + 1},\ldots,Y^*_{t,100t^*}\}$ represents window $\tilde t^*$. 

We simulate $Y_t^*$ in three steps: 1) simulate the average acceleration within each window $\left(\hat{Y}^{*(0)}_{t,\tilde t^*}\right)$, 2) simulate the rest of the Fourier modes within that window $\left(\hat{Y}^{*(k)}_{t,\tilde t^*}, \enspace k = 1,\ldots,99\right)$, and 3) take the inverse Fourier transform of each set of Fourier coefficients $IDFT\left
\{\hat{Y}^{*}_{t,\tilde t^*}\right\}$. We go over each step in detail below:

\begin{enumerate}
\item First simulate the average acceleration within each window $\left(\hat{Y}^{*(0)}_{t,\tilde t^*}\right)$:
\begin{enumerate}
    \item Simulate $\hat{Y}^{*(0)}_{t,1}$:
    $$
    	(\hat{Y}^{*(0)}_{t,1}|\tilde X^*_{t,1} = i^*) \sim \mathcal{N} \left(0, \left(100\sigma_A^{*(\cdot,i^*)}\right)^2 \right), \quad i^* \in \{1,2\}
    $$
    where $\sigma_A^{*(\cdot,1)} = 0.05s$ and $\sigma_A^{*(\cdot,2)} = 0.1s$. 
    %
    \item Simulate $\hat{Y}^{*(0)}_{t,2}, \enspace \ldots \enspace \hat{Y}^{*(0)}_{t,\tilde{T}_t^*} $ using Equation (\ref{eqn:yhat_0}) below:
    \begin{align}	
        \Big(\hat{Y}^{*(0)}_{t,\tilde t^*}|\tilde X^*_{t, \tilde t^*} = i^*,\hat{Y}^{*(0)}_{t,\tilde t^*-1}\Big) &\sim \mathcal{N} \left(\phi_A^{*(\cdot,i^*)} \hat{Y}^{*(0)}_{t,\tilde t^*-1}, \left(100\sigma_A^{*(\cdot,i^*)}\right)^2 \right), \nonumber \\
        %
        &t^* = 1,2,\ldots,\tilde{T}_t^*
    	\label{eqn:yhat_0}
    \end{align}
    where $\phi_A^{*(\cdot,1)} = 0.99$ and $\phi_A^{*(\cdot,2)} = 0.95$. 
\end{enumerate}
%
This process simulates the average acceleration within each two-second window $\tilde t^* = 1, \ldots, \tilde T^*_t$ (see Equation (\ref{eqn:avg_val}) below). 
\item Next simulate the rest of the Fourier modes within each window ($\hat{Y}^{*(k)}_{t,\tilde t^*}, \enspace k = 1,\ldots,99$):
%
\begin{enumerate}
    \item Simulate Fourier coefficients 1 through 49 within each window:
    \begin{equation}
        \hat{Y}^{*(k)}_{t,\tilde t^*} = a_{t,\tilde t^*}^{(k)} i\sqrt{b^{(k)}_{t,\tilde t^*}}, \quad k = 1,\ldots,49
        \label{eqn:abYhat}
    \end{equation}
    where $a^{(k)}_{t,\tilde t^*}$ is equal to either 1 or -1 with probability 1/2 each, and $b^{(k)}_{t,\tilde t^*}$ has the following distribution:
    \begin{align}
    \begin{split}
    	(b^{(k)}_{t,\tilde t^*}|\tilde X^*_{t,\tilde t^*} = 1) &\sim {\rm{Gamma}}(5/k^2, 1) \\
    	%
    	(b^{(k)}_{t,\tilde t^*}|\tilde X^*_{t,\tilde t^*} = 2) &\sim \left\{\begin{array}{lr}
    	{\rm{Gamma}}(250,1), & k = 1 \\
    	{\rm{Gamma}}(50,1), & k = 2 \\
    	{\rm{Gamma}}(5/k^2, 1), \quad & k \notin \{1,2\}.
    	\end{array}\right. 
    \end{split}
    \label{eqn:bdist}
    \end{align}
    The first argument of ${\rm{Gamma}}\left(\cdot,\cdot\right)$ is the scale parameter and the second is the shape parameter. The expected value $\mathbb{E}\left(|b^{(k)}_{t,\tilde t^*}|\right)$ decays like $1/k^2$ to ``smooth out" the raw acceleration data.
    %
    \item Set Fourier coefficients 50 through 99 to ensure that the inverse DFT is real-valued:
    $$
    \hat{Y}^{*(50)}_{t,\tilde t^*} = 0, \qquad
	\hat{Y}^{*(k)}_{t,\tilde t^*} = -\hat{Y}^{*(100-k)}_{t,\tilde t^*} \enspace \text{for} \enspace k = 51,\ldots,99.
    $$
\end{enumerate}
This completes the process of simulating $\hat{Y}^{*(k)}_{t,\tilde t^*}$ for all windows $\tilde t^* = 1,\ldots,\tilde T_t^*$ and Fourier coefficients $k = 0,\ldots,99$. 
\item Finally, set each window of raw acceleration $\left\{Y^*_{t,100(\tilde t^* - 1) + 1},\ldots,Y^*_{t,100\tilde t^*}\right\}$ using the inverse discrete Fourier transform of $\hat{Y}^*_{t,\tilde t^*}$:
%
$$Y^*_{t,100(\tilde t^* - 1)+ n+1} = IDFT\left\{\hat{Y}^*_{t,\tilde t^*}\right\}(n), \quad n = 0,\ldots,99, \quad \tilde t^* = 1,2,\ldots,\tilde T_t^*.$$
%
This completes the construction of $Y^*_t$.
\end{enumerate}

We now show that this construction results in the distributions defined in Section \ref{subsec:data_simulation}. First, note that $\hat{Y}^{(0)}_{t,\tilde t^*}$ is just the sum of the raw acceleration values within the corresponding window:
%
\begin{align}
    \hat{Y}^{(0)}_{t,\tilde t^*} &= DFT\{Y^*_{t,100(\tilde t^*-1) + 1},\ldots,Y^*_{t,100\tilde t^*}\}(0) \nonumber \\
    &= \sum_{n=0}^{99} Y^*_{t,100(\tilde t^*-1) + 1 + n} \exp\left(-\frac{i2\pi}{99}*0*n\right) \nonumber \\
    &= \sum_{n=1}^{100} Y^*_{t,100(\tilde t^*-1) + n} \nonumber \\
    &= 100\Zone_{t,\tilde t^*}.
    \label{eqn:avg_val}
\end{align}
%
Combining Equation (\ref{eqn:avg_val}) above with Equation (\ref{eqn:yhat_0}) gives the distribution of $\Zone_{t,t^*}$: 
%
$$\left(100\Zone_{t,\tilde t^*} | \tilde X^*_{t,\tilde t^*} = i^*, \Zone_{t,\tilde t^*-1} \right) \sim \mathcal{N} \left(100\phi_A^{*(\cdot,i^*)} \Zone_{t,\tilde t^*-1}, \left(100\sigma_A^{*(\cdot,i^*)}\right)^2 \right)$$
%
$$\left(\Zone_{t,t^*} | \tilde X^*_{t,t^*} = i^*, \Zone_{t,\tilde t^*-1} \right) \sim \mathcal{N} \left(\phi_A^{*(\cdot,i^*)} \Zone_{t,\tilde t^*-1}, \left(\sigma_A^{*(\cdot,i^*)}\right)^2 \right).$$
%
This distribution is consistent with a CarHMM with Normal emission distributions and $\mu_A^{*(\cdot,1)} = \mu_A^{*(\cdot,2)} = \mu_A^{*(\cdot,i^*)} = 0$. We also have that $\sigma_A^{*(\cdot,1)} = 0.05s$, $\phi_A^{*(\cdot,1)} = 0.99$, $\sigma_A^{*(\cdot,2)} = 0.1s$, and $\phi_A^{*(\cdot,2)} = 0.95$.

To calculate the wiggliness $\Ztwo_{t,\tilde t^*}$, we pick $\tilde{\omega} = 10$ periods per window (or 5 Hz). Then, from Equation (\ref{eqn:abYhat}):
%
\begin{equation}
    \Ztwo_{t,\tilde t^*} = \sum_{k=1}^{10} \big|\big|\hat{Y}^{(k)}_{t,\tilde t^*}\big|\big|^2 = \sum_{k=1}^{10} \big|\big|a^{(k)}_{t,\tilde t^*}i\sqrt{b^{(k)}_{t,\tilde t^*}}\big|\big|^2 = \sum_{k=1}^{10} b^{(k)}_{t,\tilde t^*}
    \label{eqn:Wb}
\end{equation}
%
since $a^{(k)}_{t,t^*} \in \{-1,1\}$ and $b^{(k)}_{t,\tilde t^*} \geq 0$. Equation (\ref{eqn:Wb}) shows that $\Ztwo_{t,\tilde t^*}$ is the sum of Gamma-distributed random variables with identical scale parameters, so $\Ztwo_{t,\tilde t^*}$ is also Gamma-distributed. Combining Equation (\ref{eqn:Wb}) above with Equation (\ref{eqn:bdist}) gives
%
$$\left(\Ztwo_{t,\tilde t^*}|\tilde X^*_{t,\tilde t^*} = 1\right) \sim {\rm{Gamma}}\left(\sum_{k=1}^{10} 5/k^2 = 10.10, 1.00 \right) \text { and }$$
%
$$\left(\Ztwo_{t,t^*}|\tilde X^*_{t,\tilde t^*} = 2\right) \sim {\rm{Gamma}}\left(300 + \sum_{k=3}^{10} 5/k^2 = 305.94 , 1.00 \right),$$
%
where the first argument of ${\rm{Gamma}}\left(\cdot,\cdot\right)$ is the scale parameter and the second is the shape parameter. Simple calculation of the mean and variance of a Gamma distribution shows that $\mu_W^{*(\cdot,1)} = 10.10$, $\sigma_W^{*(\cdot,1)} = 3.18$, $\mu_W^{*(\cdot,2)} = 305.94$, and $\sigma_W^{*(\cdot,2)} = 17.49$.

%%%%%%%%%%%%%%%%%%%%%%%%%%%%%%%%%%%%%%%%%%%%%%%%%%%%%%%%%%%%%%%%%%%%%%%%%%%%%%%%%%%%%%%%%%%%%%%%%%%%%%%%%%%%%%%%%%%%%%%%%%%%%%%%%%%%%%%%

\subsection{Likelihood of simulation study model}

Define $\z = \{\z_1,\ldots,\z_{T}\}$. The overall likelihood of the CarHHMM-DFT model is as follows:
%
$$\calL_{\text{CarHHMM-DFT}}(\theta,\theta^*,\Gamma,\Gamma^*;y,\z) = \delta P(y_1,\z_1;\theta,\theta^*,\Gamma^*) \prod_{t=2}^T \Gamma P(y_t,\z_t;\theta,\theta^*,\Gamma^*) \mathbf{1}_N.$$
%
In particular,
%
\begin{align*}
P(y_t,\z_t;\theta,\theta^*,\Gamma^*)  = \text{diag}\Big[&f^{(1)}(y_t;\theta^{(1)})\calL_{\text{CarHMM-DFT}}\left(\theta^*,\Gamma^{*(1)};\z_t\right), \ldots , \\
&f^{(N)}(y_t;\theta^{(N)})\calL_{\text{CarHMM-DFT}}\left(\theta^*,\Gamma^{*(N)};\z_t\right) \Big],
\end{align*}
%
where $f^{(i)}(y_t;\theta^{(i)})$ is the emission distribution of dive duration given that $X_t = i$. The likelihood $\calL_{\text{CarHMM-DFT}}$ corresponds to the fine-scale model and is equal to the following:
%
$$\calL_{\text{CarHMM-DFT}}\left(\theta^{*},\Gamma^{*(i)};\z_t\right) = \delta^{*(i)} \prod_{\tilde t^* = 1}^{\tilde T_t^*} \Gamma^{*(i)} P(\z_{t,\tilde t^*}|\z_{t,\tilde t^*-1};\theta^*) \mathbf{1}_N,$$
%
where $P(\z_{t,\tilde t^*}|\z_{t,\tilde t^*-1};\theta^*)$ is an $N^* \times N^*$ diagonal matrix with $(i^*,i^*)^{th}$ entry equal to $f^{*(\cdot,i^*)}(\z_{t,\tilde t^*}|\z_{t,\tilde t^*-1}; \theta^{*(\cdot,i^*)})$.
%
Recall that $f^{*(\cdot,i^*)}(\cdot|\z_{t,\tilde t^*-1}; \theta^{*(\cdot,i^*)})$ is the probability density function of $\Z_{t,\tilde t^*}$ conditioned on $\tilde X^*_{t,\tilde t^*} = i^*$ and $\Z_{t,\tilde t^*-1} = \z_{t,\tilde t^*-1}$.

%%%%%%%%%%%%%%%%%%%%%%%%%%%%%%%%%%%%%%%%%%%%%%%%%%%%%%%%%%%%%%%%%%%%%%%%%%%%%%%%%%%%%%%%%%%%%%%%%%%%%%%%%%%%%%%%%%%%%%%%%%%%%%%%%%%%%%%%%%%%%%%%%%%%%%%%%%

\iffalse

\subsection{Equivalence of state-switching OU process and CarHMM}

\citet{Michelot:2019} model the movement of an animal as the solution to the following stochastic differential equation:
%
\begin{equation}
    \label{eqn:SDE}
    dY_t = \beta^{(X_t)}(\gamma^{(X_t)} - Y_t)dt + \omega^{(X_t)} dW_t,
\end{equation}
%
where $X_t$ is some stochastic process which defines the hidden behaviour of the animal at time $t$, $\beta^{(X_t)}$ relates to rate at which the process returns to its mean value, $\gamma^{(X_t)}$ is the long-term mean value of the process in state $X_t$, $\omega^{(X_t)}$ is related to short-term variance, and $W$ is a Wiener process. They refer to this as a state-switching Ornstein-Uhlenbeck process. Unlike an HMM, $t \in \mathbb{R}$ indexes a continuous time process and is therefore not necessarily an integer in this case. If the behavioural state $X_t$ is known and does not change between observations $\Big(X_s = X_t$ for all $s \in [t,t+\Delta t)\Big)$, the solution to Equation (\ref{eqn:SDE}) is known to be the following \citep{Michelot:2019}:
%
\begin{align}
    Y_{t+\Delta t} | X_{t} \sim \mathcal{N}\left((1-e^{-\beta^{(X_t)}\Delta t})\gamma^{(X_t)} + e^{-\beta^{(X_t)}\Delta t} Y_t,\quad \frac{\omega^{(X_t)^2}}{2\beta^{(X_t)}} (1-e^{-2\beta^{(X_t)}\Delta t})\right),
    \label{eqn:OU_sol}
\end{align}
%
where $\Delta t$ is the time difference between any two observations $Y_t$ and $Y_{t+\Delta t}$. In fact, under certain conditions, the output of a state-switching OU process is identical to that of a CarHMM, as described by Theorem 1 below.

\begin{theorem}{Theorem 1.}{}%
If the following conditions are met:
\begin{enumerate}
    \item The hidden behavioural process $X$ from Equation (\ref{eqn:SDE}) follows a Markov chain with $N$ possible states and transitions occur at equi-spaced time stamps $\left(\Delta t, \ldots, (T-1)\Delta t\right)$,
    
    \item Observations of the SDE from Equation (\ref{eqn:OU_sol}) are taken at times $\left(0, \Delta t, \ldots, (T-1)\Delta t\right)$,
\end{enumerate}
then the observations $Y$ are equivalent to the output of a conditionally auto-regressive hidden Markov model with normal emission distributions and parameters $\theta = \{\theta^{(1)}, \ldots, \theta^{(N)}\}; \enspace \theta^{(i)} = \{\mu^{(i)},\sigma^{(i)},\phi^{(i)}\}$, where:

\begin{align}
\mu^{(i)} = \gamma^{(i)}, \qquad \sigma^{(i)} = \sqrt{\frac{\omega^{(i)^2}}{2\beta^{(i)}} (1-e^{-2\beta^{(i)}\Delta t})}, \qquad \phi^{(i)} = e^{-\beta^{(i)}\Delta t} \label{eqn:CarHMM_to_OU}
\end{align}

\end{theorem}

\begin{proof}{Proof of Theorem 1}{}
Combining Equation (\ref{eqn:OU_sol}) with Equation (\ref{eqn:CarHMM_to_OU}) gives
%
\begin{align*}
    \left(Y_{(s+1) \Delta t} | X_{s\Delta t} = i \right) \sim \mathcal{N}\left((1-\phi^{(i)}) \mu^{(i)} + \phi^{(i)} Y_{s \Delta t}, \enspace \left(\sigma^{(i)}\right)^2 \right),\\
    s = 0, \ldots, T-2.
\end{align*}
%
If $X_t$ follows a Markov chain with transitions at the observation times, then the behavioural state $X_t$ does not change between observations and we can re-index $Y_{(s-1) \Delta t} = Y'_s$ and $X_{(s-2)\Delta t} = X'_s$ for $s = 2,\ldots T$, yielding the desired result:

\begin{align*}
    \left(Y'_s| X'_s = i \right) \sim \mathcal{N}\left((1-\phi^{(i)}) \mu^{(i)} + \phi^{(i)} Y'_{s-1}, \enspace \left(\sigma^{(i)}\right)^2 \right)\\
    s = 2, \ldots, T.
\end{align*}
%
$X'$ is assumed to follow a (possibly unobserved) Markov chain, and the distribution of $(Y'_s|X'_s,Y'_{s-1})$ is consistent with that of a CarHMM with Normal emission distributions.
\end{proof}

\fi