% !TeX root = ../main.tex

%\section{Data Collection and Model Construction}

To illustrate the process of constructing a model using these HMM building blocks, we investigate the dive behavior of a Northern Resident Killer Whale (NRKW) off the coast of British Columbia, Canada, and construct several candidate generative models to describe the associated data.

Understanding animal behaviour is important for conservation efforts, as environmental changes often directly impact behaviour \citep{Sutherland:1998}. NRKW prefer to feed on calorie-rich Chinook salmon (\textit{Oncorhynchus
tshawytscha}) \citep{Ford:2006}, but Chinook occur deeper than other prey, requiring significant energy expenditure from NRKW for predation \citep{Williams:2009,Noren:2011}. It is therefore of interest to determine the energetic expenditure of NRKW to forage for Chinook, especially since many Chinook salmon in the area are either threatened or endangered \citep{Ford:2015}. Acceleration data can be used to estimate an animal's energy expenditure \citep{Green:2009,Wilson:2019}, but studies suggest that behavioral state must be taken into account to obtain accurate estimates \citep{Dot:2016}. Therefore, understanding both the behavioral state of the killer whale and the distribution of acceleration within each behavioral state is important to understand the true energetic requirements of a NRKW.

\subsection{Data collection and preprocessing}

The data used in this study were collected on September 2, 2019 from 12:49 pm to 6:06 pm and consist of depth and acceleration in three orthogonal directions. Observations were collected at a rate of 50 Hz. Tagging the killer whale caused anomalous behavior before 1:20 pm and after 6:00 pm, so observations in these time periods are ignored. In addition, the tagging technology malfunctioned between 2:25pm and 2:37pm as well as between 4:07 and 5:07 pm, so all dives within this time range are ignored as well. A killer whale ``dive" is any continuous section of data that occurs below 0.5 meters in depth and lasts for at least 10 seconds. Accelerometer and depth data are smoothed by taking a moving average within a window of 1/10th of a second. Data preprocessing is done in part with the \textit{divebomb} package in Python \citep{Nunes:2018}. The preprocessed data contain a total of 267 dives, all of which are displayed in Figure \ref{fig:data}. Each dive is treated as one curve, and the sequence of dives makes up the coarse-scale process. Specifically, $Y$ is a time series of dive durations in seconds ($s$) with corresponding hidden dive types $X$. The fine-scale process $Y^*$ is the within-dive acceleration data in meters per second squared ($m/s^2$) with corresponding subdive states $X^*$.

\subsection{Model Selection}

A lag plot is used to determine where auto-correlation should be incorporated into the model (see Section S-2.1). The coarse-scale level exhibits no significant amount of auto-correlation and no obvious intricate structure prior to fitting the model. As such, we select a simple HMM to model the coarse-scale process. The dive durations $Y_t$ are assumed to follow a gamma distribution with unknown parameters $\{\mu,\sigma\}$:
%
$$\mathbb{E}(Y_t|X_t = i) = \mu^{(i)}, \qquad \mathbb{V}(Y_t|X_t = i) = \left(\sigma^{(i)}\right)^2.$$

The fine-scale acceleration data exhibits significant sinusoidal behavior; thus $Y^*$ is transformed to $Z^*$ using the DFT summary statistics. A window size of two-seconds is selected since it appears be the smallest window size that comfortably captures at least one full period of each ``wiggle" in the acceleration data. The acceleration data was also collected in three dimensions which together represent the complete range of movement of an animal (forward/backward, upward/downward, right/left). Tri-axial acceleration readings are common in these types of tags \citep{Cade:2017,Fehlmann:2017,Wright:2017}. As a result, the transformation from the raw accelerometer data $Y^*$ to $Z^*$ is calculated slightly differently than in Equation (\ref{eqn:z}):
%
\begin{equation*}
    Z_{t,t^*}^{*(1)} := \text{Re}\left(DFT\{Y^*_{t,t^*}\}(0)\right), \qquad Z_{t,t^*}^{*(2)} := \frac{1}{h}\sum_{k=1}^{\tilde{\omega}}\bigg|\bigg|DFT\{Y^*_{t,t^*}\}(k)\bigg|\bigg|^2.
\end{equation*}
%
Each observation is made up of $Z_{t,t^*}^{*(1)}$, a 3-dimensional vector, and $Z_{t,t^*}^{*(2)}$, a scalar. We calculate $Z_t^{*(2)}$ by summing the first $\tilde \omega = 10$ Fourier modes of each window, which corresponds to a maximum frequency of 5 Hz. 

Even after accounting for fine-scale structure in the raw acceleration data, there is still strong auto-correlation within each axis of $Z^{*(1)}_{t,t^*}$ prior to fitting the model (Section S-2.1). Therefore, a CarHMM is used to model $Z^{*(1)}_{t,t^*}$. In particular, if $X^*_{t,t^*} = i^*$, then we assume $Z^{*(1)}_{t,t^*}$ is Normally distributed with unknown parameters $\phi_1^{*(i^*)} \in \mathbb{R}$, $\mathbf{\mu}_1^{*(i^*)} \in \mathbb{R}^3$, and $\mathbf{\sigma}_1^{*(i^*)} \in \mathbb{R}^3$. In words, each axis of acceleration has its own mean and variance, but all axes share an auto-correlation parameter.
%
%$$\mathbb{E}(\mathbf{Z}^{*(1)}_{t,t^*}|\mathbf{Z}^{*(1)}_{t,t^*-1} = \mathbf{z}, X^*_{t,t^*} = i) = \phi_1^{*(i)} \mathbf{z} + (1-\phi_1^{*(i)}) \mathbf{\mu}_1^{*(i)},$$
%$$\mathbb{V}(\mathbf{Z}^{*(1)}_{t,t^*}|\mathbf{Z}^{*(1)}_{t,t^*-1} = z,X^*_{t,t^*} = i) = \text{diag}\left[\left(\mathbf{\sigma}_1^{*(i)}\right)^2\right].$$
%

While $Z^{*(2)}$ also exhibits some auto-correlation prior to fitting the model, the relationship is less strong than that of $Z^{*(1)}$, and the biological interpretation of auto-correlation within $Z^{*(2)}$ is less clear. In addition, much of the auto-correlation evident from the lag plot may be explained by subsequent observations sharing a subdive state $X^*_{t,t^*}$. Therefore, a simple HMM is used to model $Z^{*(2)}$. Given that $X^*_{t,t^*} = i^*$, the distribution of $Z^{*(2)}_{t,t^*}$ is assumed to be gamma and parameterized by its mean $\mu_2^{*(i^*)}$ and variance $\sigma_2^{*(i^*)}$.
%
%$$\mathbb{E}(Z^{*(2)}_{t,t^*}|Z^{*(1)}_{t,t^*-1} = z,X^*_{t,t^*} = i) = \mu_2^{*(i)}$$
%$$\mathbb{V}(Z^{*(2)}_{t,t^*}|Z^{*(1)}_{t,t^*-1} = z,X^*_{t,t^*} = i) = \left(\sigma_2^{*(i)}\right)^2.$$
%

We do not use information criteria to select the number of dive types and subdive types since they tend to overestimate the number of states in biological processes \citep{Pohle:2017}. If the hidden states are well-separated, a lag plot reveals $N$ distinct patterns corresponding to each dive type and $N^*$ distinct patterns corresponding to each subdive state \cite{Lawler:2019}. This is unfortunately not the case for our killer whale data, but we still consult our lag plots together with a heuristic approach to choose $N = 2$ dive types and $N^* = 3$ subdive behaviors. The absence of a more principled method highlights the importance of model validation techniques in lieu of information criteria (see section \ref{subsec:model_validation}). 

The final model is a hierarchical hidden Markov model with a traditional HMM on the coarse scale. On the fine scale, $Y^*$ is transformed to $Z^*$ using a DFT, $Z^{*(1)}$ is modelled with a CarHMM, and $Z^{*(2)}$ is modeled with an HMM. We also force the fine-scale emission parameters to be shared across coarse-scale hidden states $\left(\theta^{*(1,i^*)} = \ldots = \theta^{*(N,i^*)} \text{ for } i^* = 1,2,3\right)$ to reduce model complexity. This implies that dive types 1 and 2 share subdive states. In total, the parameters to estimate are
%
\begin{gather*}
    \Gamma, \qquad \Gamma^{*} = \{\Gamma^{*(1)},\Gamma^{*(2)}\} \qquad \text{(probability transition matrices)}, \\
    %
    \Theta = \{\{\mu^{(1)},\sigma^{(1)}\},\{\mu^{(2)},\sigma^{(2)}\}\} \qquad \text{($Y$ emission parameters), and} \\
    %
    \Theta^* = \{\theta^{*(\cdot,1)},\theta^{*(\cdot,2)},\theta^{*(\cdot,3)}\}  \qquad \text{(fine-scale emission parameters), where} \\
    %
    \theta^{*(\cdot,i^*)} =  \{\{\mu_1^{*(\cdot,i^*)},\sigma_1^{*(\cdot,i^*)},\phi_1^{*(\cdot,i^*)}\},\{\mu_2^{*(\cdot,i^*)},\sigma_2^{*(\cdot,i^*)}\}\} \qquad \text{(}Z^{*(1)} \text{ and } Z^{*(2)} \text{ parameters).}
\end{gather*}
%
Recall that $\theta_j^{*(\cdot,i^*)}$ are the parameters describing the distribution of $Z^{*(j)}_{t,t^*}$ when conditioning on $X^*_{t,t^*} = i^*$. 

We refer to this final model as the \textbf{CarHHMM-DFT} since it has elements of all HMM models discussed so far. The likelihood of this model is easy to calculate using the forward algorithm, and it can be maximized with respect to the parameters above. See the appendix for details of likelihood evaluation. Figure \ref{fig:CarHHMM-DFT} also shows the corresponding graphical model for the CarHHMM-DFT. In addition to the CarHHMM-DFT we consider three similar variations for comparison:
\begin{enumerate}
    \item An \textbf{HHMM-DFT}, which models the coarse-scale observations with an HMM and transforms the fine-scale observations using the DFT, but models \textit{both} $Z^{*(1)}$ \textit{and} $Z^{*(2)}$ with a simple HMM rather than a CarHMM.
    \item A \textbf{CarHHMM}, which models the coarse-scale observations with an HMM, transforms the fine-scale observations using the DFT, and models $Z^{*(1)}$ using a CarHMM. However, the Fourier sums $Z^{*(2)}$ are omitted from this model altogether. The CarHHMM therefore does not know about the ``wiggliness" of the accelerometer data.
    \item A \textbf{CarHMM-DFT}, which models the coarse-scale observations \textit{as an independent and identically distributed sequence of dives}, transforms the fine-scale observations using the DFT, and models $Z^{*(1)}$ with a CarHMM and $Z^{*(2)}$ with an HMM. This model assumes that there is only one dive type, so the fine-scale probability transition matrix $\Gamma^*$ is the same for every dive. 
\end{enumerate}
All models have emission distributions identical to the CarHHMM-DFT when applicable. Each of the three candidate models above leave out one important aspect of the full CarHHMM-DFT: the HHMM-DFT is missing auto-correlation within the fine-scale observations, the CarHHMM does not have access to the ``wiggliness" $(Z^{*(2)})$, and the CarHMM-DFT lacks a hierarchical structure.