% !TeX root = ../main.tex

%\section{Data Collection and Model Construction}

To illustrate the process of constructing a model using these HMM building blocks, we investigate the dive behaviour of a Northern Resident Killer Whale (NRKW) off the coast of British Columbia, Canada, and construct several candidate generative models to describe the associated data.

Understanding animal behaviour is important for conservation efforts, as environmental changes often directly impact behaviour \citep{Sutherland:1998}. HMMs have been used to understand how diving behaviours of various species are affected by disturbances (e.g. \citep{DeRuiter:2017} and \citep{Isojunno:2017}). For NRKWs, we are interested in categorizing different diving behaviours, with a particular focus on differentiating potential foraging dives. NRKWs prefer to feed on calorie-rich Chinook salmon (\textit{Oncorhynchus tshawytscha}) \citep{Ford:2006}, but Chinook occur deeper than other prey, requiring significant energy expenditure from NRKW for predation \citep{Williams:2009,Noren:2011}. It is therefore of interest to determine the energetic expenditure of NRKW to forage for Chinook, especially since many Chinook salmon in the area are either threatened or endangered \citep{Ford:2015}. Acceleration data can be used to estimate an animal's energy expenditure \citep{Green:2009,Wilson:2019}, but studies suggest that the behavioural state must be taken into account to obtain accurate estimates \citep{Dot:2016}. Therefore, understanding both the behavioural state of the killer whale and the distribution of acceleration within each behavioural state is important to understand the true energetic requirements of a NRKW.

\subsection{Data collection and preprocessing}

The data used in this study were collected on September 2, 2019 from 12:49 pm to 6:06 pm and consist of depth and acceleration in three orthogonal directions. Observations were collected at a rate of 50 Hz using
% an accelerometer and time-depth recorder.
[NAME OF TAGS - do you know Sarah?]. The acceleration data were collected in three dimensions, which together represent the complete range of movement of an animal (forward/backward, upward/downward, right/left). Tri-axial acceleration readings are common in these types of tags \citep{Cade:2017,Fehlmann:2017,Wright:2017}. Tagging the killer whale caused anomalous behaviour before 1:20 pm and after 6:00 pm, so observations in these time periods are ignored. We also ignore all dives between 2:25 pm and 2:37 pm and between 4:07 pm and 5:07 pm, when the tagging technology malfunctioned. A killer whale ``dive" is any continuous section of data that occurs below 0.5 meters in depth and lasts for at least 10 seconds. We smooth the accelerometer and depth data by taking a moving average within a window of $1/10^{th}$ of a second. Data are preprocessed in part with the \textit{divebomb} package in Python \citep{Nunes:2018}. The preprocessed data contain a total of 267 dives, all of which are displayed in Figure \ref{fig:data}. Each dive is treated as one curve, and the sequence of dives makes up the coarse-scale process. Specifically, the vector of observed coarse-scale observations $y = \Big(y_1,\ldots,y_{267}\Big)$ is a time series of dive durations in seconds ($s$) with corresponding vector of unobserved dive types $x = \Big(x_1,\ldots,x_{267}\Big)$. For dive $t^*$, the fine-scale observations are $y^*_{t^*}$, the vector of the within-dive acceleration data in meters per second squared ($m/s^2$).  The corresponding vector of subdive states is $x_{t^*}^*$.  As usual, $y^* = \Big\{y^*_1,\ldots,y^*_{267}\Big\}$ and $x^* = \Big\{x^*_1,\ldots,x^*_{267}\Big\}$ is the corresponding set of subdive states.

\subsection{Model development and selection}

Building an suitable model for these data involves selecting appropriate model structures, emission distributions, and numbers of states on both the coarse and fine scales. To choose how to model the coarse-scale observations, we first inspect lag plots to determine whether auto-correlation should be incorporated in the model structure (see section 2.1 in the supplementary material). The coarse-scale level exhibits no significant amount of auto-correlation and no obvious intricate structure prior to fitting the model. As such, we select a simple HMM to model the coarse-scale process.
Given that dive $t$ has type $i$, the dive duration $Y_t$ is assumed to follow a gamma distribution with unknown parameters $\mu^{(i)}$ and $\sigma^{(i)}$:
%
$$\mathbb{E}(Y_t|X_t = i) = \mu^{(i)}, \qquad \mathbb{V}(Y_t|X_t = i) = \left(\sigma^{(i)}\right)^2.$$

In contrast, the fine-scale acceleration data exhibits significant sinusoidal behaviour. Thus, we transform $Y^*$ to $Z^*$ using the DFT for this model. We select a window size of two seconds since it appears be the smallest window size that comfortably captures at least one full period of each ``wiggle" in the acceleration data. As a result of the tri-axial acceleration readings, the transformation from the raw accelerometer data $Y^*$ to $Z^*$ is calculated slightly differently than in Equation (\ref{eqn:z}):
%
\begin{equation}
    \label{eqn:z0}
    \Zone_{t,t^*} := \frac{1}{h}\sum_{n=0}^{h-1}Y^*_{t,t^*+n}, \qquad \Ztwo_{t,t^*} := \frac{1}{h}\sum_{k=1}^{\tilde{\omega}}\bigg|\bigg|DFT\{Y^*_{t,t^*}\}(k)\bigg|\bigg|^2.
\end{equation}
%
Each observation is made up of $\Zone_{t,t^*}$ (a 3-dimensional vector of average acceleration) and $\Ztwo_{t,t^*}$ (a scalar describing ``wiggliness"). We calculate $\Ztwo_{t,t^*}$ by summing the first $\tilde \omega = 10$ Fourier modes of each window, which corresponds to a maximum frequency of 5 Hz. 

Even after accounting for fine-scale structure in the raw acceleration data, there is still strong auto-correlation within each axis of $\Zone_{t,t^*}$ prior to fitting the model (see section 2.1 of the supplementary material). Therefore, a CarHMM is used to model $\Zone_{t,t^*}$. We assume that $\Zone_{t,t^*}$ is Normally distributed with unknown parameters $\phi_A^{*(i^*)} \in \mathbb{R}$, $\mathbf{\mu}_A^{*(i^*)} \in \mathbb{R}^3$, and $\mathbf{\sigma}_A^{*(i^*)} \in \mathbb{R}^3$, where the $j^{th}$ component of $\Zone_{t,t^*}$ follows
%
\begin{align*}
    \mathbb{E}\left[\left(\Zone_{t,t^*}\right)_j\Big|\left(\Zone_{t,t^*-1}\right)_j ,X^*_t=i^*\right] &= \phi_A^{*(i^*)} ~ \Big(\Zone_{t,t^*-1}\Big)_j ~+ ~(1-\phi_A^{*(i^*)}) ~\left(\mu_A^{*(i^*)}\right)_j,\\
    %
    \mathbb{V}\left[\left(\Zone_{t,t^*}\right)_j\Big|\left(\Zone_{t,t^*-1}\right)_j ,X^*_t=i^*\right] &= \left(\sigma^{*(i^*)}\right)^2_j.
\end{align*}
%
Each axis of acceleration has its own mean and variance, but all axes have the same auto-correlation parameter. We assume that all three components of $\Zone$ are independent.
%
%$$\mathbb{E}(\Zone_{t,t^*}|\\Zone_{t,t^*-1} = \mathbf{z}, X^*_{t,t^*} = i) = \phi_A^{*(i)} \mathbf{z} + (1-\phi_A^{*(i)}) \mathbf{\mu}_A^{*(i)},$$
%$$\mathbb{V}(\Zone_{t,t^*}|\Zone_{t,t^*-1} = z,X^*_{t,t^*} = i) = \text{diag}\left[\left(\mathbf{\sigma}_A^{*(i)}\right)^2\right].$$
%
While $\Ztwo$ also exhibits some auto-correlation prior to fitting the model, the relationship is less strong than that of $\Zone$, and the biological interpretation of auto-correlation within $\Ztwo$ is less clear. In addition, much of the auto-correlation evident from the lag plot may be explained by subsequent observations sharing a subdive state $X^*_{t,t^*}$. Therefore, a simple HMM is used to model $\Ztwo$. Given that $X^*_{t,t^*} = i^*$, the distribution of $\Ztwo_{t,t^*}$ is assumed to be gamma distributed and parameterized by its mean $\mu_W^{*(i^*)} \in \mathbb{R}$ and standard deviation $\sigma_W^{*(i^*)} \in \mathbb{R}$.
%
%$$\mathbb{E}(\Ztwo_{t,t^*}|\Zone_{t,t^*-1} = z,X^*_{t,t^*} = i) = \mu_W^{*(i)}$$
%$$\mathbb{V}(\Ztwo_{t,t^*}|\Zone_{t,t^*-1} = z,X^*_{t,t^*} = i) = \left(\sigma_W^{*(i)}\right)^2.$$
%

We do not use information criteria to select the number of dive types and subdive types since they tend to overestimate the number of states in biological processes \citep{Pohle:2017}. If the hidden states are well-separated, the lag plot should reveal $N$ distinct patterns corresponding to each dive type and $N^*$ distinct patterns corresponding to each subdive state \citep{Lawler:2019}. This is unfortunately not the case for our killer whale data, but we still consult our lag plots together with a heuristic approach to choose $N = 2$ dive types and $N^* = 3$ subdive behaviours. The absence of a more principled method highlights the importance of model validation techniques in lieu of information criteria (see section \ref{subsec:model_validation}). 

The final model is a hierarchical hidden Markov model with a traditional HMM on the coarse scale. On the fine scale, $Y^*$ is transformed to $Z^*$ using Equation (\ref{eqn:z0}), $\Zone$ is modelled with a CarHMM, and $\Ztwo$ is modelled with an HMM. We assume that all components of $\Zone$ and $\Ztwo$ are independent. We also force the fine-scale emission parameters to be shared across coarse-scale hidden states $\left(\theta^{*(1,i^*)} = \theta^{*(2,i^*)} \text{ for } i^* = 1,2,3\right)$ to reduce model complexity. This implies that dive types 1 and 2 share subdive states. In total, the parameters to estimate are
%
\begin{gather*}
    \Gamma, \qquad \Gamma^{*} = \{\Gamma^{*(1)},\Gamma^{*(2)}\} \qquad \text{(probability transition matrices)}, \\
    %
    \Theta = \{\{\mu^{(1)},\sigma^{(1)}\},\{\mu^{(2)},\sigma^{(2)}\}\} \qquad \text{($Y$ emission parameters), and} \\
    %
    \Theta^* = \{\theta^{*(\cdot,1)},\theta^{*(\cdot,2)},\theta^{*(\cdot,3)}\}  \qquad \text{(fine-scale emission parameters), where} \\
    %
    \theta^{*(\cdot,i^*)} =  \{\{\mu_A^{*(\cdot,i^*)},\sigma_A^{*(\cdot,i^*)},\phi_A^{*(\cdot,i^*)}\},\{\mu_W^{*(\cdot,i^*)},\sigma_W^{*(\cdot,i^*)}\}\}.
\end{gather*}
%
Recall that $\theta^{*(\cdot,i^*)}$ is the set of parameters describing the distribution of $Z_{t,t^*}$ when conditioned on $X^*_{t,t^*} = i^*$. 

We refer to this final model as the \textbf{CarHHMM-DFT} since it has elements of all HMMs discussed so far. Figure \ref{fig:CarHHMM-DFT} shows the corresponding graphical model for the CarHHMM-DFT. The likelihood of this model is easy to calculate using the forward algorithm, and it can be maximized with respect to the parameters above. See the appendix for details of likelihood evaluation. In addition to the CarHHMM-DFT, we consider three similar variations for comparison:
\begin{enumerate}
    \item An \textbf{HHMM-DFT}, which models the coarse-scale observations with an HMM and transforms the fine-scale observations using Equation (\ref{eqn:z0}), but models both $\Zone$ and $\Ztwo$ with a simple HMM rather than a CarHMM.
    \item A \textbf{CarHHMM}, which models the coarse-scale observations with an HMM, transforms the fine-scale observations using Equation (\ref{eqn:z0}), and models $\Zone$ using a CarHMM. However, the Fourier sums $\Ztwo$ are omitted from this model altogether. The CarHHMM therefore does not know about the ``wiggliness" of the accelerometer data.
    \item A \textbf{CarHMM-DFT}, which models the coarse-scale observations as an independent and identically distributed sequence of dives, transforms the fine-scale observations using Equation (\ref{eqn:z0}), and models $\Zone$ with a CarHMM and $\Ztwo$ with an HMM. This model assumes that there is only one dive type, so the fine-scale probability transition matrix $\Gamma^*$ is the same for every dive. 
\end{enumerate}
%
Each of the three candidate models above leave out one important aspect of the full CarHHMM-DFT: the HHMM-DFT is missing auto-correlation within the fine-scale observations, the CarHHMM does not have access to the ``wiggliness" $\Ztwo$, and the CarHMM-DFT lacks a hierarchical structure.

\subsection{Results}

We fit all four models to the data shown in Figure \ref{fig:data} to test their performance in a real-world setting. We first report the results from the full CarHHMM-DFT in detail and then asses the quality of the fit. We also compare the results of the CarHHMM-DFT with those of the other candidate models.

The emission distribution parameter estimates suggest that the killer whale has at least two distinct dive behaviours (Table \ref{table:emis_dists_CarHHMM-DFT}, Figure \ref{fig:coarse_emis}). 
Dive type 1 corresponds to shorter, shallower dives which may serve a variety of purposes including resting, traveling, and searching for prey.
On the other hand, dive type 2 is longer and deeper, and these dives have been associated with behaviours such as foraging and socialization \citep{Tennessen:2019b}. No dive in this data set has a maximum depth deeper than $\approx$ 30 meters, and \citet{Wright:2017} note that most prey captures occur at depths deeper than 100m. It is therefore unlikely that any of the dives used in this study are successful foraging dives.

The three subdive behaviours have distinct average ``wiggliness" values, as the means of $\Ztwo$ for each state are separated by an order of magnitude (Table \ref{table:emis_dists_CarHHMM-DFT}, Figure \ref{fig:fine_emis}). 
Subdive state 1 has the smallest mean of $\Ztwo$ and the smallest variance corresponding to $\Zone$. It also has the highest auto-correlation in $\Zone$. This implies less overall activity and more consistent acceleration compared to the other subdive states. 
Subdive state 2 has a mean ``wiggliness" ($\Ztwo$) that is one order of magnitude higher than subdive state 1 and its acceleration has about twice the variance compared to subdive state 1. The auto-correlation of acceleration is also slightly lower in than subdive state 1. We therefore hypothesize that subdive type 2 corresponds to fluking (active swimming), as strong sinusoidal behaviour in acceleration is characteristic of this behaviour in marine mammals \citep{Simon:2012}.
Finally, the mean of $\Ztwo$ and variance of $\Zone$ in subdive state 3 is much higher than the other two states. The auto-correlation of $\Zone$ is also much lower, implying more variation in acceleration between segments. This corresponds to vigorous swimming activity, especially as the killer whale begins or ends a dive (see Figure \ref{fig:labeled_dives}). 

The estimated probability transition matrices and associated stationary distributions are
%
$$\hat \Gamma = \begin{pmatrix} 
0.788 & 0.212 \\
0.809 & 0.191
\end{pmatrix} \text{ and }$$
$$\hat \delta = \begin{pmatrix} 0.792 & 0.208 \end{pmatrix}$$
%
for the transitions between dives. Within dives, we have
$$\hat \Gamma^{*(1)} = \begin{pmatrix} 
0.679 & 0.321 & 0.000 \\
0.038 & 0.904 & 0.058 \\
0.000 & 0.232 & 0.768
\end{pmatrix}, \qquad 
\hat \Gamma^{*(2)} = \begin{pmatrix} 
0.859 & 0.141 & 0.000 \\
0.114 & 0.841 & 0.045 \\
0.000 & 0.216 & 0.784
\end{pmatrix},$$
$$\hat \delta^{*(1)} = \begin{pmatrix} 0.087 & 0.731 & 0.182 \end{pmatrix}, \enspace \text{and} \enspace \hat \delta^{*(2)} = \begin{pmatrix} 0.401 & 0.496 & 0.103 \end{pmatrix}$$
%
for dive types 1 and 2.
The estimated coarse-scale stationary distribution, $\hat{\delta}$, indicates that most dives are short, type 1 dives. The estimated coarse-scale transition probability matrix $\hat \Gamma$ further indicates that the whale usually performs multiple short type 1 dives before doing only one (or a few) long type 2 dive. This finding is consistent with those of \citet{Tennessen:2019b} and \citet{Williams:2009}, both of whom describe common bouts of short resting dives before a killer whale performs a longer, more energy-intensive deep dive. The fine-scale stationary distributions $\hat{\delta}^{*(i^*)}$ indicate that this killer whale is more likely to be in the less active subdive state 1 when performing long deep dives than when performing short shallow dives. Using less active swimming behaviour is consistent with the need for marine mammals to conserve energy when diving at depth and holding breath for long periods of time \citep{Williams:1999,Hastie:2006}. Figure \ref{fig:labeled_dives} shows the decoded dive behaviour of six selected dives. The supplementary material also shows the estimated probability of each dive and subdive state given the data (see section 2.2 of the supplementary material).

\subsection{Model validation}
\label{subsec:model_validation}

We use two visual tools to evaluate this model: pseudo-residual plots and empirical histograms. A pseudo-residual is the marginal cumulative distribution function of an observation conditioned on all other observations \citep{Zucchini:2016}. To easily visualize outliers, pseudo-residuals are often passed through the quantile function of the standard Normal distribution. Mathematically, the pseudo-residual of an observation $y_t$ for a traditional HMM is equal to $\Phi^{-1} \left(Pr(Y_t < y_t|\{Y_1,\ldots,Y_T\}/\{Y_t\}) \right)$, where $\Phi$ is the cumulative distribution function of a standard Normal distribution. If the model is correct, then all pseudo-residuals are independent and follow a standard Normal distribution. Histograms of the pseudoresiduals mostly support that this model is well-specified. One exception is $\Ztwo$, whose pseudo-residuals are noticeably right-skewed (Figure \ref{fig:pseudoresids}). This implies that the true distribution of $\Ztwo$ may follow a heavier-tailed distribution than the gamma distribution used in the case study. See section 2 of the supplementary material for pseudo-residual plots for all observations.

We also plot histograms of $Y$ corresponding to each dive type in Figure \ref{fig:empirical_dist}. Observations are weighted by the decoded probability that they correspond to a particular dive type. This results in 2 histogram corresponding to $Y$ - one for dive type 1 and another for dive type 2. These histograms are then plotted over their corresponding emission distribution learned by the CarHHMM-DFT (see Figure \ref{fig:empirical_dist}). See section 2 of the supplementary material for histograms corresponding to the fine-scale observations. Our results again mostly support a well-specified model with some exceptions. In particular, $\Ztwo$, is again right-skewed while $\Zone$ has heavy tails in subdive state 3, indicating the existence of rare events corresponding to exceptionally violent thrashing of the killer whale. These outliers are potential subjects for future study and may indicate biologically relevant phenomena such as prey capture \citep{Tennessen:2019a}.

\subsection{Comparison with candidate models}

The CarHMM-DFT, which lacks a hierarchical structure, produces very similar parameter estimates and state estimates compared to those of the CarHHMM-DFT on the fine scale. However, its lack of a hierarchical structure means that it fails to differentiate between short and long dives. This model therefore does not give any information regarding the dive-level Markov chain or the relationship between the dive and subdive levels. For example, the CarHMM-DFT does not indicate that the whale is much more likely to be in subdive state 1 when engaged in longer dives compared to shorter one.

The HHMM-DFT, which ignores auto-correlation in $\Zone$, decodes dive types and subdive states similarly to the CarHHMM-DFT, but it is less likely to categorize the behaviour at the beginning and end of dives as subdive type 3. In addition, the HHMM-DFT produces estimates of $\sigma_A^{*(1)}$, $\sigma_A^{*(2)}$, and $\sigma_A^{*(3)}$ (the standard deviation of acceleration) which are approximately 50-100 \% larger than those of the CarHHMM-DFT across all axes of acceleration. The estimated uncertainties of the estimates of $\mu_A^{*(1)}$, $\mu_A^{*(2)}$, and $\mu_A^{*(3)}$ are also less than half of those for the CarHHMM-DFT across all axes of acceleration. Further, the empirical auto-correlation of $\Zone$ within each decoded subdive state was at least 0.5 for all acceleration axes and subdive states, and the pseudoresiduals of the HHMM-DFT contain several outliers. These findings suggest that the HHMM-DFT is a significantly worse fit to this data than the full CarHHMM-DFT.

Finally, the CarHHMM does not use the ``wiggliness" of the acceleration data as an observation, so it regularly fails to pick up obvious behavioural changes corresponding to the periodicity shown in Figure \ref{fig:labeled_dives}. This fact alone essentially disqualifies the CarHHMM as a viable model for this data. The pseudoresiduals corresponding to the CarHHMM also appear to be less heavy-tailed than a normal distribution. See section 2 of the supplementary material for a more complete set of results for each of the candidate models.