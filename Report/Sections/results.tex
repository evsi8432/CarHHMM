% !TeX root = ../main.tex

%\section{Data Collection and Model Construction}

To illustrate the process of constructing a model using these building blocks, we analyze the dive behaviour of a northern resident killer whale in Queen Charlotte Sound off the coast of British Columbia, Canada, and construct several candidate models to categorize and describe its diving behaviour.

Understanding animal behaviour is important for conservation efforts, as environmental changes caused by anthropogenic activity can directly impact animal behaviour \citep{Sutherland:1998}. HMMs have been used to understand how diving behaviours of various species are affected by disturbances (e.g. \citet{DeRuiter:2017} and \citet{Isojunno:2017}). For killer whales, we are interested in categorizing different diving behaviours to identify potential foraging dives. Northern resident killer whales feed almost exclusively on calorie-rich Chinook salmon (\textit{Oncorhynchus tshawytscha}) \citep{Ford:2006}, which typically occur deeper and are less numerous than smaller types of salmon \citep{Ford:2009}. Northern resident killer whales therefore must expend significant amounts of energy to capture Chinook \citep{Williams:2009,Noren:2011,Wright:2017}. 
%It is therefore of interest to determine the energetic expenditure of killer whales hunting for Chinook as well as the frequency at which they catch prey, especially since many Chinook salmon in the area are either threatened or endangered \citep{Ford:2015}. 
Acceleration data can be used to estimate an animal's energy expenditure \citep{Green:2009,Wilson:2019}, behavioural state must be accounted for in order to obtain accurate estimates \citep{Dot:2016}. Therefore, understanding both the behavioural state of the killer whale and the distribution of acceleration within each behavioural state is needed to determine the true energetic requirements of a killer whale.

\subsection{Data collection and preprocessing}

The data we use were collected on September 2, 2019 from 12:49 pm to 6:06 pm PDT and consist of depth and acceleration in three orthogonal directions. Observations were collected at a rate of 50 Hz using CATs biologger (Customizable Animal Tracking Solutions, www.cats.is). The acceleration was measured in three dimensions, which together represent the complete range of movement of an animal (forward/backward, upward/downward, right/left). Tri-axial acceleration readings are common in these types of tags and are often used to infer animal behaviour such as foraging \citep{Cade:2017,Fehlmann:2017,Wright:2017}. The act of attaching and detaching the tag caused anomalous behaviour before 1:20 pm and after 6:00 pm, so observations taken during these time periods are ignored. There were also periods of time when the tag failed to record observations, resulting in data gaps between 2:25 pm and 2:37 pm and between 4:07 pm and 5:07 pm. To process the data, we smooth the accelerometer and time-depth recorder data by taking a moving average within a window of $1/10^{th}$ of a second. We then define a killer whale ``dive" as any continuous interval of data that occurs below 0.5 meters in depth and lasts for at least 10 seconds. Data are preprocessed in part with the \textit{divebomb} package in Python \citep{Nunes:2018}. The preprocessed data contain a total of 267 dives, all of which are displayed in Figure \ref{fig:data}. Each dive is treated as one curve, and the sequence of dives makes up the coarse-scale process. Specifically, the vector of observed coarse-scale observations $y = \left\{y_1,\ldots,y_{267}\right\}$ is a time series of dive durations in seconds. For dive $t$, the fine-scale observations are contained in $y^*_{t} \equiv \left\{y^*_{t,1},\ldots \right\}$, which is a vector of the within-dive acceleration data in units of meters per second squared. Naturally, the collection of all acceleration data is denoted as $y^* = \left\{y^*_1,\ldots,y^*_{267}\right\}$.

\subsection{Model definition and selection}
\label{subsec:model_selection}

Defining a suitable model to describe this killer whale kinematic data set involves selecting an appropriate number of hidden states, model structure, and emission distributions for both the coarse and fine scale observations.

We do not use information criteria to select the number of dive types $N$ since these metrics tend to overestimate the number of states in biological processes \citep{Pohle:2017}. We instead plot the duration of each dive versus the duration of the dive preceding it (i.e., $y_t$ vs $y_{t-1}$ for $t \geq 2$). This type of visualization is known as a lag plot. If the hidden states are well-separated, a lag plot should reveal $N$ distinct patterns, where each pattern corresponds to one dive type \citep{Lawler:2019}. This is unfortunately not the case for our killer whale data, as there is one cluster of data centred at approximately $30$ seconds. However, long dives appear to be characterized by bouts of less ``wiggly" behaviour in the acceleration data compared to shorter dives, so we choose $N = 2$ to differentiate these dive types. The absence of a more principled method to select $N$ highlights the importance of model validation techniques in lieu of information criteria (see Section \ref{subsec:model_validation}).
%
Prior to fitting the model, lag plots reveal no significant auto-correlation between dive duration observations (see Section 1.1 of the supplementary material), and visual inspection shows no obvious intricate structure. Therefore, we select a simple HMM to model the coarse-scale process since neither a CarHMM nor a moving-window transformation is called for.
%
Given that dive $t$ is of type $i$, we assume that the dive duration $Y_t$ follows a Gamma distribution with unknown parameters $\mu^{(i)}$ and $\sigma^{(i)}$:
%
$$\mathbb{E}(Y_t|X_t = i) = \mu^{(i)}, \qquad \mathbb{V}(Y_t|X_t = i) = \left(\sigma^{(i)}\right)^2.$$
%
This is consistent with previous studies, including that of \citet{Barajas:2017}. 
%

We then select a model corresponding to the fine-scale observations of acceleration. Similarly to the coarse model, we rely on lag plots and visual inspection to select $N^*=3$. Although $N^*$ is selected heuristically, we test the validity of this model in Section \ref{subsec:model_validation}.
%
In contrast to the coarse-scale observations, the fine-scale acceleration data exhibit significant sinusoidal behaviour. Thus, we transform each fine-scale observation sequence $y_t^*$ into $\z_t$ using Equation (\ref{eqn:z}) with window size $h=100$ (two seconds) and maximum frequency $\tilde{\omega}=10$ (5 Hz). 
%We select a window size of two seconds since it appears be the smallest window size that captures at least one full period of each ``wiggle" in the acceleration data. 
We then have that $\z_{t,\tilde t^*} = \left\{\zone_{t,\tilde t^*},\ztwo_{t,\tilde t^*}\right\}$, where $\zone_{t,\tilde t^*}$ is a three dimensional vector of component-wise average acceleration and $\ztwo_{t,\tilde t^*}$ is a scalar describing the ``wiggliness" of a particular window. Even after transforming the raw acceleration data, there is still strong auto-correlation within each component of $\zone_{t,\tilde t^*}$ prior to fitting the model (see Section 1.1 of the supplementary material). Therefore, we choose a CarHMM as defined in Section 2.2 as the fine-scale model.

We then select the specific emission distribution of $\Z_{t,\tilde t^*}$. Given the dive types and subdive states, we assume that $\Ztwo_{t,\tilde t^*}$ and all components of $\Zone_{t,\tilde t^*}$ are independent of one another. To reduce model complexity, we also assume that the three fine-scale emission parameters are shared across the two dive types $\left(\theta^{*(1,i^*)} = \theta^{*(2,i^*)} \equiv \theta^{*(\cdot,i^*)} \text{ for } i^* = 1,2,3\right)$. This implies that the subdive states within dive type 1 have the same interpretation as those within dive type 2.

All components of $\Zone_{t,\tilde t^*}$ are assumed to be Normally distributed as in Equation (\ref{eqn:carhmm}) for all $t$ and $\tilde t^*$. Each component is assumed to have its own mean and variance parameters, but all components have the same auto-correlation parameter. 
%Given that $X^*_{t,\tilde t^*} = i^*$, the corresponding unknown parameters are then $\mathbf{\mu}_A^{*(\cdot,i^*)} \in \mathbb{R}^3$, $\mathbf{\sigma}_A^{*(\cdot,i^*)} \in \mathbb{R}_{>0}^3$, and $\phi_A^{*(\cdot,i^*)} \in [0,1]$.
Given that $\tilde X^*_{t,\tilde t^*} = i^*$, $\tilde \Ztwo_{t,\tilde t^*}$ is assumed to follow a Gamma distribution parameterized by its mean $\mu_W^{*(\cdot,i^*)}$ and standard deviation $\sigma_W^{*(\cdot,i^*)}$. We do not include $\tilde \Ztwo_{t,\tilde t^*-1}$ in the distribution of $\Ztwo_{t,\tilde t^*}$ because the auto-correlation evident from the lag plot is not severe and may be explained by subsequent observations occurring within the same subdive state. 
%
%The final model is a hierarchical hidden Markov model with a traditional HMM on the coarse scale. Each fine-scale sequence $y^*_t$ is transformed to $\z$ using Equation (\ref{eqn:z}) and $\Z_{t,\tilde t^*}$ is modelled with a CarHMM.
%We assume $\Zone_{t,\tilde t^*}$ is Normally distributed with auto-correlation while $\Ztwo_{t,\tilde t^*}$ is Gamma distributed without auto-correlation. We also assume that $\Ztwo_{t,\tilde t^*}$ and all components of $\Zone_{t,\tilde t^*}$ are each independent of one another when conditioned on the subdive state. The emission distribution of each subdive state does not depend upon the dive type. 
In total, the parameters to estimate are
%
\begin{gather*}
    \Gamma, \qquad \Gamma^{*} = \{\Gamma^{*(1)},\Gamma^{*(2)}\} \qquad \text{(probability transition matrices)}, \\
    %
    \theta = \{\mu^{(1)},\sigma^{(1)},\mu^{(2)},\sigma^{(2)}\} \qquad \text{($Y$ emission parameters), and} \\
    %
    \theta^* = \{\theta^{*(\cdot,1)},\theta^{*(\cdot,2)},\theta^{*(\cdot,3)}\}  \qquad \text{($\Z$ emission parameters), where} \\
    %
    \theta^{*(\cdot,i^*)} =  \{\mu_A^{*(\cdot,i^*)},\sigma_A^{*(\cdot,i^*)},\phi_A^{*(\cdot,i^*)},\mu_W^{*(\cdot,i^*)},\sigma_W^{*(\cdot,i^*)}\}.
\end{gather*}
%
Recall that $\theta^{*(\cdot,i^*)}$ is the set of parameters describing the distribution of $\Z_{t,\tilde t^*}$ conditioned on $\tilde X^*_{t,\tilde t^*} = i^*$. 
%
We refer to this final model as the \textbf{CarHHMM-DFT} since it includes a CarHMM, hierarchical HMM, and DFT-based transformation (see Figure \ref{fig:CarHHMM-DFT}). The likelihood of this model is easily calculated using the forward algorithm and can be maximized with respect to the parameters above (see appendix for details). In addition to the CarHHMM-DFT, we consider three similar variations for comparison. As in the full model, all components of $\Z_{t,\tilde t^*}$ are assumed to be independent when conditioned on dive type and subdive state for each of the following models:
\begin{enumerate}
    \item An \textbf{HHMM-DFT}, which models the coarse-scale observations with an HMM and transforms the fine-scale observations using Equation (\ref{eqn:z}), but models both $\Z_{t,\tilde t^*}$ and as emissions of a simple HMM rather than a CarHMM.
    \item A \textbf{CarHHMM}, which models the coarse-scale observations with an HMM, transforms the fine-scale observations using Equation (\ref{eqn:z}), and models $\Zone_{t,\tilde t^*}$ as emissions of a CarHMM. However, the ``wiggliness"  $\Ztwo_{t,\tilde t^*}$ is omitted from this model altogether.
    \item A \textbf{CarHMM-DFT}, which models the coarse-scale observations as an independent and identically distributed sequence of dives, transforms the fine-scale observations using Equation (\ref{eqn:z}), and models $\Z_{t,\tilde t^*}$ as emissions of a CarHMM. This model assumes that there is only one dive type, so the fine-scale probability transition matrix $\Gamma^*$ is the same for every dive. 
\end{enumerate}
%
Each of the three candidate models above leave out one important aspect of the full CarHHMM-DFT: the HHMM-DFT assumes there is no auto-correlation between fine-scale observations, the CarHHMM does not incorporate the ``wiggliness" $\big(\Ztwo_{t,\tilde t^*}\big)$, and the CarHMM-DFT lacks a hierarchical structure and thus does not distinguish between dive types.

\subsection{Case study results}

To illustrate an application of this method and compare the candidate models, we fit all four models to the data shown in Figure \ref{fig:data}. We first report the results from the full CarHHMM-DFT in detail and assess the quality of the fit. We then compare the results of the CarHHMM-DFT with those of the other candidate models.

The coarse-scale emission distribution parameter estimates suggest that the killer whale has at least two distinct dive behaviours (see Table \ref{table:emis_dists_CarHHMM-DFT} and Figure \ref{fig:coarse_emis}). 
Dive type 1 corresponds to shorter, shallower dives which likely reflects resting, travelling, and (to a lesser extent) searching for prey.
On the other hand, dive type 2 is longer, deeper, and may be associated with behaviours such as hunting and socialization \citep{Tennessen:2019b}. No dive in this data set has a maximum depth deeper than 30 meters, and a study by \citet{Wright:2017} of killer whales in Johnstone Strait found that most prey captures occur at depths deeper than 100 meters. However, the killer whale studied here was tagged north of Johnstone Strait in Queen Charlotte Sound, and significant numbers of fish have been observed to be caught near the surface (Fortune and Trites, unpublished data). It is therefore unclear whether any of the dives in this study are successful foraging dives.

The means of the ``wiggliness" $\Ztwo_{t,\tilde t^*}$ are each separated by an order of magnitude between subdive states (see Table \ref{table:emis_dists_CarHHMM-DFT} and Figure \ref{fig:fine_emis}). 
Subdive state 1 has the smallest mean corresponding to $\Ztwo_{t,\tilde t^*}$ and the smallest variance corresponding to $\Zone_{t,\tilde t^*}$. It also has the highest auto-correlation in $\Zone_{t,\tilde t^*}$. This implies less overall activity and more consistent acceleration compared to the other subdive states. 
Subdive state 2 has a mean ``wiggliness" ($\Ztwo_{t,\tilde t^*}$) that is one order of magnitude higher than subdive state 1 and its acceleration has about twice the variance compared to subdive state 1. The auto-correlation of acceleration is also slightly lower than in subdive state 1. We therefore hypothesize that subdive state 2 corresponds to fluking (active swimming), as strong sinusoidal behaviour in acceleration is characteristic of this behaviour in marine mammals \citep{Simon:2012}.
Finally, the mean of $\Ztwo_{t,\tilde t^*}$ and variance of $\Zone_{t,\tilde t^*}$ in subdive state 3 are both much higher than in the other two states. The auto-correlation of $\Zone_{t,\tilde t^*}$ is also much lower. This corresponds to vigorous swimming activity, especially as the killer whale begins or ends a dive (see Figure \ref{fig:labeled_dives}). 

The estimated probability transition matrices and associated stationary distributions are
%
$$\hat \Gamma = \begin{pmatrix} 
0.788 & 0.212 \\
0.809 & 0.191
\end{pmatrix} \text{ and }$$
$$\hat \delta = \begin{pmatrix} 0.792 & 0.208 \end{pmatrix}$$
%
for the transitions between dives. Within dives, we have
$$\hat \Gamma^{*(1)} = \begin{pmatrix} 
0.679 & 0.321 & 0.000 \\
0.038 & 0.904 & 0.058 \\
0.000 & 0.232 & 0.768
\end{pmatrix}, \qquad 
\hat \Gamma^{*(2)} = \begin{pmatrix} 
0.859 & 0.141 & 0.000 \\
0.114 & 0.841 & 0.045 \\
0.000 & 0.216 & 0.784
\end{pmatrix},$$
$$\hat \delta^{*(1)} = \begin{pmatrix} 0.087 & 0.731 & 0.182 \end{pmatrix}, \enspace \text{and} \enspace \hat \delta^{*(2)} = \begin{pmatrix} 0.401 & 0.496 & 0.103 \end{pmatrix}$$
%
for dive types 1 and 2.
The estimated coarse-scale stationary distribution, $\hat{\delta}$, indicates that about 79\% of dives are short dives of type 1. The estimated coarse-scale transition probability matrix $\hat \Gamma$ further indicates that the whale performs an average of 4.72 short type 1 dives before switching to dive type 2 and an average of 1.24 longer type 2 dives before switching back to dive type 1. This finding is consistent with those of \citet{Tennessen:2019b} and \citet{Williams:2009}, both of whom describe common bouts of short resting dives before a killer whale performs a longer, more energy-intensive deep dive. The fine-scale stationary distributions $\hat{\delta}^{*(i^*)}$ indicate that this killer whale is in the less active subdive state 1 40\% of the time during a dive of type 1 but only 9\% of the time during a dive of type 2. Less active swimming behaviour is consistent with the need for marine mammals to conserve energy when diving to depth and holding their breath for long periods \citep{Williams:1999,Hastie:2006}. Figure \ref{fig:labeled_dives} shows the decoded dive behaviour of six selected dives. Section 1.3 of the supplementary material also shows the probability of each dive type and subdive state given the data and the fitted model.

\subsection{Model validation}
\label{subsec:model_validation}

We use two visual tools to evaluate the CarHHMM-DFT: pseudoresidual plots and empirical histograms. The pseudoresidual of an observation $y_t$ for a traditional HMM is equal to $\Phi^{-1} \left(Pr(Y_t < y_t|\{Y_1,\ldots,Y_T\}/\{Y_t\}) \right)$, where $\Phi$ is the cumulative distribution function of a standard Normal distribution. If the model is correct, then all pseudoresiduals are independent and follow a standard Normal distribution. Histograms of the pseudoresiduals mostly support that this model is well-specified. One exception is $\Ztwo_{t,\tilde t^*}$, whose pseudoresiduals are noticeably right-skewed (see Figure \ref{fig:pseudoresids}). This implies that the true distribution of $\Ztwo_{t,\tilde t^*}$ may follow a heavier-tailed distribution than the Gamma distribution used in the case study. See Sections 1.4 through 1.6 of the supplementary material for pseudoresidual plots for all observations.

We also plot histograms of dive duration corresponding to each dive type in Figure \ref{fig:empirical_dist}. Each observed dive duration is weighted by the decoded probability that it corresponds to a particular dive type. This results in two histograms -- one corresponding to dive type 1 and another corresponding to dive type 2. Each histogram is then plotted with the corresponding emission distribution learned by the CarHHMM-DFT. Analogous histograms corresponding to the fine-scale observations $\z_{t,\tilde t^*}$ are contained in Sections 1.4 through 1.6 of the supplementary material. Our results mostly show that the CarHHMM-DFT explains the data well, but there are some exceptions. In particular, $\Ztwo_{t,\tilde t^*}$ is right-skewed and $\Zone_{t,\tilde t^*}$ has heavy tails corresponding to subdive state 3, indicating the existence of rare events corresponding to exceptionally sudden changes in acceleration of the killer whale. These outliers are potential subjects for future study and may indicate biologically relevant phenomena such as prey capture \citep{Tennessen:2019a}.

\subsection{Comparison with candidate models}

The HHMM-DFT, which ignores auto-correlation in acceleration, decodes dive types and subdive states similarly to the CarHHMM-DFT, but it is less likely to categorize the behaviour at the beginning and end of dives as subdive state 3 (see Figures 4 and 5 of the supplementary material). In addition, the HHMM-DFT produces estimates of $\sigma_A^{*(\cdot,1)}$, $\sigma_A^{*(\cdot,2)}$, and $\sigma_A^{*(\cdot,3)}$ which are approximately $50$ to $100$ percent larger than those of the CarHHMM-DFT for all axes of acceleration. The estimated uncertainties of the estimates of $\mu_A^{*(\cdot,1)}$, $\mu_A^{*(\cdot,2)}$, and $\mu_A^{*(\cdot,3)}$ are also less than half of those for the CarHHMM-DFT for all axes of acceleration (see Tables 1 and 2 of the supplementary material). Further, the empirical auto-correlation of acceleration within each decoded subdive state was at least 0.5 for all acceleration axes and subdive states, and the pseudoresiduals of the HHMM-DFT do not follow a standard normal distribution due to several outliers (see Figure 15 of the supplementary material). These findings suggest that the HHMM-DFT is a significantly worse fit to this data than the full CarHHMM-DFT.

The CarHHMM does not model the ``wiggliness" of the acceleration data, so it regularly fails to pick up obvious behavioural changes corresponding to the periodicity shown in Figure \ref{fig:labeled_dives}. This essentially disqualifies the CarHHMM as a viable model for these data. The pseudoresiduals corresponding to the CarHHMM also do not appear to fit a Normal distribution.

Finally, the CarHMM-DFT, which lacks a hierarchical structure, produces similar fine-scale parameter estimates and state estimates compared to those of the CarHHMM-DFT. However, its lack of hierarchical structure means that it fails to differentiate between short and long dives. This model therefore does not give any information regarding the dive-level Markov chain or the relationship between the dive and subdive levels. In particular, the CarHMM-DFT does not indicate that the whale is more likely to be in subdive state 1 when engaged in longer dives compared to shorter dives. A more complete set of results for each of the candidate models is in Section 1 of the supplementary material.

