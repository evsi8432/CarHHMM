% !TeX root = ../main.tex

%\section{Killer Whale Case Study}

%In order to test the advantages of the CarHHM-DFT over the other models on real-world data, we used the four models described in the simulation study to analyze dive data from a Northern Resident Killer Whale (NRKW) off the coast of British Columbia, Canada. 

%\section{Results}

We report the results from fitting the CarHHMM-DFT in detail, validate the results, and then compare the results of this model with those from the other candidate models.

The estimates for the emission distribution parameters suggest that the killer whale has at least two distinct dive behaviors (Table \ref{table:emis_dists_CarHHMM-DFT}, Figure \ref{fig:coarse_emis}). 
Dive type 1 corresponds to shorter, shallower dives. Ecologists have attributed a variety of purposes to such short dives, including resting (CITATION, you can ask Ian and Sarah for help here - look at the Tennessen paper with HMM).
Dive type 2 is longer and deeper. These types of deep sustained dives have been associated with foraging behaviors in killer whales (Citation Tennessen maybe HMM and the other Tennesen with foraging data).

The three subdive behaviors also appeared distinct with respect to their Fourier sums, as the mean of $Z^{*(2)}$ is separated by an order of magnitude between the three subdive types (Table \ref{table:emis_dists_CarHHMM-DFT}, Figure \ref{fig:fine_emis}). 
For subdive type 1, the mean of $Z^{*(2)}$ and the variance of $\mathbf{Z}^{*(1)}$ in this state is smaller than the other subdive types. The auto-correlation of $\mathbf{Z}^{*(1)}$ also is higher for subdive type 1 than every other subdive type. This corresponds to less overall activity compared to the other subdive types. 
Subdive type 2 is characterized by a mean Fourier sum ($Z^{*(2)}$) one order of magnitude higher than subdive type 1. Acceleration has about twice the variance in this subdive type compared to subdive type 1, and slightly lower auto-correlation. Subdive type 2 corresponds to fluking (active swimming), as strong sinusoidal behavior in acceleration is characteristic of this behavior in marine mammals \citep{Simon:2012}.
In subdive type 3, the mean of $Z^{*(2)}$ and variance of $\mathbf{Z}^{*(1)}$ is much higher compared to every other state. The auto-correlation of $\mathbf{Z}^{*(1)}$ is also much lower in this subdive type, implying more variation in acceleration. This corresponds to vigorous swimming activity, especially as the killer whale begins a dive or surfaces from a dive. 

The estimated probability transition matrices and associated stationary distributions are
%
$$\hat \Gamma = \begin{pmatrix} 
0.788 & 0.212 \\
0.809 & 0.191
\end{pmatrix},$$
$$\hat \delta = \begin{pmatrix} 0.792 & 0.208 \end{pmatrix},$$
%
$$\hat \Gamma^{*(1)} = \begin{pmatrix} 
0.679 & 0.321 & 0.000 \\
0.038 & 0.904 & 0.058 \\
0.000 & 0.232 & 0.768
\end{pmatrix}, \qquad 
\hat \Gamma^{*(2)} = \begin{pmatrix} 
0.859 & 0.141 & 0.000 \\
0.114 & 0.841 & 0.045 \\
0.000 & 0.216 & 0.784
\end{pmatrix},$$
$$\hat \delta^{*(1)} = \begin{pmatrix} 0.087 & 0.731 & 0.182 \end{pmatrix}, \enspace \text{and} \enspace \hat \delta^{*(2)} = \begin{pmatrix} 0.401 & 0.496 & 0.103 \end{pmatrix}.$$
%
The coarse-scale stationary distribution indicates that most dives are short type 1 dives. The coarse-scale transition probability matrix further indicates that the whale usually performs multiple short type 1 dives before doing only one (or a few) long type 2 dive. The fine-scale stationary distributions indicate that this killer whale is more likely to be in the less active subdive type 1 when performing long deep dives than when performing short shallow dives. Using less active swimming behavior when diving, and holding breath, for long periods of time has been suggested to be an energy reduction strategy in bottle-nose dolphins \citep{Williams:1999}. Figure \ref{fig:labeled_dives} shows the decoded dive behavior of 6 selected dives. The supplementary material also shows the estimated probability of each dive and subdive type (section S-2.2).

\subsection{Model Validation}
\label{subsec:model_validation}

Two visual tools were used to evaluate this model: pseudo-residuals and empirical histograms. A pseudo-residual of a particular observation is the marginal CDF of an observation conditioned on all other observations under the learned model \citep{Zucchini:2016}. To easily visualize outliers, this pseudo-residual is often passed through the quantile function of the standard Normal distribution. Mathematically, the pseudo-residual of an observation $y_t$ in a traditional HMM is equal to $\Phi^{-1} \left(Pr(Y_t < y_t|\{Y_1,\ldots,Y_T\}/\{Y_t\}) \right)$, where $\Phi$ is the cumulative distribution function of a standard Normal distribution. If the model is correct, then all pseudo-residuals are independent and follow a standard Normal distribution. We find that histograms of the pseudoresiduals of this model mostly support that the model is well-specified. $Z^{*(2)}$ is an exception, as its pseudo-residuals are noticeably right-skewed (Figure \ref{fig:pseudoresids}). This implies that the true distribution of $Z^{*(2)}$ may follow a heavier-tailed distribution than the gamma distribution. 

We also plotted histograms of each feature where observations were weighted by the probability of a particular hidden state. This empirical distribution was then plotted over its corresponding fitted emission distribution (Figure \ref{fig:empirical_dist}). Our results mostly support a well-specified model with the exception of $Z^{*(2)}$, which is again right-skewed. In addition, $\mathbf{Z}^{*(1)}$ has heavy tails for subdive state 3, indicating the existence of rare events corresponding to exceptionally violent thrashing of the killer whale. These outliers are potential subjects for future study. See the supplementary material for empirical distributions of every feature and every hidden state. 

\subsection{Comparison with candidate models}

The CarHMM-DFT produced very similar to those of the CarHHMM-DFT on the fine scale. However, it failed to differentiate between short and long dives, and it therefore did not infer any information relating to the dive level Markov chain or the relationship between the dive level and the subdive level. For example, the CarHMM-DFT missed the insight that the whale was much more likely to be in subdive type 1 when engaged in longer dives than when engaged in shorter dives.

The HHMM-DFT decoded the dive and subdive types similarly to the CarHHMM-DFT, but was less likely to categorize the beginning and end of dives as subdive type 3. In addition, the HHMM-DFT produced estimates $\hat \sigma_1^*$ that were approximately 50-100 \% larger than those for the CarHHMM-DFT, and the estimated uncertainty of $\hat \mu^*_1$ was less than half of that for the CarHHMM-DFT for all dimensions and subdive states. These findings are consistent with the trend to overestimate variance and produce overconfident parameter estimates when auto-correlation is ignored. Further, the empirical auto-correlation of $\mathbf{Z}^{*(1)}$ within each decoded subdive type was at least 0.5 for all dimensions and subdive types. Finally, the pseudoresiduals of the HHMM-DFT had many more outliers than the pseudoresiduals of the CarHHMM-DFT, providing evidence that the HHMM-DFT was a worse fit to this data than the CarHHMM-DFT.

The CarHHMM did not have access to the Fourier sums of the acceleration data, so it failed to decode the subdive types into biologically interpretable results. In addition, the pseudoresiduals corresponding to the CarHHMM appears to be less heavy-tailed than a normal distribution. 

See section S-??? for a more complete set of results for each of the candidate models.

