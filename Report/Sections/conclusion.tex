% !TeX root = ../main.tex

\section{Discussion}

THE FOLLOWING IS FROM THE STAT 548 PAPER:

While incorporating autocorrelation within a hidden Markov model to analyze animal movement is not new, Lawler et al. introduce a new formulation of this model in the CarHMM. They also review several useful preprocessing tools such as the lag plot and a method for interpolation that includes dividing the data into separate groups if observations are far apart. 

This paper summarizes the CarHMM and provides intuition behind the interpolation scheme laid out by Lawler et al. In addition, it shows that if the emission distributions are normal for the sequence of step-sizes $\bfD$, then the CarHMM models $\bfD$ as a one dimensional Ornstein-Uhlenbeck process.

Finally, the CarHMM is adapted to dive data and used to model the behavior of a killer whale off the coast of British Columbia, Canada. In particular, that whale exhibited auto-correlated velocities in the $z$-direction as well as sinusoidal behavior in dynamic body acceleration. The adjusted CarHMM is able to capture within-dive behaviors of active swimming and passive gliding, but more analysis is necessary to determine the exact number of within-dive behaviors and generalize the method to a larger number of dive types.