%\section{Models and parameter estimation}

Hidden Markov models are useful tools to model state-switching Markovian processes in discrete time and will be used as the core structure of our models. Traditional HMMs model process auto-correlation by assuming that the underlying behavior of a processes follows a single Markov chain. However, when conditioned on the hidden state sequence observations are assumed to be independent. Therefore, classical HMMs do not hold when the observations exhibit other forms of significant correlation in time. Examples of this include continuous-time processes such as a Wiener process, systems which exhibit temporal dependencies acting at several different time scales \citep{Barajas:2017}, and periodic behaviour including the motion of an ideal spring.  
%\textit{blah blah blah -quick restatements (5-15 words to potentially many sentences) of the main issues you are going to fix made a quick try at it, work on smoothing things. You can use some of the text I removed from below}.
To account for each of these common additional dependence structures, we will explore three variations on the traditional HMM. First, we explain the conditionally auto-regressive hidden Markov model (CarHMM), which explicitly models auto-correlation between observations. Next, we review the hierarchical hidden Markov model (HHMM), which models simultaneous temporal dependencies at different time scales as two distinct hidden Markov models in a hierarchical model structure. Finally, we describe an HMM over a discrete Fourier transform (HMM-DFT), which applies the DFT to a moving window over the raw data to summarize its periodic behavior as a function of time. We will show how each of these variations can be altered and combined to form a wide variety of new models.

%%%%%%%%%%%%%%%%%%%%%%%%%%%%%%%%%%%%%%%%%
\subsection{The base structure of HMMs}

An HMM is comprised of a sequence of unobserved states $X_t$, $t = 1, \ldots, T$, and an associated sequence of  possibly high-dimensional observations $Y_t$, $t = 1, \ldots, T$.
The $Y_t$'s are often referred to as ``emissions'' and the index $t$ typically refers to time. 
The $X_t$'s form a Markov chain and take possible values $1, \ldots, N$. Their distribution is governed by the distribution of the initial state $X_1$ and the $N \times N$ transition probability matrix $\Gamma$, where $\Gamma_{ij} = \Pr(X_{t+1} = j | X_t = i)$, for $t=1,\ldots, T-1$, and $i, j = 1,\ldots, N$. 
%
We assume that $X_1$ follows the chain's stationary distribution, which is denoted by $\delta \in \bbR^N$, with $i^{th}$ component
$\delta_i = \Pr\{X_1 = i\},~ i = 1,\ldots,N.$
A Markov chain's stationary distribution is determined by its probability transition matrix via $\delta = \delta \Gamma$ and $\sum_{i=1}^N \delta_i = 1$.
%
The distribution of an emission $Y_t$ depends only on the corresponding state $X_t$ and no other observations or hidden states: $p\left(y_t|\{X_1,\ldots, X_T\},\{Y_1,\ldots, Y_T\}/ \{Y_t\}\right) = p(y_t|X_t)$.
%
These conditional distributions are governed by state-dependent parameters. If $X_t = i$, then the state-dependent parameter is $\theta^{(i)}$ and we denote the conditional distribution of $Y_t$ given $X_t=i$ by its conditional density or probability mass function, denoted $f^{(i)}(\cdot ; \theta^{(i)})$, or sometimes simply $f^{(i)}(\cdot)$.
%
Figure \ref{fig:models} represents the dependence structure of an HMM.

Using observed emissions, here denoted $y = (y_1,\ldots,y_T)$, we can find the maximum likelihood estimates of the parameters $\Gamma$ and $\Theta \equiv (\theta^{(1)},\ldots,\theta^{(N)})$. We write the likelihood $\calL_{\text{HMM}}$ using the  well-known \textit{forward algorithm} \citep{Zucchini:2016}:
%
$$\calL_{\text{HMM}}(y;\Theta,\Gamma) = \delta P(y_1;\Theta) \prod_{t=2}^T \Gamma P(y_t;\Theta) \mathbf{1}_N$$
%
where $\mathbf{1}_N$ is an $N$-dimensional column vector of ones and
%
$P(y_t;\Theta)$ is an $N \times N$ diagonal matrix with $(i,i)^{th}$ entry  $f^{(i)}(y_t; \theta^{(i)})$.
%

Following \citet{Barajas:2017}, we parameterize the $N \times N$ transition probability matrix $\Gamma$ such that the entries of the matrix are forced to be non-negative and the rows to sum to 1:
%
\[
\Gamma_{ij} = \frac{\exp(\eta_{ij})}{\sum_{k=1}^N \exp(\eta_{ik})}, 
\]
%
where $\eta \in \bbR^{N \times N}$ and $\eta_{ii}$ is set to zero for identifiability.  Then $\calL_{\text{HMM}}(y;\Theta,\Gamma)$ can be maximized using a wide range of optimizers.  For simplicity, we will continue to use $\Gamma$ in our notation, suppressing the reparameterization in terms of  $\eta$.


%%%%%%%%%%%%%%%%%%%%%%%%%%%%%%%%%%%%%%%%%
\subsection{Relaxing the conditional independence assumption with a CarHMM}

The CarHMM, introduced by \citet{Lawler:2019}, explicitly models auto-correlation in the observations of an HMM. Like a traditional HMM, a CarHMM is made up of a Markov chain of unobserved states $X_1,\ldots, X_T$ that can take on values $1, \ldots, N$, with transition probability matrix $\Gamma$ and initial distribution $\delta$ equal to the stationary distribution of $\Gamma$. Unlike a traditional HMM, the CarHMM assumes that the distribution of $Y_t$ conditioned on $X_1,\ldots, X_T$ and $Y_1,\ldots, Y_{t-1}$, depends on \textit{both} $X_t$ \textit{and} $Y_{t-1}$ rather than only $X_t$. 
The first emission $Y_1$ is assumed to be fixed as an initial value which does not depend upon $X_1$. Figure \ref{fig:models} shows the dependence structure of a CarHMM.

%
We denote the conditional distribution of $Y_t$ given $Y_{t-1}= y_{t-1}$ and $ X_t=i$ as $f^{(i)}( \cdot | y_{t-1}; \theta^{(i)})$ or simply $f^{(i)}( \cdot | y_{t-1})$.
For example, one could assume that this conditional distribution is Normal with parameters $\theta^{(i)} = \{\mu^{(i)},\sigma^{(i)},\phi^{(i)}\}$ where:
%
\[
\mathbb{E}(Y_t|Y_{t-1} = y_{t-1},X_t=i) = \phi^{(i)} ~ y_{t-1} ~+ ~(1-\phi^{(i)})  ~\mu^{(i)}
\]
and
\[
\mathbb{V}(Y_t| Y_{t-1} =y_{t-1}, X_t = i) = (\sigma^{(i)})^2.
\]
%
The likelihood for the CarHMM can be easily calculated using the forward algorithm. As previously, let $y$ be the vector of observed emissions. Then
\begin{equation}
    \calL_{\text{CarHMM}}(y;\Theta,\Gamma) = \delta \prod_{t=2}^T \Gamma P(y_t|y_{t-1};\Theta) \mathbf{1}_N
    \label{CarHMM_likelihood}
\end{equation}
where
%
$P(y_t|y_{t-1};\Theta)$ is an $N \times N$ diagonal matrix with $(i,i)^{th}$ entry equal to $f^{(i)}(y_t|y_{t-1}; \theta^{(i)})$.

\subsection{Incorporating correlation on multiple time scales with an HHMM}

An HHMM accounts for different levels of correlation by modelling a coarse-scale process and a fine-scale process, each of which with an HMM \citep{Barajas:2017,Adam:2019}. The coarse-scale process is an HMM as defined previously, where $X_1, \ldots, X_T$ make up an unobserved Markov chain with $N$ possible states and $Y_1,\ldots, Y_T$ are the corresponding observed responses.   
%
In the hierarchical setting, each state $X_t$ emits yet another sequence of fine-scale unobserved states, $X_t^* \equiv (X_{t,1}^*,\ldots, X_{t,T_t^*})$ and a sequence of fine-scale observed emissions, $Y_t^* \equiv (Y_{t,1}^*,\ldots, Y_{t,T_t^*})$. The fine-scale process ($X_t^*, Y_t^*$) makes up another HMM with parameters that depend on the value of $X_t$. Specifically, if $X_t=i$, then the distribution of $X_t^*$ is characterized by an $N^*_t \times N^*_t$ transition probability matrix $\Gamma^{*(i)}$ and initial distribution $\delta^{*(i)}$, which we assume is equal to the stationary distribution of the chain. For simplicity, we take $N_t^* \equiv N^*$ although this is not necessary.

The distribution of $Y_{t, t^*}$ given $X_{t, t^*}=i^*$ and $X_t=i$ is governed by a parameter $\theta^{(i,i^*)}$ and has density or probability mass function denoted $f^{*(i,i^*)}\left(\cdot; \theta^{(i,i^*)}\right)$ or simply $f^{*(i,i^*)}(\cdot)$. We denote the fine-scale emission parameter vector corresponding to $X_t=i$ as $\Theta^{*(i)}=\left(\theta^{(i,1)}, \ldots, \theta^{(i,N^*)}\right)$.

Given the coarse-scale states, $X_1,\ldots, X_T$, the $T$ fine-scale processes $(X_1^*, Y_1^*), \ldots, (X_T^*, Y_T^*)$, are independent HMMs. Depending upon the process being modeled, it is possible to force certain parameters to be shared across different coarse or fine states. For example, in the killer whale case study (Section \ref{sec:case_study}), we force the fine-scale emission parameters to be shared across coarse-scale hidden states (i.e. $\theta^{(1,i^*)} = \ldots = \theta^{(N,i^*)}$ for $i^* = 1, \ldots, N^*$). Figure \ref{fig:models} represents the dependence structure for an HHMM.

Due to the nested structure of the hierarchical hidden Markov model, the likelihood is easy to calculate via the forward algorithm.
%
Let $y$ be the $T$-vector of the observed coarse-scale emissions and
$y^*$ be the $(T_1^* + \cdots + T_T^*)$-vector of the observed fine-scale emissions.
%
In addition, let $\Theta^* \equiv \{\Theta^{*(1)}, \ldots, \Theta^{*(N)}\}$ denote the collection of all fine-scale emission parameters and $\Gamma^* \equiv \{\Gamma^{*(1)}, \ldots, \Gamma^{*(N)}\}$ denote the collection of all fine-scale transition probability matrices. The likelihood of the observed data is then
%
\[
\calL_{\text{HHMM}}(y,y^*;\Theta,\Theta^*,\Gamma,\Gamma^*) = \delta P(y_1,y_1^*;\Theta,\Theta^*,\Gamma^*) \prod_{t=2}^T \Gamma P(y_t,y_t^*;\Theta,\Theta^*,\Gamma^*) \mathbf{1}_N
\]
%
where $P(y_t,y_t^*;\Theta,\Theta^*,\Gamma^*)$ is an $N \times N$ diagonal matrix with $(i,i)^{th}$ entry corresponding to $X_t=i$ and equal to 
$f^{(i)}(y_t)\calL_{\text{HMM}}\left(y_t^*;\Theta^{*(i)},
\Gamma^{*(i)}\right)$. 

For more information on specific considerations for HHMMs, such as incorporating covariates into the probability transition matrix, state decoding, model selection and model checking, see \citet{Adam:2019}.

%%%%%%%%%%%%%%%%%%%%%%%%%%%%%%%%%%%%%%%%%
\subsection{Accounting for periodicity with the HMM-DFT}
\label{subsec:STFT}

The HMM-DFT incorporates hierarchical structure into an HMM differently than an HHMM (Fig \ref{fig:models}). In particular, the fine-scale process is no longer modeled with an HMM and instead summarized using its Fourier transform. For simplicity, we assume that the length of the fine-scale processes is constant (i.e. that $T^*_t = T^*$), although this need not be the case. Suppose that the fine-scale process $y^*_t$ does not switch hidden states, but does exhibit significant periodic behaviour. We then suggest using the discrete Fourier transform (DFT) on $y^*_t$:
%
\begin{align*}
    DFT\{y^*_t\}(k) := \hat{y}^{*(k)}_{t} = \sum_{t^* = 1}^{T^*} y^*_{t,t^*}\exp\left(-i \frac{2\pi k}{T^*} (t^*-1)\right), \quad k = 0, 1, \ldots, T^*-1.
\end{align*}
%
Summary statistics can then drastically reduce the dimension of $\hat{y}^*_t$. One example is as follows:
%
\begin{equation}
    \label{eqn:z}
    z_t^{*(1)} := \mathcal{R}\left(\hat{y}^{(0)}_t\right) \qquad z_t^{*(2)} := \frac{1}{T^*}\sum_{k=1}^{\tilde{\omega}}|\hat{y}^{(k)}_t|^2
\end{equation}
%
In words, $z_t^{*(1)}$ is the average value of $y^*_t$ and $z_t^{*(2)}$ is the squared 2-norm of the component of $y^*_t$ that can be attributed to frequencies between $1$ and $\tilde{\omega}$ periods per window length $T^*$. The maximum frequency $\tilde{\omega}$ is a problem-specific tuning parameter which should be selected with care. These summary statistics are just one possible choice to describe each window; other choices include the dominant frequency and amplitude of $y^*_t$. Figure \ref{fig:models} represents the structure of the HMM-DFT.

Once $z^*_t$ is calculated, it can be treated as an observation of the HMM and incorporated into the emission distribution $f^{(i)}\left(y_t,z^*_t;\theta^{(i)}\right)$, or more succinctly $f^{(i)}\left(y_t,z^*_t\right)$. The likelihood of the HMM-DFT is as follows:
\begin{equation}
    \calL_{\text{HMM-DFT}}(y,z^*;\Theta,\Gamma) = \delta P(y_1,z^*_1;\Theta) \prod_{t=2}^T \Gamma P(y_t,z^*_t;\Theta) \mathbf{1}_N
    \label{HMMDFT_likelihood}
\end{equation}
where $P(y_t,z^*_t;\Theta)$ is an $N \times N$ diagonal matrix with $(i,i)^{th}$ entry equal to $f^{(i)}\left(y_t,z^*_t;\theta^{(i)}\right)$.

It is possible to accommodate unequal time steps within $y_t^*$ by using the non-uniform discrete Fourier transform (NDFT). We do not describe the method in detail here, but the generalization is straightforward. See \citet{Bagchi:1999} for details.

\subsection{General structure for building complex models}

In addition to the models described above, the fine-scale process $Y^*_t$ can be modeled using \textit{any} parametric model which admits an easy-to-compute likelihood. The fine-scale likelihood $\calL_{\text{HMM}}$ from the HHMM likelihood can therefore be replaced by the likelihood of a general fine-scale model, $\calL_{\text{fine}}(y^*_t;\Theta^{*(i)})$:
\[
\calL_{\text{coarse}}(y,y^*;\Theta,\Theta^*,\Gamma) = \delta P(y_1,y^*_1;\Theta,\Theta^*) \prod_{t=2}^T \Gamma P(y_t,y^*_t;\Theta,\Theta^*) \mathbf{1}_N
\]
where $P(y_t,y^*_t;\Theta,\Theta^*) $ is an $N \times N$ diagonal matrix with $(i,i)^{th}$ entry corresponding to $X_t=i$ and equal to $f^{(i)}\left(y_t;\Theta^{(i)}\right)\calL_{\text{fine}}\left(y^*_t;\Theta^{*(i)}\right)$. This definition is straightforward to extend to the CarHMM as well:
\[
\calL_{\text{coarse}}(y,y^*;\Theta,\Theta^*,\Gamma) = \delta \prod_{t=2}^T \Gamma P(y_t,y^*_t|y_{t-1};\Theta,\Theta^*) \mathbf{1}_N
\]
where $P(y_t,y^*_t|y_{t-1};\Theta,\Theta^*) $ is an $N \times N$ diagonal matrix with $(i,i)^{th}$ entry corresponding to $X_t=i$ and equal to $f^{(i)}\left(y_t|y_{t-1};\Theta^{(i)}\right)\calL_{\text{fine}}\left(y^*_t;\Theta^{*(i)}\right)$.

Possible candidates for the fine-scale model include, but are not limited to, any of the models described in the previous subsections (HMM, CarHMM, HHMM, and HMM-DFT). These alternative models can act as initial building blocks in a practitioner's toolbox to construct increasingly complex hierarchical models based on HMMs. In the sections that follow, we perform both a simulation study and real-world case study modelling killer whale dive behaviour using models constructed from these building blocks.