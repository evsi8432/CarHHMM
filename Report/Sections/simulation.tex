% !TeX root = ../main.tex

%\section{Simulation Study}

To evaluate the performance of each candidate model when the ground-truth is known, we perform a simulation study based on data generated from the full CarHHMM-DFT. The parameters used to generate the data are loosely based on those estimated in the case study (see Table \ref{table:emis_dists_CarHHMM-DFT}), with slight modifications made for simplicity. In particular, we set $\Gamma$ such that the simulated data has equal numbers of short and long dives in expectation, the number subdive states is set to $N^*=2$, and $\Ztwo_{t,t^*}$ is set to a scalar. Metrics used to evaluate each model include decoding accuracy of hidden states, bias in parameter estimates, standard error of parameter estimates, and fitting times. To assess the accuracy of uncertainty estimates, we also compare the empirical standard error of each model's parameter estimates with the estimated standard error using the inverse of the observed Fisher information under that model.

\subsection{Data simulation}
\label{subsec:data_simulation}

We generate 500 independent data sets from the CarHHMM-DFT. Each data set consists of a sequence of 100 curves which each represent a simulated killer whale dives. Each dive can be one of $N=2$ dive types based on a Markov chain with probability transition matrix
%
$$\Gamma = \begin{pmatrix} 0.5 & 0.5 \\ 0.5 & 0.5 \end{pmatrix}.$$
%
Dive duration is gamma distributed, and the coarse-scale emission parameters $\theta$ are
$$
    \mu^{(1)} = 20s, \quad \sigma^{(1)} = 5s, \quad
    \mu^{(2)} = 80s, \enspace \text{and} \enspace \sigma^{(2)} = 25s.
$$
%
After calculating the dive durations for all 100 dives in each data set, dive $t$ is broken into a sequence of $T^*_t = \lfloor Y_t/2 \rfloor$ two-second windows. Each two-second segment is assigned one of $N^*=2$ behaviours according to a fine-scale Markov chain $X^*_t \equiv \left\{X^*_{t,1}, \ldots \right\}$ with probability transition matrices
%
$$\Gamma^{*(1)} = \begin{pmatrix} 0.5 & 0.5 \\ 0.1 & 0.9 \end{pmatrix} \quad \text{ and } \quad \Gamma^{*(2)} = \begin{pmatrix} 0.8 & 0.2 \\ 0.3 & 0.7 \end{pmatrix}.$$ 
%
For each dive and subdive segment, we simulate the fine-scale transformed observations $Z^*_t \equiv \left\{Z^*_{t,1}, \ldots, \right\}$, where $Z^*_{t,t^*} = \{\Zone_{t,t^*},\Ztwo_{t,t^*}\}$, using the following parameters:
%
\begin{gather*}
    \mu_A^{*(\cdot,1)} = 0.0 s, \enspace \sigma_A^{*(\cdot,1)} = 0.05s, \enspace \phi_A^{*(\cdot,1)} = 0.99, \\
    %
    \mu_A^{*(\cdot,2)} = 0.0 s, \enspace \sigma_A^{*(\cdot,2)} = 0.1s, \enspace \phi_A^{*(\cdot,2)} = 0.95, \\
    %
    \mu_W^{*(\cdot,1)} = 10.10, \quad \sigma_W^{*(\cdot,1)} = 3.18, \\
    %
    \mu_W^{*(\cdot,2)} = 305.94, \quad \sigma_W^{*(\cdot,2)} = 17.49.
\end{gather*}
%
Recall that $\Zone_{t,t^*}$ follows a CarHMM and is Normally distributed while $\Ztwo_{t,t^*}$ follows an HMM and is Gamma distributed. It is not possible to uniquely reconstruct the raw accelerometer data $Y^*$ from $Z^*$ alone, but we describe one possible mapping in the appendix. Figure \ref{fig:sim_data} shows this reconstructed sequence along with one realization of $Z^*$ for five dives of one simulated data set. 

The two simulated dive types differ in that dives of type 1 are much shorter on average (20 seconds) than dives of type 2 (80 seconds). The two subdive types differ primarily due to $\mu_W^*$ and $\sigma_W^*$, both of which are much higher in subdive state 2 than subdive state 1. This corresponds to much more vigorous and variable periodic behaviour in the acceleration data.

All four candidate models (CarHHMM-DFT, HHMM-DFT, CarHHMM, and CarHMM-DFT) are fit to all 500 data sets using the Cedar Compute Canada cluster with 1 CPU and 4 GB of dedicated memory per model.

\subsection{Simulation results}

The full CarHHMM-DFT is the best-performing model of the four candidates, which is to be expected as the generating model. If $X^*_{t,t^*}$ is the unknown subdive state and $x^*_{t,t^*}$ is the known true subdive state for dive $t$ and window $t^*$, then using the forward-backward algorithm we find that $\Pr(X^*_{t,t^*} = x^*_{t,t^*}|Y,Y^*)$ is $(1 - 10^{-12})$ on average across all simulations, dives, and windows. We refer to this quantity as the average accuracy of the model. For dive type, this average accuracy was above 0.9 (see Table \ref{table:accuracy}). All parameter estimates of mean values and fine-scale probability transition matrices, $\hat \mu$ and $\hat \Gamma^*$, respectively, have statistically insignificant biases. In addition, biases for $\hat \sigma$, $\hat \phi$, and $\hat \Gamma$ are no more severe for the CarHHMM-DFT than the other three models. The empirical standard error of all parameter estimates ($\hat \theta$, $\hat \Gamma$, $\hat \theta^*$, $\hat \Gamma^*$) are well-approximated by the inverse of the observed Fisher information matrix, although the estimated standard errors tend to be slightly smaller than the empirical standard error. See Tables 6-9 of the supplementary material for more information.

The HHMM-DFT performs almost identically to the CarHHMM-DFT in most respects, including its average accuracy and parameter bias. However, the HHMM-DFT differs from the CarHHMM-DFT in its estimates of $\sigma_A^{*(\cdot,1)}$ and $\sigma_A^{*(\cdot,2)}$, both of which are greatly overestimated. In addition, the estimated standard errors of $\hat \mu_A^{*(\cdot,1)}$, $\hat \mu_A^{*(\cdot,2)}$, $\hat \sigma_A^{*(\cdot,1)}$, and $\hat \sigma_A^{*(\cdot,2)}$ are much smaller than the associated empirical standard errors (see Table 7 of the supplementary material). These results imply that estimated standard errors of parameter estimates are too small when using the HHMM-DFT since auto-correlation is ignored. This finding is consistent with the results from the case study, where the HHMM-DFT produced higher estimates of $\sigma_A^*$ and smaller estimates of standard error compared to the CarHHMM-DFT.

The CarHHMM is the worst-performing model in terms of decoding accuracy, with an average accuracy below $0.9$ for both dive and subdive types. This result is consistent with expectations because the CarHHMM does not model the ``wiggliness" of the fine-scale process, which is the most distinct observation when determining subdive type. The CarHHMM is also much more likely to misclassify subdive type 1 than subdive type 2 (see Table \ref{table:accuracy}). In addition to its poor average accuracy, the CarHHMM is also the worst of the four candidate models at estimating parameters, with the fine-scale probability transition matrices associated with dive type 2 ($\hat \theta^{*(2)}$ and $\hat \Gamma^{*(2)}$) being especially poor estimates (see Table 9 and Figure 33 of the supplementary material).

Finally, the CarHMM-DFT is comparable to the CarHHMM-DFT in terms of decoding accuracy, parameter bias, and parameter standard error for the fine-scale observations. In addition, the time required to fit the CarHMM-DFT is less than half of that of the other models (see Table \ref{table:accuracy}). However, this model is unable to estimate dive types as it lacks any hierarchical structure. The CarHMM-DFT nonetheless fits a gamma distribution to the dive duration of all dives. The resulting parameter estimates ($\hat \mu$ and $\hat \sigma$) are highly correlated, and $\hat \sigma$ underestimates the true standard deviation of all dive types by approximately five seconds. See Figures \ref{fig:acc_coarse} and \ref{fig:acc_fine} for a visualization of the simulated data set as well as the decoded dive types and subdive states for each model.