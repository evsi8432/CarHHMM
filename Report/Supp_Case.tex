% Initial and revised submissions should be 12 point; this will be removed in the final version.
%\documentclass[12pt]{TD-CJS}
\documentclass{article}

% Initial and revised submissions should also be double spaced.  This command will be removed in the final version.
%\renewcommand{\baselinestretch}{2}

\usepackage{latexsym}
\usepackage{amsmath}
\usepackage{amsfonts}
\usepackage{amssymb}
\usepackage{psfrag}
\usepackage{graphicx}
\usepackage[dvipsnames]{xcolor}
\usepackage{url}
\usepackage{float}
\usepackage[margin=1in]{geometry}
% header.tex
% this is where you load pacakges, specify custom formats, etc.

\usepackage[left=0.5in,right=0.5in,top=1in,footskip=25pt]{geometry} 
% \usepackage{changepage}
\usepackage{amsmath,amsthm,amssymb,amsfonts}
\usepackage{mathtools}
\usepackage{bbm}
% enumitem for custom lists
\usepackage{enumitem}
% Load dsfont this to get proper indicator function (bold 1) with \mathds{1}:
\usepackage{dsfont}
\usepackage{centernot}

\usepackage[usenames,dvipsnames]{xcolor}

% set up commenting code (I will use this during marking)
\definecolor{CommentColor}{rgb}{0,.50,.50}
\newcounter{margincounter}
\newcommand{\displaycounter}{{\arabic{margincounter}}}
\newcommand{\incdisplaycounter}{{\stepcounter{margincounter}\arabic{margincounter}}}
\newcommand{\COMMENT}[1]{\textcolor{CommentColor}{$\,^{(\incdisplaycounter)}$}\marginpar{\scriptsize\textcolor{CommentColor}{ {\tiny $(\displaycounter)$} #1}}}

\usepackage{appendix}

% set up graphics
\usepackage{graphicx}
\DeclareGraphicsExtensions{.pdf,.png,.jpg}
\graphicspath{ {fig/} }
\input{defs}

\begin{document}
%\firstpage{1}
%\lastpage{25}
%\jvol{xx}
%\issue{yy}
%\jyear{2020}
%\jid{CJS}
%\aid{???}
% The running head contains the author names
%\rhauthor{BlindedA and BlindedB}
%\copyrightline{Statistical Society of Canada}
%\Frenchcopyrightline{Soci\'et\'e statistique du Canada}
% History: received and accepted dates
%\received{\rec{9}{July}{2009}}
%\accepted{\acc{8}{July}{2010}}

% User-defined commands go here
%\renewcommand{\eqref}[1]{(\ref{#1})}
%\newcommand{\mb}[1]{\mathbf{#1}}
%\newcommand{\mbb}[1]{\mathbb{#1}}
%\newcommand{\mt}[1]{\mathrm{#1}}
%\newcommand{\rv}{random variable}
%\newcommand{\newblock}{}
\bibliographystyle{abbrvnat}

% Title, authors, affiliations
\title{Modelling multi-scale, state-switching functional data with hidden Markov models: Supplement A (case study)}
\date{}
\author{Evan Sidrow, Nancy Heckman, Sarah M. E. Fortune, \\ Andrew W. Trites, Ian Murphy, and Marie Auger-M\'eth\'e}

% Abstract, keywords, and classification codes

\maketitle

\newcounter{tablenum}
\addtocounter{tablenum}{1}
\newcounter{fignum}
\addtocounter{fignum}{1}

    \section{Lag Plots}
        
        \begin{center}
        \includegraphics[height=5in]{../Plots/2019/20190902-182840-CATs_OB_1_0_267_CarHHMM2_lagplot.png}
        \end{center}
        
        \noindent Figure S\arabic{fignum}: Collection of five lag plots, where a given observation is on the $x$-axis and the subsequent observation is on the $y$-axis. Each plot corresponds to a feature in the killer whale dive data. Lag plots are used to decide whether to include autocorrelation in the HMMs for modelling the killer whale's kinematic data. These plots can also be used to determine the number of hidden states if multiple distinct patterns are visible. 
        \addtocounter{fignum}{1}
        
        \newpage
        
    \section{Likelihood Surface of the CarHHMM-DFT}
    
        \begin{center}
        \includegraphics[width=2.25in]{../Plots/2019/20190902-182840-CATs_OB_1_0_267_CarHHMM2_coarse-theta-likelihood-dive_duration-1.png}
        \includegraphics[width=2.25in]{../Plots/2019/20190902-182840-CATs_OB_1_0_267_CarHHMM2_coarse-theta-likelihood-dive_duration-2.png}
        \end{center}
        
        \noindent Figure S\arabic{fignum}: Log-likelihood of the CarHHMM-DFT as a function of the dive duration parameters $\mu_Y^{(i)}$ and $\sigma_Y^{(i)}$ for dive types $i = 1,2$. All other parameters and transition probability matrices are set to be equal to the maximum likelihood estimates shown in Table 1 of the main text. The log-likelihood is added to the negative log-likelihood of the maximum likelihood estimate $(\hat \mu_Y,\hat \sigma_Y)$, which is denoted by red stars.
        \addtocounter{fignum}{1}
        
        
        \begin{center}
        \includegraphics[width=2.1in]{../Plots/2019/20190902-182840-CATs_OB_1_0_267_CarHHMM2_coarse-gamma-likelihood.png}
        \end{center}
        
        \noindent Figure S\arabic{fignum}: Log-likelihood of the CarHHMM-DFT as a function of the coarse-scale transition probability matrix parameters $\Gamma_{1,2}$ and $\Gamma_{2,1}$. All other parameters and transition probability matrices are set to be equal to the maximum likelihood estimates shown in Table 1 of the main text. The log-likelihood is added to the negative log-likelihood of the maximum likelihood estimate $(\hat \Gamma_{1,2},\hat \Gamma_{2,1})$, which is denoted by a red star.
        \addtocounter{fignum}{1}
        
        % Ahat_low
        \begin{center}
        \includegraphics[width=2.1in]{../Plots/2019/20190902-182840-CATs_OB_1_0_267_CarHHMM2_fine-theta-likelihood-Ahat_low-0.png}
        \includegraphics[width=2.1in]{../Plots/2019/20190902-182840-CATs_OB_1_0_267_CarHHMM2_fine-theta-likelihood-Ahat_low-1.png}
        \includegraphics[width=2.1in]{../Plots/2019/20190902-182840-CATs_OB_1_0_267_CarHHMM2_fine-theta-likelihood-Ahat_low-2.png}
        \end{center}
        
        \noindent Figure S\arabic{fignum}: Log-likelihood of the CarHHMM-DFT as a function of the ``wiggliness" parameters $\mu_W^{*(\cdot,i^*)}$ and $\sigma_W^{*(\cdot,i^*)}$ for subdive types $i^* = 1,2,3$. All other parameters and transition probability matrices are set to be equal to the maximum likelihood estimates shown in Table 1 of the main text. The log-likelihood is added to the negative log-likelihood of the maximum likelihood estimate $(\hat \mu_W^{*(\cdot,i^*)}, \hat \sigma_W^{*(\cdot,i^*)})$, which is denoted by red stars.
        \addtocounter{fignum}{1}
        
        \newpage
        
        % Ax
        \begin{center}
        \includegraphics[width=2.1in]{../Plots/2019/20190902-182840-CATs_OB_1_0_267_CarHHMM2_fine-theta-likelihood-Ax-0.png}
        \includegraphics[width=2.1in]{../Plots/2019/20190902-182840-CATs_OB_1_0_267_CarHHMM2_fine-theta-likelihood-Ax-1.png}
        \includegraphics[width=2.1in]{../Plots/2019/20190902-182840-CATs_OB_1_0_267_CarHHMM2_fine-theta-likelihood-Ax-2.png}
        \end{center}
        
        \noindent Figure S\arabic{fignum}: Log-likelihood of the CarHHMM-DFT as a function of the $x$-acceleration parameters $\mu_{A_x}^{*(\cdot,i^*)}, \sigma_{A_x}^{*(\cdot,i^*)}$ and the logit of $\phi_{A}^{*(\cdot,i^*)}$ for subdive types $i^* = 1,2,3$. All other parameters and transition probability matrices are set to be equal to the maximum likelihood estimates shown in Table 1 of the main text. The log-likelihood is added to the negative log-likelihood of the maximum likelihood estimate $(\hat \mu_{A_x}^{*(\cdot,i^*)}, \hat \sigma_{A_x}^{*(\cdot,i^*)}, \hat \phi_{A}^{*(\cdot,i^*)})$, which is denoted by red stars.
        \addtocounter{fignum}{1}
        
        
        % Ay
        \begin{center}
        \includegraphics[width=2.1in]{../Plots/2019/20190902-182840-CATs_OB_1_0_267_CarHHMM2_fine-theta-likelihood-Ay-0.png}
        \includegraphics[width=2.1in]{../Plots/2019/20190902-182840-CATs_OB_1_0_267_CarHHMM2_fine-theta-likelihood-Ay-1.png}
        \includegraphics[width=2.1in]{../Plots/2019/20190902-182840-CATs_OB_1_0_267_CarHHMM2_fine-theta-likelihood-Ay-2.png}
        \end{center}
        
        \noindent Figure S\arabic{fignum}: Log-likelihood of the CarHHMM-DFT as a function of the $y$-acceleration parameters $\mu_{A_y}^{*(\cdot,i^*)}, \sigma_{A_y}^{*(\cdot,i^*)}$ and the logit of $\phi_{A}^{*(\cdot,i^*)}$ for subdive types $i^* = 1,2,3$. All other parameters and transition probability matrices are set to be equal to the maximum likelihood estimates shown in Table 1 of the main text. The log-likelihood is added to the negative log-likelihood of the maximum likelihood estimate $(\hat \mu_{A_y}^{*(\cdot,i^*)}, \hat \sigma_{A_y}^{*(\cdot,i^*)}, \hat \phi_{A}^{*(\cdot,i^*)})$, which is denoted by red stars.
        \addtocounter{fignum}{1}
        
        % Az
        \begin{center}
        \includegraphics[width=2.1in]{../Plots/2019/20190902-182840-CATs_OB_1_0_267_CarHHMM2_fine-theta-likelihood-Az-0.png}
        \includegraphics[width=2.1in]{../Plots/2019/20190902-182840-CATs_OB_1_0_267_CarHHMM2_fine-theta-likelihood-Az-1.png}
        \includegraphics[width=2.1in]{../Plots/2019/20190902-182840-CATs_OB_1_0_267_CarHHMM2_fine-theta-likelihood-Az-2.png}
        \end{center}
        
        \noindent Figure S\arabic{fignum}: Log-likelihood of the CarHHMM-DFT as a function of the $z$-acceleration parameters $\mu_{A_z}^{*(\cdot,i^*)}, \sigma_{A_z}^{*(\cdot,i^*)}$ and the logit of $\phi_{A}^{*(\cdot,i^*)}$ for subdive types $i^* = 1,2,3$. All other parameters and transition probability matrices are set to be equal to the maximum likelihood estimates shown in Table 1 of the main text. The log-likelihood is added to the negative log-likelihood of the maximum likelihood estimate $(\hat \mu_{A_z}^{*(\cdot,i^*)}, \hat \sigma_{A_z}^{*(\cdot,i^*)}, \hat \phi_{A}^{*(\cdot,i^*)})$, which is denoted by red stars.
        \addtocounter{fignum}{1}
        
        \newpage
        
        % eta1
        \begin{center}
        \includegraphics[width=3in]{../Plots/2019/20190902-182840-CATs_OB_1_0_267_CarHHMM2_fine-gamma-likelihood-0-row_0.png}
        \includegraphics[width=3in]{../Plots/2019/20190902-182840-CATs_OB_1_0_267_CarHHMM2_fine-gamma-likelihood-1-row_0.png}
        \includegraphics[width=3in]{../Plots/2019/20190902-182840-CATs_OB_1_0_267_CarHHMM2_fine-gamma-likelihood-0-row_1.png}
        \includegraphics[width=3in]{../Plots/2019/20190902-182840-CATs_OB_1_0_267_CarHHMM2_fine-gamma-likelihood-1-row_1.png}
        \includegraphics[width=3in]{../Plots/2019/20190902-182840-CATs_OB_1_0_267_CarHHMM2_fine-gamma-likelihood-0-row_2.png}
        \includegraphics[width=3in]{../Plots/2019/20190902-182840-CATs_OB_1_0_267_CarHHMM2_fine-gamma-likelihood-1-row_2.png}
        \end{center}
        
        \noindent Figure S\arabic{fignum}: Log-likelihood of the CarHHMM-DFT as a function of the fine-scale transition probability matrix parameters associated with both dive types ($\Gamma^{*(i)}_{i^*,j^*}$ for dive types $i = 1,2$ and subdive states $i^*,j^* = 1,2,3$). All other parameters and transition probability matrices are set to be equal to the maximum likelihood estimates shown in Table 1 of the main text. The log-likelihood is added to the negative log-likelihood of the maximum likelihood estimate $\hat \Gamma^{*(i)}$, which is denoted by red stars. Note that $\hat \Gamma^{*(i)}_{1,3} \approx \hat \Gamma^{*(i)}_{3,1} \approx 0$: T=this indicates that the whale rarely switches directly between subdive states 1 and 3.
        \addtocounter{fignum}{1}
        
        \newpage    
    
    \section{Parameter Estimates}

        \subsection{CarHHMM-DFT}
        
            %\begin{table}[ht]
            %
            %\caption{Estimates and empirical standard errors of emission parameters for the distributions of dive duration ($Y_t$), acceleration ($\Zone_{t,\tilde t^*}$), and ``wiggliness" ($\Ztwo_{t,\tilde t^*}$) of the killer whale case study data under the full CarHHMM-DFT.}
            \begin{center}
            \scalebox{0.75}{
            \bgroup
            \centering
            \def\arraystretch{1.5}
            \begin{tabular}{ccccc}
                \multirow{2}{*}{Feature}                                                       & \multirow{2}{*}{Dive type / subdive state} & \multicolumn{3}{c}{Parameter estimate}  \\
                                                                                               &                                      & $\hat \mu$    & $\hat \sigma$ & $\hat \phi$   \\ \hline
                \multirow{2}{*}{Dive Duration $(s)$ -- $Y_t$}                                  & 1                                    & $27.342 \pm 0.633$ & $10.961 \pm 0.560$ & ---           \\
                                                                                               & 2                                    & $127.548 \pm 11.341$ & $63.888 \pm 9.032$ & ---           \\ \hline
                \multirow{3}{*}{$x$-Acc. $(m/s^2)$ -- $\left(\Zone_{t,\tilde t^*}\right)_x$}   & 1                                    & $0.449 \pm 0.030$ & $0.039 \pm 0.001$ & $0.968 \pm 0.002$ \\
                                                                                               & 2                                    & $0.210 \pm 0.012$ & $0.096 \pm 0.002$ & $0.829 \pm 0.007$ \\
                                                                                               & 3                                    & $0.232 \pm 0.035$ & $0.296 \pm 0.010$ & $0.607 \pm 0.023$ \\ \hline
                \multirow{3}{*}{$y$-Acc. $(m/s^2)$ -- $\left(\Zone_{t,\tilde t^*}\right)_y$}   & 1                                    & $0.450 \pm 0.038$ & $0.051 \pm 0.001$ & $0.968 \pm 0.002$ \\
                                                                                               & 2                                    & $0.437 \pm 0.012$ & $0.094 \pm 0.002$ & $0.829 \pm 0.007$ \\
                                                                                               & 3                                    & $0.366 \pm 0.042$ & $0.365 \pm 0.012$ & $0.607 \pm 0.023$ \\ \hline
                \multirow{3}{*}{$z$-Acc. $(m/s^2)$ -- $\left(\Zone_{t,\tilde t^*}\right)_z$}   & 1                                    & $-0.691 \pm 0.043$ & $0.058 \pm 0.001$ & $0.968 \pm 0.002$ \\
                                                                                               & 2                                    & $-0.573 \pm 0.014$ & $0.111 \pm 0.002$ & $0.829 \pm 0.007$ \\
                                                                                               & 3                                    & $-0.303 \pm 0.041$ & $0.354 \pm 0.012$ & $0.607 \pm 0.023$ \\ \hline
                \multirow{3}{*}{``Wiggliness" -- $\Ztwo_{t,\tilde t^*}$}                          & 1                                    & $34.015 \pm 0.368$ & $22.986 \pm 0.378$ & ---           \\
                                                                                               & 2                                    & $490.068 \pm 5.584$ & $502.558 \pm 6.776$ & ---           \\
                                                                                               & 3                                    & $9154.156 \pm 220.765$ & $13538.747 \pm 354.281$ & ---           \\ \hline
            \end{tabular}
            \egroup
            }
            \end{center}
            
            \noindent Table S\arabic{tablenum}: Estimates and predicted standard errors of emission parameters for the distributions of dive duration ($Y_t$), acceleration ($\Zone_{t,\tilde t^*}$), and ``wiggliness" ($\Ztwo_{t,\tilde t^*}$) of the killer whale case study data under the full CarHHMM-DFT. Figures following $\pm$ refer to estimated standard errors derived from the inverse of the observed information matrix.
            \addtocounter{tablenum}{1}
    
        \subsection{HHMM-DFT}
        
            \begin{center}
            \scalebox{0.75}{
                \bgroup
                \centering
                \def\arraystretch{1.5}
                \begin{tabular}{ccccc}
                \multirow{2}{*}{Feature}                                                       & \multirow{2}{*}{Dive / Subdive Type} & \multicolumn{3}{c}{Parameter Estimate}          \\
                                                                                               &                                      & $\hat \mu$    & $\hat \sigma$ & $\hat \phi$     \\ \hline
                \multirow{2}{*}{Dive Duration $(s)$ -- $Y_t$}                                  & 1                                    & $26.613 \pm 0.608$ & $10.161 \pm 0.519$ & ---             \\
                                                                                               & 2                                    & $116.785 \pm 10.626$ & $64.712 \pm 8.574$ & ---             \\ \hline
                \multirow{3}{*}{$x$-Acc. $(m/s^2)$ -- $\left(\Zone_{t,\tilde t^*}\right)_x$}   & 1                                    & $0.000 \pm 0.002$ & $0.065 \pm 0.001$ & ---             \\
                                                                                               & 2                                    & $0.085 \pm 0.004$ & $0.175 \pm 0.003$ & ---             \\
                                                                                               & 3                                    & $0.291 \pm 0.020$ & $0.484 \pm 0.014$ & ---             \\ \hline
                \multirow{3}{*}{$y$-Acc. $(m/s^2)$ -- $\left(\Zone_{t,\tilde t^*}\right)_y$}   & 1                                    & $0.365 \pm 0.002$ & $0.068 \pm 0.001$ & ---             \\
                                                                                               & 2                                    & $0.408 \pm 0.002$ & $0.108 \pm 0.002$ & ---             \\
                                                                                               & 3                                    & $0.275 \pm 0.017$ & $0.407 \pm 0.012$ & ---             \\ \hline
                \multirow{3}{*}{$z$-Acc. $(m/s^2)$ -- $\left(\Zone_{t,\tilde t^*}\right)_z$}   & 1                                    & $-0.442 \pm 0.003$ & $0.101 \pm 0.002$ & ---             \\
                                                                                               & 2                                    & $-0.471 \pm 0.003$ & $0.158 \pm 0.002$ & ---             \\
                                                                                               & 3                                    & $-0.298 \pm 0.017$ & $0.415 \pm 0.012$ & ---             \\ \hline
                \multirow{3}{*}{``Wiggliness" -- $\Ztwo_{t,\tilde t^*}$}                           & 1                                    & $43.652 \pm 0.521$ & $33.662 \pm 0.576$ & ---             \\
                                                                                               & 2                                    & $608.019 \pm 6.764$ & $717.285 \pm 9.194$ & ---             \\
                                                                                               & 3                                    & $7101.915 \pm 155.145$ & $13617.196 \pm 312.209$ & ---             \\ \hline
                \end{tabular}
                \egroup
            }
            \end{center}
            
            \noindent Table S\arabic{tablenum}: Estimates and predicted standard errors of emission parameters for the distributions of dive duration ($Y_t$), acceleration ($\Zone_{t,\tilde t^*}$), and ``wiggliness" ($\Ztwo_{t,\tilde t^*}$) of the killer whale case study data under the HHMM-DFT. Figures following $\pm$ refer to estimated standard errors derived from the inverse of the observed information matrix.
            \addtocounter{tablenum}{1}
            
        \subsection{CarHHMM}
            
            \begin{center}
            \scalebox{0.75}{
                \bgroup
                \centering
                \def\arraystretch{1.5}
                \begin{tabular}{ccccc}
                    \multirow{2}{*}{Feature}                                                       & \multirow{2}{*}{Dive / Subdive Type} & \multicolumn{3}{c}{Parameter Estimate}        \\
                                                                                                   &                                      & $\hat \mu$    & $\hat \sigma$ & $\hat \phi$   \\ \hline
                    \multirow{2}{*}{Dive Duration $(s)$ -- $Y_t$}                                  & 1                                    & $27.163 \pm 0.627$ & $10.780 \pm 0.549$ & ---           \\
                                                                                                   & 2                                    & $124.235 \pm 10.965$ & $64.590 \pm 8.816$ & ---           \\ \hline
                    \multirow{3}{*}{$x$-Acc. $(m/s^2)$ -- $\left(\Zone_{t,\tilde t^*}\right)_x$}   & 1                                    & $0.064 \pm 0.007$ & $0.035 \pm 0.001$ & $0.874 \pm 0.008$ \\
                                                                                                   & 2                                    & $0.342 \pm 0.017$ & $0.095 \pm 0.002$ & $0.879 \pm 0.005$ \\
                                                                                                   & 3                                    & $0.257 \pm 0.037$ & $0.305 \pm 0.010$ & $0.605 \pm 0.023$ \\ \hline
                    \multirow{3}{*}{$y$-Acc. $(m/s^2)$ -- $\left(\Zone_{t,\tilde t^*}\right)_y$}   & 1                                    & $0.376 \pm 0.009$ & $0.044 \pm 0.001$ & $0.874 \pm 0.008$ \\
                                                                                                   & 2                                    & $0.470 \pm 0.017$ & $0.097 \pm 0.002$ & $0.879 \pm 0.005$ \\
                                                                                                   & 3                                    & $0.350 \pm 0.045$ & $0.377 \pm 0.013$ & $0.605 \pm 0.023$ \\ \hline
                    \multirow{3}{*}{$z$-Acc. $(m/s^2)$ -- $\left(\Zone_{t,\tilde t^*}\right)_z$}   & 1                                    & $-0.462 \pm 0.011$ & $0.052 \pm 0.001$ & $0.874 \pm 0.008$ \\
                                                                                                   & 2                                    & $-0.660 \pm 0.020$ & $0.113 \pm 0.002$ & $0.879 \pm 0.005$ \\
                                                                                                   & 3                                    & $-0.306 \pm 0.045$ & $0.369 \pm 0.013$ & $0.605 \pm 0.023$ \\ \hline
                \end{tabular}
                \egroup
            }
            \end{center}
            
                \noindent Table S\arabic{tablenum}: Estimates and predicted standard errors of emission parameters for the distributions of dive duration ($Y_t$) and acceleration ($\Zone_{t,\tilde t^*}$) of the killer whale case study data under the CarHHMM. Figures following $\pm$ refer to estimated standard errors derived from the inverse of the observed information matrix.
                \addtocounter{tablenum}{1}
            
        \subsection{CarHMM-DFT}
        
            \begin{center}
            \scalebox{0.75}{
                \bgroup
                \centering
                \def\arraystretch{1.5}
                \begin{tabular}{ccccc}
                    \multirow{2}{*}{Feature}                                                       & \multirow{2}{*}{Dive / Subdive Type} & \multicolumn{3}{c}{Parameter Estimate}        \\
                                                                                                   &                                      & $\hat \mu$    & $\hat \sigma$ & $\hat \phi$   \\ \hline
                    \multirow{1}{*}{Dive Duration $(s)$ -- $Y_t$}                                  & 1                                    & $41.695 \pm 1.244$ & $30.499 \pm 1.220$ & ---           \\ \hline
                    \multirow{3}{*}{$x$-Acc. $(m/s^2)$ -- $\left(\Zone_{t,\tilde t^*}\right)_x$}   & 1                                    & $0.460 \pm 0.030$ & $0.039 \pm 0.001$ & $0.968 \pm 0.002$ \\
                                                                                                   & 2                                    & $0.209 \pm 0.012$ & $0.096 \pm 0.002$ & $0.828 \pm 0.007$ \\
                                                                                                   & 3                                    & $0.232 \pm 0.035$ & $0.296 \pm 0.010$ & $0.607 \pm 0.023$ \\ \hline
                    \multirow{3}{*}{$y$-Acc. $(m/s^2)$ -- $\left(\Zone_{t,\tilde t^*}\right)_y$}   & 1                                    & $0.457 \pm 0.038$ & $0.052 \pm 0.001$ & $0.968 \pm 0.002$ \\
                                                                                                   & 2                                    & $0.436 \pm 0.012$ & $0.094 \pm 0.002$ & $0.828 \pm 0.007$ \\
                                                                                                   & 3                                    & $0.365 \pm 0.042$ & $0.366 \pm 0.012$ & $0.607 \pm 0.023$ \\ \hline
                    \multirow{3}{*}{$z$-Acc. $(m/s^2)$ -- $\left(\Zone_{t,\tilde t^*}\right)_z$}   & 1                                    & $-0.704 \pm 0.044$ & $0.059 \pm 0.001$ & $0.968 \pm 0.002$ \\
                                                                                                   & 2                                    & $-0.571 \pm 0.014$ & $0.111 \pm 0.002$ & $0.828 \pm 0.007$ \\
                                                                                                   & 3                                    & $-0.301 \pm 0.042$ & $0.355 \pm 0.012$ & $0.607 \pm 0.023$ \\ \hline
                    \multirow{3}{*}{``Wiggliness" -- $\Ztwo_{t,\tilde t^*}$}                          & 1                                    & $34.403 \pm 0.371$ & $23.331 \pm 0.381$ & ---           \\
                                                                                                   & 2                                    & $495.033 \pm 5.691$ & $505.423 \pm 6.879$ & ---           \\
                                                                                                   & 3                                    & $9189.109 \pm 223.965$ & $13588.427 \pm 360.082$ & ---           \\ \hline
                \end{tabular}
                \egroup
            }
            \end{center}
            
                \noindent Table S\arabic{tablenum}: Estimates and predicted standard errors of emission parameters for the distributions of dive duration ($Y_t$), acceleration ($\Zone_{t,\tilde t^*}$), and ``wiggliness" ($\Ztwo_{t,\tilde t^*}$) of the killer whale case study data under the CarHMM-DFT. Figures following $\pm$ refer to estimated standard errors derived from the inverse of the observed information matrix.
                \addtocounter{tablenum}{1}
    
    \newpage
    \section{Decoded Dive States}
    
        \subsection{CarHHMM-DFT}
        
        \begin{center}
        \includegraphics[width=6in]{../Plots/2019/20190902-182840-CATs_OB_1_0_267_CarHHMM2_decoded_dives.png}
        \end{center}
        
        \noindent Figure S\arabic{fignum}: The $x$-component of acceleration, $(y^*_{t,t^*})_x$ (top two panels), and dive depth (bottom two panels) of a northern resident killer whale for a sequence of six selected dives. Each panel is partitioned into dives by vertical black lines. Curve colour in the first and third panels corresponds to estimated dive type while curve colour in the second and fourth panels corresponds to estimated subdive state. Both dive type and subdive state are estimated by fitting the CarHHMM-DFT to the data and applying the forward--backward algorithm to determine the hidden state with the highest probability.
        \addtocounter{fignum}{1}
        
        \begin{center}
        \includegraphics[width=6in]{../Plots/2019/20190902-182840-CATs_OB_1_0_267_CarHHMM2_decoded_states.png}
        \end{center}
        
        \noindent Figure S\arabic{fignum}: The probability that each dive is of a particular type or that the whale is in a particular subdive state, given the fitted model. Each panel is partitioned into dives by vertical black lines. Curve colour in the first and third panels corresponds to estimated dive type while curve colour in the second and fourth panels corresponds to estimated subdive state. Both dive type and subdive state are estimated by fitting the CarHHMM-DFT to the data and applying the forward--backward algorithm.
        \addtocounter{fignum}{1}
        
        \subsection{HHMM-DFT}
        
        \begin{center}
        \includegraphics[width=6in]{../Plots/2019/20190902-182840-CATs_OB_1_0_267_HHMM_decoded_dives.png}
        \end{center}
        
        \noindent Figure S\arabic{fignum}: The $x$-component of acceleration, $(y^*_{t,t^*})_x$ (top two panels), and dive depth (bottom two panels) of a northern resident killer whale for a sequence of six selected dives. Each panel is partitioned into dives by vertical black lines. Curve colour in the first and third panels corresponds to estimated dive type while curve colour in the second and fourth panels corresponds to estimated subdive state. Both dive type and subdive state are estimated by fitting the HHMM-DFT to the data and applying the forward--backward algorithm to determine the hidden state with the highest probability.
        \addtocounter{fignum}{1}
        
        \begin{center}
        \includegraphics[width=6in]{../Plots/2019/20190902-182840-CATs_OB_1_0_267_HHMM_decoded_states.png}
        \end{center}
        
        \noindent Figure S\arabic{fignum}: The probability that each dive is of a particular type or that the whale is in a particular subdive state, given the fitted model. Each panel is partitioned into dives by vertical black lines. Curve colour in the first and third panels corresponds to estimated dive type while curve colour in the second and fourth panels corresponds to estimated subdive state. Both dive type and subdive state are estimated by fitting the HHMM-DFT to the data and applying the forward--backward algorithm.
        \addtocounter{fignum}{1}
        
        \subsection{CarHHMM}
        
        \begin{center}
        \includegraphics[width=6in]{../Plots/2019/20190902-182840-CATs_OB_1_0_267_CarHHMM1_decoded_dives.png}
        \end{center}
        
        \noindent Figure S\arabic{fignum}: The $x$-component of acceleration, $(y^*_{t,t^*})_x$ (top two panels), and dive depth (bottom two panels) of a northern resident killer whale for a sequence of six selected dives. Each panel is partitioned into dives by vertical black lines. Curve colour in the first and third panels corresponds to estimated dive type while curve colour in the second and fourth panels corresponds to estimated subdive state. Both dive type and subdive state are estimated by fitting the CarHHMM to the data and applying the forward--backward algorithm to determine the hidden state with the highest probability.
        \addtocounter{fignum}{1}
        
        \begin{center}
        \includegraphics[width=6in]{../Plots/2019/20190902-182840-CATs_OB_1_0_267_CarHHMM1_decoded_states.png}
        \end{center}
        
        \noindent Figure S\arabic{fignum}: The probability that each dive is of a particular type or that the whale is in a particular subdive state, given the fitted model. Each panel is partitioned into dives by vertical black lines. Curve colour in the first and third panels corresponds to estimated dive type while curve colour in the second and fourth panels corresponds to estimated subdive state. Both dive type and subdive state are estimated by fitting the CarHHMM to the data and applying the forward--backward algorithm.
        \addtocounter{fignum}{1}
        
        \subsection{CarHMM-DFT}
        
        \begin{center}
        \includegraphics[width=6in]{../Plots/2019/20190902-182840-CATs_OB_1_0_267_CarHMM_decoded_dives.png}
        \end{center}
        
        \noindent Figure S\arabic{fignum}: The $x$-component of acceleration, $(y^*_{t,t^*})_x$ (top two panels), and dive depth (bottom two panels) of a northern resident killer whale for a sequence of six selected dives. Each panel is partitioned into dives by vertical black lines. Curve colour in the first and third panels corresponds to estimated dive type while curve colour in the second and fourth panels corresponds to estimated subdive state. Both dive type and subdive state are estimated by fitting the CarHMM-DFT to the data and applying the forward--backward algorithm to determine the hidden state with the highest probability.
        \addtocounter{fignum}{1}
        
        \begin{center}
        \includegraphics[width=6in]{../Plots/2019/20190902-182840-CATs_OB_1_0_267_CarHMM_decoded_states.png}
        \end{center}
        
        \noindent Figure S\arabic{fignum}: The probability that each dive is of a particular type or that the whale is in a particular subdive state, given the fitted model. Each panel is partitioned into dives by vertical black lines. Curve colour in the first and third panels corresponds to estimated dive type while curve colour in the second and fourth panels corresponds to estimated subdive state. Both dive type and subdive state are estimated by fitting the CarHMM-DFT to the data and applying the forward--backward algorithm.
        \addtocounter{fignum}{1}
        
    \newpage
    \section{Model Checking: Dive Duration ($Y_t$)}
    
        \subsection{CarHHMM-DFT}
        
        \begin{center}
        \includegraphics[width=2.25in]{../Plots/2019/20190902-182840-CATs_OB_1_0_267_CarHHMM2_empirical_hist_dive_duration.png}
        \includegraphics[width=2.25in]{../Plots/2019/20190902-182840-CATs_OB_1_0_267_CarHHMM2_pseudresids_Dive_Duration.png}
        \end{center}
        
        \noindent Figure S\arabic{fignum}: Empirical histogram (left) and pseudoresidual histogram (right) for dive duration ($Y_{t}$) plotted over the estimated emission distribution and a standard normal density, respectively. Both plots are generated using the fitted CarHHMM-DFT and the killer whale case study data.
        \addtocounter{fignum}{1}
        
        \subsection{HHMM-DFT}
        
        \begin{center}
        \includegraphics[width=2.25in]{../Plots/2019/20190902-182840-CATs_OB_1_0_267_HHMM_empirical_hist_dive_duration.png}
        \includegraphics[width=2.25in]{../Plots/2019/20190902-182840-CATs_OB_1_0_267_HHMM_pseudresids_Dive_Duration.png}
        \end{center}
        
        \noindent Figure S\arabic{fignum}: Empirical histogram (left) and pseudoresidual histogram (right) for dive duration ($Y_{t}$) plotted over the estimated emission distribution and a standard normal density, respectively. Both plots are generated using the fitted HHMM-DFT and the killer whale case study data.
        \addtocounter{fignum}{1}
        
        \subsection{CarHHMM}
        
        \begin{center}
        \includegraphics[width=2.25in]{../Plots/2019/20190902-182840-CATs_OB_1_0_267_CarHHMM1_empirical_hist_dive_duration.png}
        \includegraphics[width=2.25in]{../Plots/2019/20190902-182840-CATs_OB_1_0_267_CarHHMM1_pseudresids_Dive_Duration.png}
        \end{center}
        
        \noindent Figure S\arabic{fignum}: Empirical histogram (left) and pseudoresidual histogram (right) for dive duration ($Y_{t}$) plotted over the estimated emission distribution and a standard normal density, respectively. Both plots are generated using the fitted CarHHMM-DFT and the killer whale case study data.
        \addtocounter{fignum}{1}
        
        \subsection{CarHMM-DFT}
        
        \begin{center}
        \includegraphics[width=2.25in]{../Plots/2019/20190902-182840-CATs_OB_1_0_267_CarHMM_empirical_hist_dive_duration.png}
        \includegraphics[width=2.25in]{../Plots/2019/20190902-182840-CATs_OB_1_0_267_CarHMM_pseudresids_Dive_Duration.png}
        \end{center}
        
        \noindent Figure S\arabic{fignum}: Empirical histogram (left) and pseudoresidual histogram (right) for dive duration ($Y_{t}$) plotted over the estimated emission distribution (left) and a standard normal density (right), respectively. Both plots are generated using the fitted CarHMM-DFT and the killer whale case study data.
        \addtocounter{fignum}{1}
        
    \newpage
    \section{Model Checking: Acceleration ($\Zone_{t,\tilde t^*}$)}
        
        \subsection{CarHHMM-DFT}
        
        \begin{center}
        \includegraphics[width=1.75in]{../Plots/2019/20190902-182840-CATs_OB_1_0_267_CarHHMM2_empirical_hist_Ax_0.png}
        \includegraphics[width=1.75in]{../Plots/2019/20190902-182840-CATs_OB_1_0_267_CarHHMM2_empirical_hist_Ay_0.png}
        \includegraphics[width=1.75in]{../Plots/2019/20190902-182840-CATs_OB_1_0_267_CarHHMM2_empirical_hist_Az_0.png}
        
        \includegraphics[width=1.75in]{../Plots/2019/20190902-182840-CATs_OB_1_0_267_CarHHMM2_empirical_hist_Ax_1.png}
        \includegraphics[width=1.75in]{../Plots/2019/20190902-182840-CATs_OB_1_0_267_CarHHMM2_empirical_hist_Ay_1.png}
        \includegraphics[width=1.75in]{../Plots/2019/20190902-182840-CATs_OB_1_0_267_CarHHMM2_empirical_hist_Az_1.png}
        
        \includegraphics[width=1.75in]{../Plots/2019/20190902-182840-CATs_OB_1_0_267_CarHHMM2_empirical_hist_Ax_2.png}
        \includegraphics[width=1.75in]{../Plots/2019/20190902-182840-CATs_OB_1_0_267_CarHHMM2_empirical_hist_Ay_2.png}
        \includegraphics[width=1.75in]{../Plots/2019/20190902-182840-CATs_OB_1_0_267_CarHHMM2_empirical_hist_Az_2.png}
        
        \includegraphics[width=1.75in]{../Plots/2019/20190902-182840-CATs_OB_1_0_267_CarHHMM2_pseudresids_Ax.png}
        \includegraphics[width=1.75in]{../Plots/2019/20190902-182840-CATs_OB_1_0_267_CarHHMM2_pseudresids_Ay.png}
        \includegraphics[width=1.75in]{../Plots/2019/20190902-182840-CATs_OB_1_0_267_CarHHMM2_pseudresids_Az.png}
        \end{center}
        
        \noindent Figure S\arabic{fignum}: Empirical histograms (top three rows) and pseudoresidual histograms (bottom row) for acceleration ($\Zone_{t,\tilde t^*}$) plotted over the estimated emission distribution and a standard normal density, respectively. Note that the mean acceleration at time $\tilde t^*$ depends on acceleration at time $\tilde t^*-1$, so only the deviation from the conditional mean at each particular time step is plotted. All plots are generated using the fitted CarHHMM-DFT and the killer whale case study data.
        \addtocounter{fignum}{1}
        
        \newpage
        
        \subsection{HHMM-DFT}
        
        \begin{center}
        \includegraphics[width=1.75in]{../Plots/2019/20190902-182840-CATs_OB_1_0_267_HHMM_empirical_hist_Ax_0.png}
        \includegraphics[width=1.75in]{../Plots/2019/20190902-182840-CATs_OB_1_0_267_HHMM_empirical_hist_Ay_0.png}
        \includegraphics[width=1.75in]{../Plots/2019/20190902-182840-CATs_OB_1_0_267_HHMM_empirical_hist_Az_0.png}
        
        \includegraphics[width=1.75in]{../Plots/2019/20190902-182840-CATs_OB_1_0_267_HHMM_empirical_hist_Ax_1.png}
        \includegraphics[width=1.75in]{../Plots/2019/20190902-182840-CATs_OB_1_0_267_HHMM_empirical_hist_Ay_1.png}
        \includegraphics[width=1.75in]{../Plots/2019/20190902-182840-CATs_OB_1_0_267_HHMM_empirical_hist_Az_1.png}
        
        \includegraphics[width=1.75in]{../Plots/2019/20190902-182840-CATs_OB_1_0_267_HHMM_empirical_hist_Ax_2.png}
        \includegraphics[width=1.75in]{../Plots/2019/20190902-182840-CATs_OB_1_0_267_HHMM_empirical_hist_Ay_2.png}
        \includegraphics[width=1.75in]{../Plots/2019/20190902-182840-CATs_OB_1_0_267_HHMM_empirical_hist_Az_2.png}
        
        \includegraphics[width=1.75in]{../Plots/2019/20190902-182840-CATs_OB_1_0_267_HHMM_pseudresids_Ax.png}
        \includegraphics[width=1.75in]{../Plots/2019/20190902-182840-CATs_OB_1_0_267_HHMM_pseudresids_Ay.png}
        \includegraphics[width=1.75in]{../Plots/2019/20190902-182840-CATs_OB_1_0_267_HHMM_pseudresids_Az.png}
        \end{center}
        
        \noindent Figure S\arabic{fignum}: Empirical histograms (top three rows) and pseudoresidual histograms (bottom row) for acceleration ($\Zone_{t,\tilde t^*}$) plotted over the estimated emission distribution and a standard normal density, respectively. All plots are generated using the fitted HHMM-DFT and the killer whale case study data.
        \addtocounter{fignum}{1}
        
        \newpage
        
        \subsection{CarHHMM}
        
        \begin{center}
        \includegraphics[width=1.75in]{../Plots/2019/20190902-182840-CATs_OB_1_0_267_CarHHMM1_empirical_hist_Ax_0.png}
        \includegraphics[width=1.75in]{../Plots/2019/20190902-182840-CATs_OB_1_0_267_CarHHMM1_empirical_hist_Ay_0.png}
        \includegraphics[width=1.75in]{../Plots/2019/20190902-182840-CATs_OB_1_0_267_CarHHMM1_empirical_hist_Az_0.png}
        
        \includegraphics[width=1.75in]{../Plots/2019/20190902-182840-CATs_OB_1_0_267_CarHHMM1_empirical_hist_Ax_1.png}
        \includegraphics[width=1.75in]{../Plots/2019/20190902-182840-CATs_OB_1_0_267_CarHHMM1_empirical_hist_Ay_1.png}
        \includegraphics[width=1.75in]{../Plots/2019/20190902-182840-CATs_OB_1_0_267_CarHHMM1_empirical_hist_Az_1.png}
        
        \includegraphics[width=1.75in]{../Plots/2019/20190902-182840-CATs_OB_1_0_267_CarHHMM1_empirical_hist_Ax_2.png}
        \includegraphics[width=1.75in]{../Plots/2019/20190902-182840-CATs_OB_1_0_267_CarHHMM1_empirical_hist_Ay_2.png}
        \includegraphics[width=1.75in]{../Plots/2019/20190902-182840-CATs_OB_1_0_267_CarHHMM1_empirical_hist_Az_2.png}
        
        \includegraphics[width=1.75in]{../Plots/2019/20190902-182840-CATs_OB_1_0_267_CarHHMM1_pseudresids_Ax.png}
        \includegraphics[width=1.75in]{../Plots/2019/20190902-182840-CATs_OB_1_0_267_CarHHMM1_pseudresids_Ay.png}
        \includegraphics[width=1.75in]{../Plots/2019/20190902-182840-CATs_OB_1_0_267_CarHHMM1_pseudresids_Az.png}
        \end{center}
        
        \noindent Figure S\arabic{fignum}: Empirical histograms (top three rows) and pseudoresidual histograms (bottom row) for acceleration ($\Zone_{t,\tilde t^*}$) plotted over the estimated emission distribution and a standard normal density, respectively. Note that the mean acceleration at time $\tilde t^*$ depends on acceleration at time $\tilde t^*-1$, so only the deviation from the conditional mean at each particular time step is plotted. All plots are generated using the fitted CarHHMM and the killer whale case study data.
        \addtocounter{fignum}{1}
        
        \newpage
        
        \subsection{CarHMM-DFT}
        
        \begin{center}
        \includegraphics[width=1.75in]{../Plots/2019/20190902-182840-CATs_OB_1_0_267_CarHMM_empirical_hist_Ax_0.png}
        \includegraphics[width=1.75in]{../Plots/2019/20190902-182840-CATs_OB_1_0_267_CarHMM_empirical_hist_Ay_0.png}
        \includegraphics[width=1.75in]{../Plots/2019/20190902-182840-CATs_OB_1_0_267_CarHMM_empirical_hist_Az_0.png}
        
        \includegraphics[width=1.75in]{../Plots/2019/20190902-182840-CATs_OB_1_0_267_CarHMM_empirical_hist_Ax_1.png}
        \includegraphics[width=1.75in]{../Plots/2019/20190902-182840-CATs_OB_1_0_267_CarHMM_empirical_hist_Ay_1.png}
        \includegraphics[width=1.75in]{../Plots/2019/20190902-182840-CATs_OB_1_0_267_CarHMM_empirical_hist_Az_1.png}
        
        \includegraphics[width=1.75in]{../Plots/2019/20190902-182840-CATs_OB_1_0_267_CarHMM_empirical_hist_Ax_2.png}
        \includegraphics[width=1.75in]{../Plots/2019/20190902-182840-CATs_OB_1_0_267_CarHMM_empirical_hist_Ay_2.png}
        \includegraphics[width=1.75in]{../Plots/2019/20190902-182840-CATs_OB_1_0_267_CarHMM_empirical_hist_Az_2.png}
        
        \includegraphics[width=1.75in]{../Plots/2019/20190902-182840-CATs_OB_1_0_267_CarHMM_pseudresids_Ax.png}
        \includegraphics[width=1.75in]{../Plots/2019/20190902-182840-CATs_OB_1_0_267_CarHMM_pseudresids_Ay.png}
        \includegraphics[width=1.75in]{../Plots/2019/20190902-182840-CATs_OB_1_0_267_CarHMM_pseudresids_Az.png}
        \end{center}
        
        \noindent Figure S\arabic{fignum}: Empirical histograms (top three rows) and pseudoresidual histograms (bottom row) for acceleration ($\Zone_{t,\tilde t^*}$) plotted over the estimated emission distribution and a standard normal density, respectively. Note that the mean acceleration at time $\tilde t^*$ depends on acceleration at time $\tilde t^*-1$, so only the deviation from the conditional mean at each particular time step is plotted. All plots are generated using the fitted CarHMM-DFT and the killer whale case study data.
        \addtocounter{fignum}{1}
        
    \section{Model Checking: ``Wiggliness" ($\Ztwo_{t,\tilde t^*}$)}
        
        \subsection{CarHHMM-DFT}
        
        \begin{center}
        \includegraphics[width=1.75in]{../Plots/2019/20190902-182840-CATs_OB_1_0_267_CarHHMM2_empirical_hist_ahat_0.png}
        \includegraphics[width=1.75in]{../Plots/2019/20190902-182840-CATs_OB_1_0_267_CarHHMM2_empirical_hist_ahat_1.png}
        \includegraphics[width=1.75in]{../Plots/2019/20190902-182840-CATs_OB_1_0_267_CarHHMM2_empirical_hist_ahat_2.png}
        
        \includegraphics[width=1.75in]{../Plots/2019/20190902-182840-CATs_OB_1_0_267_CarHHMM2_pseudresids_ahat.png}
        \end{center}
        
        \noindent Figure S\arabic{fignum}: Empirical histograms (top row) and pseudoresidual histograms (bottom row) for ``wiggliness" ($\Ztwo_{t,\tilde t^*}$) plotted over the estimated emission distribution and a standard normal density, respectively. All plots are generated using the fitted CarHHMM-DFT and the killer whale case study data.
        \addtocounter{fignum}{1}
        
        \subsection{HHMM-DFT}
        
        \begin{center}
        \includegraphics[width=1.75in]{../Plots/2019/20190902-182840-CATs_OB_1_0_267_HHMM_empirical_hist_ahat_0.png}
        \includegraphics[width=1.75in]{../Plots/2019/20190902-182840-CATs_OB_1_0_267_HHMM_empirical_hist_ahat_1.png}
        \includegraphics[width=1.75in]{../Plots/2019/20190902-182840-CATs_OB_1_0_267_HHMM_empirical_hist_ahat_2.png}
        
        \includegraphics[width=1.75in]{../Plots/2019/20190902-182840-CATs_OB_1_0_267_HHMM_pseudresids_ahat.png}
        \end{center}
        
        \noindent Figure S\arabic{fignum}: Empirical histograms (top row) and pseudoresidual histograms (bottom row) for ``wiggliness" ($\Ztwo_{t,\tilde t^*}$) plotted over the estimated emission distribution and a standard normal density, respectively. All plots are generated using the fitted HHMM-DFT and the killer whale case study data.
        \addtocounter{fignum}{1}
        
        \subsection{CarHHMM}
        
        The CarHHMM does not model ``wiggliness".
        
        \subsection{CarHMM-DFT}
        
        \begin{center}
        \includegraphics[width=1.75in]{../Plots/2019/20190902-182840-CATs_OB_1_0_267_CarHMM_empirical_hist_ahat_0.png}
        \includegraphics[width=1.75in]{../Plots/2019/20190902-182840-CATs_OB_1_0_267_CarHMM_empirical_hist_ahat_1.png}
        \includegraphics[width=1.75in]{../Plots/2019/20190902-182840-CATs_OB_1_0_267_CarHMM_empirical_hist_ahat_2.png}
        
        \includegraphics[width=1.75in]{../Plots/2019/20190902-182840-CATs_OB_1_0_267_CarHMM_pseudresids_ahat.png}
        \end{center}
        
        \noindent Figure S\arabic{fignum}: Empirical histograms (top row) and pseudoresidual histograms (bottom row) for ``wiggliness" ($\Ztwo_{t,\tilde t^*}$) plotted over the estimated emission distribution and a standard normal density, respectively. All plots are generated using the fitted CarHMM-DFT and the killer whale case study data.
        \addtocounter{fignum}{1}
        
\end{document}