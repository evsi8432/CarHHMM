% Initial and revised submissions should be 12 point; this will be removed in the final version.
\documentclass[12pt]{TD-CJS}

% Initial and revised submissions should also be double spaced.  This command will be removed in the final version.
\renewcommand{\baselinestretch}{2}

\usepackage{latexsym}
\usepackage{amsmath}
\usepackage{amsfonts}
\usepackage{amssymb}
\usepackage{psfrag}
\usepackage{graphicx}
\usepackage[dvipsnames]{xcolor}
\usepackage{url}
% header.tex
% this is where you load pacakges, specify custom formats, etc.

% \usepackage{changepage}
\usepackage{amsmath,amsthm,amssymb,amsfonts}
\usepackage{mathtools}
\usepackage{bbm}
% enumitem for custom lists
\usepackage{enumitem}
% Load dsfont this to get proper indicator function (bold 1) with \mathds{1}:
\usepackage{dsfont}
\usepackage{centernot}
\usepackage{appendix}

% set up graphics
\usepackage{graphicx}
\DeclareGraphicsExtensions{.pdf,.png,.jpg}
\graphicspath{ {fig/} }
% defs.tex
% this is where you define custom notation, commands, etc.

\DeclareMathOperator*{\argmax}{arg\,max}
\DeclareMathOperator*{\argmin}{arg\,min}
\DeclareMathOperator*{\del}{\nabla}

%%
% full alphabets of different styles
%%

% bf series
\def\bfA{\mathbf{A}}
\def\bfB{\mathbf{B}}
\def\bfC{\mathbf{C}}
\def\bfD{\mathbf{D}}
\def\bfE{\mathbf{E}}
\def\bfF{\mathbf{F}}
\def\bfG{\mathbf{G}}
\def\bfH{\mathbf{H}}
\def\bfI{\mathbf{I}}
\def\bfJ{\mathbf{J}}
\def\bfK{\mathbf{K}}
\def\bfL{\mathbf{L}}
\def\bfM{\mathbf{M}}
\def\bfN{\mathbf{N}}
\def\bfO{\mathbf{O}}
\def\bfP{\mathbf{P}}
\def\bfQ{\mathbf{Q}}
\def\bfR{\mathbf{R}}
\def\bfS{\mathbf{S}}
\def\bfT{\mathbf{T}}
\def\bfU{\mathbf{U}}
\def\bfV{\mathbf{V}}
\def\bfW{\mathbf{W}}
\def\bfX{\mathbf{X}}
\def\bfY{\mathbf{Y}}
\def\bfZ{\mathbf{Z}}

% bb series
\def\bbA{\mathbb{A}}
\def\bbB{\mathbb{B}}
\def\bbC{\mathbb{C}}
\def\bbD{\mathbb{D}}
\def\bbE{\mathbb{E}}
\def\bbF{\mathbb{F}}
\def\bbG{\mathbb{G}}
\def\bbH{\mathbb{H}}
\def\bbI{\mathbb{I}}
\def\bbJ{\mathbb{J}}
\def\bbK{\mathbb{K}}
\def\bbL{\mathbb{L}}
\def\bbM{\mathbb{M}}
\def\bbN{\mathbb{N}}
\def\bbO{\mathbb{O}}
\def\bbP{\mathbb{P}}
\def\bbQ{\mathbb{Q}}
\def\bbR{\mathbb{R}}
\def\bbS{\mathbb{S}}
\def\bbT{\mathbb{T}}
\def\bbU{\mathbb{U}}
\def\bbV{\mathbb{V}}
\def\bbW{\mathbb{W}}
\def\bbX{\mathbb{X}}
\def\bbY{\mathbb{Y}}
\def\bbZ{\mathbb{Z}}

% cal series
\def\calA{\mathcal{A}}
\def\calB{\mathcal{B}}
\def\calC{\mathcal{C}}
\def\calD{\mathcal{D}}
\def\calE{\mathcal{E}}
\def\calF{\mathcal{F}}
\def\calG{\mathcal{G}}
\def\calH{\mathcal{H}}
\def\calI{\mathcal{I}}
\def\calJ{\mathcal{J}}
\def\calK{\mathcal{K}}
\def\calL{\mathcal{L}}
\def\calM{\mathcal{M}}
\def\calN{\mathcal{N}}
\def\calO{\mathcal{O}}
\def\calP{\mathcal{P}}
\def\calQ{\mathcal{Q}}
\def\calR{\mathcal{R}}
\def\calS{\mathcal{S}}
\def\calT{\mathcal{T}}
\def\calU{\mathcal{U}}
\def\calV{\mathcal{V}}
\def\calW{\mathcal{W}}
\def\calX{\mathcal{X}}
\def\calY{\mathcal{Y}}
\def\calZ{\mathcal{Z}}

\def\bfTheta{\mathbf{\Theta}}


%%%%%%%%%%%%%%%%%%%%%%%%%%%%%%%%%%%%%%%%%%%%%%%%%%%%%%%%%%
% text short-cuts
\def\iid{i.i.d.\ } %i.i.d.
\def\ie{i.e.\ }
\def\eg{e.g.\ }
\def\Polya{P\'{o}lya\ }
%%%%%%%%%%%%%%%%%%%%%%%%%%%%%%%%%%%%%%%%%%%%%%%%%%%%%%%%%%

%%%%%%%%%%%%%%%%%%%%%%%%%%%%%%%%%%%%%%%%%%%%%%%%%%%%%%%%%%
% quasi-universal probabilistic and mathematical notation
% my preferences (modulo publication conventions, and clashes like random vectors):
%   vectors: bold, lowercase
%   matrices: bold, uppercase
%   operators: blackboard (e.g., \mathbb{E}), uppercase
%   sets, spaces: calligraphic, uppercase
%   random variables: normal font, uppercase
%   deterministic quantities: normal font, lowercase
%%%%%%%%%%%%%%%%%%%%%%%%%%%%%%%%%%%%%%%%%%%%%%%%%%%%%%%%%%

% operators
\def\P{\bbP} %fundamental probability
\def\E{\bbE} %expectation
% conditional expectation
\DeclarePairedDelimiterX\bigCond[2]{[}{]}{#1 \;\delimsize\vert\; #2}
\newcommand{\conditional}[3][]{\bbE_{#1}\bigCond*{#2}{#3}}
\def\Law{\mathcal{L}} %law; this is by convention in the literature
\def\indicator{\mathds{1}} % indicator function

% sets and groups
\def\borel{\calB} %Borel sets
\def\sigAlg{\calA} %sigma-algebra
\def\filtration{\calF} %filtration
\def\grp{\calG} %group

% binary relations
\def\condind{{\perp\!\!\!\perp}} %independence/conditional independence
\def\equdist{\stackrel{\text{\rm\tiny d}}{=}} %equal in distribution
\def\equas{\stackrel{\text{\rm\tiny a.s.}}{=}} %euqal amost surely
\def\simiid{\sim_{\mbox{\tiny iid}}} %sampled i.i.d

% common vectors and matrices
\def\onevec{\mathbf{1}}
\def\iden{\mathbf{I}} % identity matrix
\def\supp{\text{\rm supp}}

% misc
% floor and ceiling
\DeclarePairedDelimiter{\ceilpair}{\lceil}{\rceil}
\DeclarePairedDelimiter{\floor}{\lfloor}{\rfloor}
\newcommand{\argdot}{{\,\vcenter{\hbox{\tiny$\bullet$}}\,}} %generic argument dot
%%%%%%%%%%%%%%%%%%%%%%%%%%%%%%%%%%%%%%%%%%%%%%%%%%%%%%%%%%

%%%%%%%%%%%%%%%%%%%%%%%%%%%%%%%%%%%%%%%%%%%%%%%%%%%%%%%%%%
%% some distributions
% continuous
\def\UnifDist{\text{\rm Unif}}
\def\BetaDist{\text{\rm Beta}}
\def\ExpDist{\text{\rm Exp}}
\def\GammaDist{\text{\rm Gamma}}
% \def\GenGammaDist{\text{\rm GGa}} %Generalized Gamma

% discrete
\def\BernDist{\text{\rm Bernoulli}}
\def\BinomDist{\text{\rm Binomial}}
\def\PoissonPlus{\text{\rm Poisson}_{+}}
\def\PoissonDist{\text{\rm Poisson}}
\def\NBPlus{\text{\rm NB}_{+}}
\def\NBDist{\text{\rm NB}}
\def\GeomDist{\text{\rm Geom}}
% \def\CRP{\text{\rm CRP}}
% \def\EGP{\text{\rm EGP}}
% \def\MittagLeffler{\text{\rm ML}}
%%%%%%%%%%%%%%%%%%%%%%%%%%%%%%%%%%%%%%%%%%%%%%%%%%%%%%%%%%

%%%%%%%%%%%%%%%%%%%%%%%%%%%%%%%%%%%%%%%%%%%%%%%%%%%%%%%%%%
% Project-specific notation should go here
% (Because it's at the end of the file, it can overwrite anything that came before.)

%e.g.,
\def\Laplacian{\calL}
\def\P{\calP}

% combinatorial objects
\def\perm{\sigma} %fixed permutation
\def\Perm{\Sigma} %random permutation
\def\part{\pi} %fixed partition
\def\Part{\Pi} %random partition


%%%%%%%%%%%%%%%%%%%%%%%%%%%%%%%%%%%%%%%%%%%%%%%%%%%%%%%%%%

\begin{document}
\firstpage{1}
\lastpage{25}
\jvol{xx}
\issue{yy}
\jyear{2020}
\jid{CJS}
\aid{???}
% The running head contains the author names
\rhauthor{BlindedA and BlindedB}
\copyrightline{Statistical Society of Canada}
\Frenchcopyrightline{Soci\'et\'e statistique du Canada}
% History: received and accepted dates
\received{\rec{9}{July}{2009}}
\accepted{\acc{8}{July}{2010}}

% User-defined commands go here
\renewcommand{\eqref}[1]{(\ref{#1})}
\newcommand{\mb}[1]{\mathbf{#1}}
\newcommand{\mbb}[1]{\mathbb{#1}}
\newcommand{\mt}[1]{\mathrm{#1}}
\newcommand{\rv}{random variable}
\newcommand{\newblock}{}
\bibliographystyle{abbrvnat}

% Title, authors, affiliations
\title[]{Modelling temporal dependence and multilevel structure using hidden Markov models}%\query{Q1}
\author{BlindedA\authorref{1}\thanksref{*}}
\author{BlindedB\authorref{2}}
\affiliation[1]{Author affiliations will go here in the accepted manuscript, 
but do NOT include them in your initial submission because it must be anonymous.}
\affiliation[2]{Second Affiliation}

% Abstract, keywords, and classification codes
\startabstract{%
\keywords{
\KWDtitle{Key words and phrases}
Association parameters\sep clustered data\sep mean parameters\sep missing data\sep pairwise
likelihood\sep repeated measurements.
% MSC 2010 subject classification codes
\KWDtitle{MSC 2010}Primary 62???\sep secondary 62???}
\begin{abstract}
\abstractsection{}
\abstractsection{Abstract}
%Insert your abstract here; it should
%typically be up to ten lines long. Avoid symbols as much as possible. Formulas
%are strongly discouraged, and citations should be avoided. 
%The title and the abstract should be concise and descriptive;
%list the key words in alphabetical order.  The MSC 2010 subject classification codes can be found here:
%http://www.ams.org/mathscinet/msc/pdfs/classifications2010.pdf.

Advances in tagging technology have made a vast amount of high-frequency time series data available in several fields. These data often exhibit complicated temporal dependence structure which cannot be modelled using common methods of Functional Data Analysis (FDA). The field of animal movement suggests a hierarchical approach using hidden Markov models (HMMs), but many intricate fine-scale processes violate basic assumptions of HMMs.
We detail a generalization of the hierarchical HMM approach that incorporates a wider class of parametric models at problematic fine scales. 
%We also provide several examples of viable fine-scale models.
As a case study, we build a hierarchical model that combines an HMM with a moving-window auto-regressive process to describe the movement of a killer whale off the western coast of Canada. 
Our model outputs more interpretable results as well as more accurate parameter and standard error estimates compared to existing models.

\abscopyright
\fabstractsection{}
\fabstractsection{R\'{e}sum\'{e}}
Ins\'{e}rer votre r\'{e}sum\'{e}
ici. We will supply a French abstract for those authors who
can't prepare it themselves.\Frenchabscopyright
\end{abstract}}
\makechaptertitle

% Email address for corresponding author
%\correspondingauthor[*]{\\\email{}} % only put this in after acceptance

%%%%%%%%%%%%
%\vfill\eject
%{\centerline{\Large{Notes from Nancy}}}
%
I've replaced your Background section with my Models and Parameter Estimation (NH) section, borrowing heavily from your writing.

Here are some comments and explanations.  

General things (to always keep in mind - I may add to this list of rules as we go along)
\begin{enumerate}
\item When I recompile in overleaf, I see there are 21 warnings (in the symbol next to the recompile button).  It looks like these are from sections you have written, mostly from problems with newblock.  Please go through these since Latex may be making decisions that you don't like.  It's always good to go through the Latex errors and warnings. 
    \item Don't begin a sentence with a symbol, e.g.~don't write ``$\Theta$ is an important parameter''. 
    It's better to write ``The parameter $\Theta$ is important''.
    \item If you start a section with a subsection, there needs to be some text before the subsection starts.  This text can be a summary of what is in the section.  I see some writers don't follow my rule!  But it bugs me to have a subsection just start up with no explanation of the section.
    \item  Journal articles do not have as many things in math display mode as a thesis or a report.  This is to save space - printing is expensive.  Oh, there is no printing these days, or very little!  But it is the journal style that developed during the print era.  Of course, if you refer to an equation, it is displayed and numbered. A complicated equation (lots of fractions or layers with subscripts, long expressions) should be displayed since it would be hard to read in text mode, where you aren't allowed to use things like frac for fraction. 
    Deciding whether to display or not is always a judgement call.
    In the end, the copy editor will make decisions, but it's good to have the submission somewhat in a journal style.  
    \item Bibliography things (not crucial at this point but must be done before submission):  these issues are common in all stat journals, not just CJS.  Biology journals tend to use numbers and list papers as they appear in the article.  This is not the case in stat journals.
    \begin{enumerate}
        \item 
    Make sure you have caps where you need them - in your bib file,  markov needs to be \{M\}arkov, for instance, to force capitalization.
    \item You have a bib-style that cites in the text via a number, as opposed to via author (data).   CJS uses author (date).  Also,  CJS doesn't use full first name of authors.  See the sample.pdf document that comes with the CJS template.
    \end{enumerate}
    \item From CJS sample.pdf: ``Try to avoid double subscripts, and never use triple subscripts.''  I think you're good -- no triples, but you do have doubles.  We can take a look at those at some point.   This is a common expectation in stat journals.
    \item From CJS sample.pdf: ``Unless central or essential to the flow of the discussion, mathematical arguments should be deferred to the Appendix.''  So we'll see about that theorem you have.  This is getting more common in stat journals - but it's something you need to check.
    \item From CJS sample.pdf: ``When you submit the final version of your manuscript in Latex form, please include postscript files (.ps or .eps) for the figures, labelling them fig1.ps, fig2.ps, etc.''  That's a direction for the final submission, after review and acceptance. For this submission, you probably just submit one pdf file (I haven't checked).  If so, I'd put each figure and table on a separate page, not merged into text, at the end of the pdf document.
    \item Vector notation - boldface or not?  matrix notation - caligraphy or not?   I find it easier to keep things straight if vectors are in bold, matrices in cal.   It looks like CJS has no specific rules (some journals do) but seems to allow this (I looked at some published articles). I suggest using bold and caligraphy, but using newcommand in case we want to switch the notation.    What do you think?
\end{enumerate}

Things specific to this version.
\begin{enumerate}
    \item I have commented out the section title lines in your subfiles (the ones you input) and placed the section title lines in main.tex.  This way, Marie and I can see what sections you're planning to include.  But we can also see the titles in the input files, so we don't get confused.
    \item You have done a great job with the notation. I see that is not a simple task. 
    \item
    Some of the write-up can be a little cleaner, technically, and I think some of the explanatory material, although nice, can distract from the technical specifications of the model.  It's good to put the explanatory stuff in the introduction.   See my comments in the introductory section (which I've written in main.tex).
    \begin{enumerate}
    \item Typically there is less detail in a journal article than in, say, a comprehensive proposal.  For the journal article, you can assume that the reader either already knows about things like HMMs or can get enough of an idea from what you have said (possibly supplementing knowledge from your references).   But it is useful to lay out the notation, as you have done. So ... there is a balance!
   \item
     In the end, you consider three models: CarHMM, HHMM, CarHHMM.  So these need to be sharply defined, so the reader can glance back and skim. 
     \item fine-scale, finescale, fine scale.  I keep using different ones!  I think if you use it as a noun, then it is fine scale.  ``The Markov chain is on a fine scale.''  I think if you use it as an adjective, it is fine-scale, as in ``that is a fine-scale model''. 
     What do you think?
     \item HMMs or HMM's,  $X_t$s or $X_t$'s??  I don't know, and I am inconsistent.  I think either is acceptable.  Which do you prefer?
\end{enumerate}
\end{enumerate}



%\vfill\eject
%%%%%%%%%%%%%
%
\section{Introduction}
\input Sections/intro.tex
%
\section{Models and parameter estimation}
\label{sec:models}
\input Sections/models.tex
%
\section{Data Collection and Model Construction}
\label{sec:data}
\input Sections/data.tex
%
\section{Simulation Study}
\label{sec:sim_study}
\input Sections/simulation.tex
%
\section{Case Study Results}
\label{sec:case_study}
\input Sections/results.tex
%%
\section{Discussion}
\input Sections/discussion.tex
%
\newpage
\bibliography{references}
%
\newpage
\begin{appendix}
\input Sections/appendix.tex
\end{appendix}
%
\newpage
\section{Figures and Tables}
\input Sections/figures.tex
%
%\CJShistory

\end{document}
