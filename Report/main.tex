\documentclass[10pt]{article}
% header.tex
% this is where you load pacakges, specify custom formats, etc.

\usepackage[left=0.5in,right=0.5in,top=1in,footskip=25pt]{geometry} 
% \usepackage{changepage}
\usepackage{amsmath,amsthm,amssymb,amsfonts}
\usepackage{mathtools}
\usepackage{bbm}
% enumitem for custom lists
\usepackage{enumitem}
% Load dsfont this to get proper indicator function (bold 1) with \mathds{1}:
\usepackage{dsfont}
\usepackage{centernot}

\usepackage[usenames,dvipsnames]{xcolor}

% set up commenting code (I will use this during marking)
\definecolor{CommentColor}{rgb}{0,.50,.50}
\newcounter{margincounter}
\newcommand{\displaycounter}{{\arabic{margincounter}}}
\newcommand{\incdisplaycounter}{{\stepcounter{margincounter}\arabic{margincounter}}}
\newcommand{\COMMENT}[1]{\textcolor{CommentColor}{$\,^{(\incdisplaycounter)}$}\marginpar{\scriptsize\textcolor{CommentColor}{ {\tiny $(\displaycounter)$} #1}}}

\usepackage{appendix}

% set up graphics
\usepackage{graphicx}
\DeclareGraphicsExtensions{.pdf,.png,.jpg}
\graphicspath{ {fig/} }
\input{defs}

%%%%%%%%%%%%%%%%%%%%%%%%%%%%%%%%%%%%%%%%%%%%%%%%%%%%%%%%%%%%%%%%%%%%%%%%%%%%%%%%%

\title{Inferring Fine Scale Behaviors Within Hierarchical Hidden Markov Models}
\author{Evan Sidrow}
\date{\today}

\begin{document}

\maketitle

\abstract

In the field of animal movement, recent advances in high-frequency tagging technology have made available a vast amount of data which can exhibit simultaneous behavioral processes occurring at different time scales. One way to model this data is to use a hierarchical hidden Markov model (HHMM), where the system is modeled as a nested structure of hidden Markov models (HMMs). At very short time scales, however, observations can exhibit complicated dependence structures which cannot be easily captured by traditional HMMs. This work investigates how to incorporate fine-scale processes into the larger structure of HHMMs while maintaining computational efficiency. We apply our method to dive data collected from a northern resident killer whale off the coast of British Columbia, Canada.

The field of animal movement is in the midst of a ``data renaissance" where advancements in tagging technology have given rise to an explosion of data available for statistical modeling. In particular, tagging technologies are capable of recording observations at rates of tens of hertz, resulting in time series containing millions of observations over the course of several hours. This results in a vast amount of data which often exhibits many different simultaneous behavioral processes occurring at different time scales. 

One solution to this issue is to use a hierarchical hidden Markov model (HHMM). HHMMs model the entire system as a nested structure of hidden Markov models (HMM) where each HMM corresponds to one behavioral process. One nice property of HHMMs is that its likelihood is relatively easy to compute, facilitating fast maximum likelihood estimates for its associated parameters.  

At the shortest time scales, however, observations often exhibit complicated dependence structures which cannot be easily captured by a traditional HMMs. To address this issue, it is possible to model small-scale animal behavior as the solution to some stochastic differential equation, but these methods tend to be computationally intractable and require approximate inference techniques such as Markov-chain Monte Carlo (MCMC).

This work investigates how to incorporate fine-scale processes into the larger structure of hierarchical hidden Markov models while maintaining computational efficiency. We bridge the gap between discrete hidden Markov models and continuous-time stochastic process models by showing that the two are equivalent under certain conditions. In addition, we extract features from highly structured sub-dive behaviors that otherwise could not be modeled with a simple HMM. Finally, we apply our method to dive data collected from a Northern resident killer whale off the coast of British Columbia, Canada.

%\input Sections/intro.tex
\input Sections/HHMM.tex
\input Sections/carHMM.tex
\input Sections/fourier.tex
\input Sections/simulation.tex
%\input Sections/results.tex
%\input Sections/conclusion.tex

\newpage

\bibliographystyle{plain}
\bibliography{references}

\newpage

\input Sections/appendix.tex

\end{document}
