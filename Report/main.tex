% Initial and revised submissions should be 12 point; this will be removed in the final version.
\documentclass[12pt]{TD-CJS}

% Initial and revised submissions should also be double spaced.  This command will be removed in the final version.
\renewcommand{\baselinestretch}{2}

\usepackage{latexsym}
\usepackage{amsmath}
\usepackage{amsfonts}
\usepackage{amssymb}
\usepackage{psfrag}
\usepackage{graphicx}

% header.tex
% this is where you load pacakges, specify custom formats, etc.

\usepackage[left=0.5in,right=0.5in,top=1in,footskip=25pt]{geometry} 
% \usepackage{changepage}
\usepackage{amsmath,amsthm,amssymb,amsfonts}
\usepackage{mathtools}
\usepackage{bbm}
% enumitem for custom lists
\usepackage{enumitem}
% Load dsfont this to get proper indicator function (bold 1) with \mathds{1}:
\usepackage{dsfont}
\usepackage{centernot}

\usepackage[usenames,dvipsnames]{xcolor}

% set up commenting code (I will use this during marking)
\definecolor{CommentColor}{rgb}{0,.50,.50}
\newcounter{margincounter}
\newcommand{\displaycounter}{{\arabic{margincounter}}}
\newcommand{\incdisplaycounter}{{\stepcounter{margincounter}\arabic{margincounter}}}
\newcommand{\COMMENT}[1]{\textcolor{CommentColor}{$\,^{(\incdisplaycounter)}$}\marginpar{\scriptsize\textcolor{CommentColor}{ {\tiny $(\displaycounter)$} #1}}}

\usepackage{appendix}

% set up graphics
\usepackage{graphicx}
\DeclareGraphicsExtensions{.pdf,.png,.jpg}
\graphicspath{ {fig/} }
\input{defs}

\begin{document}
\firstpage{1}
\lastpage{25}
\jvol{xx}
\issue{yy}
\jyear{2020}
\jid{CJS}
\aid{???}
% The running head contains the author names
\rhauthor{BlindedA and BlindedB}
\copyrightline{Statistical Society of Canada}
\Frenchcopyrightline{Soci\'et\'e statistique du Canada}
% History: received and accepted dates
\received{\rec{9}{July}{2009}}
\accepted{\acc{8}{July}{2010}}

% User-defined commands go here
\renewcommand{\eqref}[1]{(\ref{#1})}
\newcommand{\mb}[1]{\mathbf{#1}}
\newcommand{\mbb}[1]{\mathbb{#1}}
\newcommand{\mt}[1]{\mathrm{#1}}
\newcommand{\rv}{random variable}

% Title, authors, affiliations
\title[]{Modelling fine-scale dive behavior with hidden Markov models }%\query{Q1}
\author{BlindedA\authorref{1}\thanksref{*}}
\author{BlindedB\authorref{2}}
\affiliation[1]{Author affiliations will go here in the accepted manuscript, 
but do NOT include them in your initial submission because it must be anonymous.}
\affiliation[2]{Second Affiliation}

% Abstract, keywords, and classification codes
\startabstract{%
\keywords{
\KWDtitle{Key words and phrases}
Association parameters\sep clustered data\sep mean parameters\sep missing data\sep pairwise
likelihood\sep repeated measurements.
% MSC 2010 subject classification codes
\KWDtitle{MSC 2010}Primary 62???\sep secondary 62???}
\begin{abstract}
\abstractsection{}
\abstractsection{Abstract}
Insert your abstract here; it should
typically be up to ten lines long. Avoid symbols as much as possible. Formulas
are strongly discouraged, and citations should be avoided. 
The title and the abstract should be concise and descriptive;
list the key words in alphabetical order.  The MSC 2010 subject classification codes can be found here:
http://www.ams.org/mathscinet/msc/pdfs/classifications2010.pdf.

The field of animal movement is in the midst of a ``data renaissance" where advancements in tagging technology have given rise to an explosion of data available for statistical modeling. In particular, tagging technologies are capable of recording observations at rates of tens of hertz, resulting in time series containing millions of observations over the course of several hours. This results in a vast amount of data which often exhibits many different simultaneous behavioral processes occurring at different time scales. 

One solution to this issue is to use a hierarchical hidden Markov model (HHMM). HHMMs model the entire system as a nested structure of hidden Markov models (HMM) where each HMM corresponds to one behavioral process. One nice property of HHMMs is that its likelihood is relatively easy to compute, facilitating fast maximum likelihood estimates for its associated parameters.  

At the shortest time scales, however, observations often exhibit complicated dependence structures which cannot be easily captured by a traditional HMMs. To address this issue, it is possible to model small-scale animal behavior as the solution to some stochastic differential equation, but these methods tend to be computationally intractable and require approximate inference techniques such as Markov-chain Monte Carlo (MCMC).

This work investigates how to incorporate fine-scale processes into the larger structure of hierarchical hidden Markov models while maintaining computational efficiency. We bridge the gap between discrete hidden Markov models and continuous-time stochastic process models by showing that the two are equivalent under certain conditions. In addition, we extract features from highly structured sub-dive behaviors using signal processing techniques. These features otherwise could not be modeled with a simple HMM. Finally, we apply our method to dive data collected from two Northern resident killer whales off the coast of British Columbia, Canada.

\abscopyright
\fabstractsection{}
\fabstractsection{R\'{e}sum\'{e}}
Ins\'{e}rer votre r\'{e}sum\'{e}
ici. We will supply a French abstract for those authors who
can't prepare it themselves.\Frenchabscopyright
\end{abstract}}
\makechaptertitle

% Email address for corresponding author
\correspondingauthor[*]{\\\email{Insert your email address here only after your paper has been accepted}}

%\input Sections/intro.tex
\input Sections/background.tex
\input Sections/carhmm_ou.tex
\input Sections/model.tex
\input Sections/simulation.tex
\input Sections/results.tex
\input Sections/conclusion.tex

\newpage

\bibliographystyle{plain}
\bibliography{references}

\newpage

\input Sections/appendix.tex

\end{document}
