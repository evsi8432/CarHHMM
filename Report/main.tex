% Initial and revised submissions should be 12 point; this will be removed in the final version.
%\documentclass[12pt]{TD-CJS}
\documentclass{TD-CJS}

% Initial and revised submissions should also be double spaced. This command will be removed in the final version.
%\renewcommand{\baselinestretch}{2}

\usepackage{latexsym}
\usepackage{amsmath}
\usepackage{amsfonts}
\usepackage{amssymb}
\usepackage{psfrag}
\usepackage{graphicx}
\usepackage[dvipsnames]{xcolor}
\usepackage{url}
% header.tex
% this is where you load pacakges, specify custom formats, etc.

\usepackage[left=0.5in,right=0.5in,top=1in,footskip=25pt]{geometry} 
% \usepackage{changepage}
\usepackage{amsmath,amsthm,amssymb,amsfonts}
\usepackage{mathtools}
\usepackage{bbm}
% enumitem for custom lists
\usepackage{enumitem}
% Load dsfont this to get proper indicator function (bold 1) with \mathds{1}:
\usepackage{dsfont}
\usepackage{centernot}

\usepackage[usenames,dvipsnames]{xcolor}

% set up commenting code (I will use this during marking)
\definecolor{CommentColor}{rgb}{0,.50,.50}
\newcounter{margincounter}
\newcommand{\displaycounter}{{\arabic{margincounter}}}
\newcommand{\incdisplaycounter}{{\stepcounter{margincounter}\arabic{margincounter}}}
\newcommand{\COMMENT}[1]{\textcolor{CommentColor}{$\,^{(\incdisplaycounter)}$}\marginpar{\scriptsize\textcolor{CommentColor}{ {\tiny $(\displaycounter)$} #1}}}

\usepackage{appendix}

% set up graphics
\usepackage{graphicx}
\DeclareGraphicsExtensions{.pdf,.png,.jpg}
\graphicspath{ {fig/} }
\input{defs}

\begin{document}
\firstpage{1}
\lastpage{25}
\jvol{xx}
\issue{yy}
\jyear{2020}
\jid{CJS}
\aid{???}
% The running head contains the author names
\rhauthor{BlindedA and BlindedB}
\copyrightline{Statistical Society of Canada}
\Frenchcopyrightline{Soci\'et\'e statistique du Canada}
% History: received and accepted dates
\received{\rec{9}{July}{2009}}
\accepted{\acc{8}{July}{2010}}

% User-defined commands go here
\renewcommand{\eqref}[1]{(\ref{#1})}
\newcommand{\mb}[1]{\mathbf{#1}}
\newcommand{\mbb}[1]{\mathbb{#1}}
\newcommand{\mt}[1]{\mathrm{#1}}
\newcommand{\rv}{random variable}
\newcommand{\newblock}{}
\bibliographystyle{apalike}

% Title, authors, affiliations
%\title[]{Modelling temporal dependence and multilevel structure using hidden Markov models}%\query{Q1}
\title[]{Modelling state-switching functional and biologging data with hidden Markov models}
\author{BlindedA\authorref{1}\thanksref{*}}
\author{BlindedB\authorref{2}}
\affiliation[1]{Author affiliations will go here in the accepted manuscript, 
but do NOT include them in your initial submission because it must be anonymous.}
\affiliation[2]{Second Affiliation}

% Abstract, keywords, and classification codes
\startabstract{%
\keywords{
\KWDtitle{Key words and phrases}
Biologging\sep statistical ecology\sep animal movement\sep accelerometer data\sep diving behaviour\sep killer whales \sep multi-state modelling\sep hierarchical modelling\sep time series.
% MSC 2010 subject classification codes
\KWDtitle{MSC 2010}Primary 62M05 \sep secondary 62P12}
\begin{abstract}
\abstractsection{}
\abstractsection{Abstract}
%Insert your abstract here; it should
%typically be up to ten lines long. Avoid symbols as much as possible. Formulas
%are strongly discouraged, and citations should be avoided. 
%The title and the abstract should be concise and descriptive;
%list the key words in alphabetical order.  The MSC 2010 subject classification codes can be found here:
%http://www.ams.org/mathscinet/msc/pdfs/classifications2010.pdf.

\iffalse
Advances in biologging technology have made a vast amount of high-frequency time series data available in a variety of fields. These data often exhibit complicated temporal dependence structures that cannot be modelled using common methods of Functional Data Analysis (FDA). Animal movement modellers often employ a hierarchical approach using hidden Markov models (HMMs) instead, but many intricate fine-scale processes violate basic assumptions of HMMs. We detail a general hierarchical approach based on HMMs which incorporates a wider class of models at fine scales. As a case study, we build a hierarchical model that combines an HMM with a moving-window auto-regressive process to describe the movement of a killer whale off the western coast of Canada. Our model produces more interpretable results as well as more accurate parameter and standard error estimates compared to existing baselines.
\fi

Data sets comprised of sequences of curves sampled at high frequencies in time are increasingly common in practice, but they can exhibit complicated dependence structures that cannot be modelled using common methods of Functional Data Analysis (FDA). We detail a hierarchical approach which treats the curves as observations from a hidden Markov model (HMM). The distribution of each curve is then defined by another fine-scale model which may involve auto-regression and require data transformations using moving-window summary statistics or Fourier analysis. This approach is broadly applicable to sequences of curves exhibiting intricate dependence structures. As a case study, we use this framework to model the fine-scale kinematic movement of a northern resident killer whale off the western coast of Canada. Through simulations, we show that our model produces more interpretable state estimation and more accurate parameter estimates compared to existing methods.

\abscopyright
\fabstractsection{}
\fabstractsection{R\'{e}sum\'{e}}
Ins\'{e}rer votre r\'{e}sum\'{e}
ici. We will supply a French abstract for those authors who
can't prepare it themselves.\Frenchabscopyright
\end{abstract}}
\makechaptertitle

% Email address for corresponding author
%\correspondingauthor[*]{\\\email{}} % only put this in after acceptance
%
\section{INTRODUCTION}
\input Sections/intro.tex
%
\section{MODELS AND PARAMETER ESTIMATION}
\label{sec:models}
\input Sections/models.tex
%
\section{KILLER WHALE CASE STUDY}
\label{sec:data}
\input Sections/results.tex
%
\section{SIMULATION STUDY}
\label{sec:sim_study}
\input Sections/simulation.tex
%
\section{DISCUSSION}
\input Sections/discussion.tex
%
\begin{ack}{ACKNOWLEDGEMENTS}
% This research was enabled in part by support provided by WestGrid (www.westgrid.ca) and Compute Canada (www.computecanada.ca).
% We acknowledge the support of the Natural Sciences and Engineering Research Council of Canada (NSERC). 
% Marie Auger M'eth'e thanks the NSERC Discovery and the Canadian Research Chair programs.
% Evan Sidrow thanks the University of British Columbia and the Four-Year Doctoral Fellowship program.
\end{ack}
%
\newpage
\bibliography{references}
%
\newpage
\begin{appendix}
%\input Sections/appendix.tex
\input Sections/appendixnh.tex
\end{appendix}
%
\newpage
\section{FIGURES AND TABLES}
\input Sections/figures.tex
%
%\CJShistory

\end{document}
